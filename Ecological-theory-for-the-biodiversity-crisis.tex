% Options for packages loaded elsewhere
\PassOptionsToPackage{unicode}{hyperref}
\PassOptionsToPackage{hyphens}{url}
%
\documentclass[
  letterpaper,
  oneside,
  open=any]{book}

\usepackage{amsmath,amssymb}
\usepackage{iftex}
\ifPDFTeX
  \usepackage[T1]{fontenc}
  \usepackage[utf8]{inputenc}
  \usepackage{textcomp} % provide euro and other symbols
\else % if luatex or xetex
  \usepackage{unicode-math}
  \defaultfontfeatures{Scale=MatchLowercase}
  \defaultfontfeatures[\rmfamily]{Ligatures=TeX,Scale=1}
\fi
\usepackage{lmodern}
\ifPDFTeX\else  
    % xetex/luatex font selection
\fi
% Use upquote if available, for straight quotes in verbatim environments
\IfFileExists{upquote.sty}{\usepackage{upquote}}{}
\IfFileExists{microtype.sty}{% use microtype if available
  \usepackage[]{microtype}
  \UseMicrotypeSet[protrusion]{basicmath} % disable protrusion for tt fonts
}{}
\makeatletter
\@ifundefined{KOMAClassName}{% if non-KOMA class
  \IfFileExists{parskip.sty}{%
    \usepackage{parskip}
  }{% else
    \setlength{\parindent}{0pt}
    \setlength{\parskip}{6pt plus 2pt minus 1pt}}
}{% if KOMA class
  \KOMAoptions{parskip=half}}
\makeatother
\usepackage{xcolor}
\setlength{\emergencystretch}{3em} % prevent overfull lines
\setcounter{secnumdepth}{5}
% Make \paragraph and \subparagraph free-standing
\makeatletter
\ifx\paragraph\undefined\else
  \let\oldparagraph\paragraph
  \renewcommand{\paragraph}{
    \@ifstar
      \xxxParagraphStar
      \xxxParagraphNoStar
  }
  \newcommand{\xxxParagraphStar}[1]{\oldparagraph*{#1}\mbox{}}
  \newcommand{\xxxParagraphNoStar}[1]{\oldparagraph{#1}\mbox{}}
\fi
\ifx\subparagraph\undefined\else
  \let\oldsubparagraph\subparagraph
  \renewcommand{\subparagraph}{
    \@ifstar
      \xxxSubParagraphStar
      \xxxSubParagraphNoStar
  }
  \newcommand{\xxxSubParagraphStar}[1]{\oldsubparagraph*{#1}\mbox{}}
  \newcommand{\xxxSubParagraphNoStar}[1]{\oldsubparagraph{#1}\mbox{}}
\fi
\makeatother
\usepackage{color}
\usepackage{fancyvrb}
\newcommand{\VerbBar}{|}
\newcommand{\VERB}{\Verb[commandchars=\\\{\}]}
\DefineVerbatimEnvironment{Highlighting}{Verbatim}{commandchars=\\\{\}}
% Add ',fontsize=\small' for more characters per line
\usepackage{framed}
\definecolor{shadecolor}{RGB}{241,243,245}
\newenvironment{Shaded}{\begin{snugshade}}{\end{snugshade}}
\newcommand{\AlertTok}[1]{\textcolor[rgb]{0.68,0.00,0.00}{#1}}
\newcommand{\AnnotationTok}[1]{\textcolor[rgb]{0.37,0.37,0.37}{#1}}
\newcommand{\AttributeTok}[1]{\textcolor[rgb]{0.40,0.45,0.13}{#1}}
\newcommand{\BaseNTok}[1]{\textcolor[rgb]{0.68,0.00,0.00}{#1}}
\newcommand{\BuiltInTok}[1]{\textcolor[rgb]{0.00,0.23,0.31}{#1}}
\newcommand{\CharTok}[1]{\textcolor[rgb]{0.13,0.47,0.30}{#1}}
\newcommand{\CommentTok}[1]{\textcolor[rgb]{0.37,0.37,0.37}{#1}}
\newcommand{\CommentVarTok}[1]{\textcolor[rgb]{0.37,0.37,0.37}{\textit{#1}}}
\newcommand{\ConstantTok}[1]{\textcolor[rgb]{0.56,0.35,0.01}{#1}}
\newcommand{\ControlFlowTok}[1]{\textcolor[rgb]{0.00,0.23,0.31}{\textbf{#1}}}
\newcommand{\DataTypeTok}[1]{\textcolor[rgb]{0.68,0.00,0.00}{#1}}
\newcommand{\DecValTok}[1]{\textcolor[rgb]{0.68,0.00,0.00}{#1}}
\newcommand{\DocumentationTok}[1]{\textcolor[rgb]{0.37,0.37,0.37}{\textit{#1}}}
\newcommand{\ErrorTok}[1]{\textcolor[rgb]{0.68,0.00,0.00}{#1}}
\newcommand{\ExtensionTok}[1]{\textcolor[rgb]{0.00,0.23,0.31}{#1}}
\newcommand{\FloatTok}[1]{\textcolor[rgb]{0.68,0.00,0.00}{#1}}
\newcommand{\FunctionTok}[1]{\textcolor[rgb]{0.28,0.35,0.67}{#1}}
\newcommand{\ImportTok}[1]{\textcolor[rgb]{0.00,0.46,0.62}{#1}}
\newcommand{\InformationTok}[1]{\textcolor[rgb]{0.37,0.37,0.37}{#1}}
\newcommand{\KeywordTok}[1]{\textcolor[rgb]{0.00,0.23,0.31}{\textbf{#1}}}
\newcommand{\NormalTok}[1]{\textcolor[rgb]{0.00,0.23,0.31}{#1}}
\newcommand{\OperatorTok}[1]{\textcolor[rgb]{0.37,0.37,0.37}{#1}}
\newcommand{\OtherTok}[1]{\textcolor[rgb]{0.00,0.23,0.31}{#1}}
\newcommand{\PreprocessorTok}[1]{\textcolor[rgb]{0.68,0.00,0.00}{#1}}
\newcommand{\RegionMarkerTok}[1]{\textcolor[rgb]{0.00,0.23,0.31}{#1}}
\newcommand{\SpecialCharTok}[1]{\textcolor[rgb]{0.37,0.37,0.37}{#1}}
\newcommand{\SpecialStringTok}[1]{\textcolor[rgb]{0.13,0.47,0.30}{#1}}
\newcommand{\StringTok}[1]{\textcolor[rgb]{0.13,0.47,0.30}{#1}}
\newcommand{\VariableTok}[1]{\textcolor[rgb]{0.07,0.07,0.07}{#1}}
\newcommand{\VerbatimStringTok}[1]{\textcolor[rgb]{0.13,0.47,0.30}{#1}}
\newcommand{\WarningTok}[1]{\textcolor[rgb]{0.37,0.37,0.37}{\textit{#1}}}

\providecommand{\tightlist}{%
  \setlength{\itemsep}{0pt}\setlength{\parskip}{0pt}}\usepackage{longtable,booktabs,array}
\usepackage{calc} % for calculating minipage widths
% Correct order of tables after \paragraph or \subparagraph
\usepackage{etoolbox}
\makeatletter
\patchcmd\longtable{\par}{\if@noskipsec\mbox{}\fi\par}{}{}
\makeatother
% Allow footnotes in longtable head/foot
\IfFileExists{footnotehyper.sty}{\usepackage{footnotehyper}}{\usepackage{footnote}}
\makesavenoteenv{longtable}
\usepackage{graphicx}
\makeatletter
\def\maxwidth{\ifdim\Gin@nat@width>\linewidth\linewidth\else\Gin@nat@width\fi}
\def\maxheight{\ifdim\Gin@nat@height>\textheight\textheight\else\Gin@nat@height\fi}
\makeatother
% Scale images if necessary, so that they will not overflow the page
% margins by default, and it is still possible to overwrite the defaults
% using explicit options in \includegraphics[width, height, ...]{}
\setkeys{Gin}{width=\maxwidth,height=\maxheight,keepaspectratio}
% Set default figure placement to htbp
\makeatletter
\def\fps@figure{htbp}
\makeatother
% definitions for citeproc citations
\NewDocumentCommand\citeproctext{}{}
\NewDocumentCommand\citeproc{mm}{%
  \begingroup\def\citeproctext{#2}\cite{#1}\endgroup}
\makeatletter
 % allow citations to break across lines
 \let\@cite@ofmt\@firstofone
 % avoid brackets around text for \cite:
 \def\@biblabel#1{}
 \def\@cite#1#2{{#1\if@tempswa , #2\fi}}
\makeatother
\newlength{\cslhangindent}
\setlength{\cslhangindent}{1.5em}
\newlength{\csllabelwidth}
\setlength{\csllabelwidth}{3em}
\newenvironment{CSLReferences}[2] % #1 hanging-indent, #2 entry-spacing
 {\begin{list}{}{%
  \setlength{\itemindent}{0pt}
  \setlength{\leftmargin}{0pt}
  \setlength{\parsep}{0pt}
  % turn on hanging indent if param 1 is 1
  \ifodd #1
   \setlength{\leftmargin}{\cslhangindent}
   \setlength{\itemindent}{-1\cslhangindent}
  \fi
  % set entry spacing
  \setlength{\itemsep}{#2\baselineskip}}}
 {\end{list}}
\usepackage{calc}
\newcommand{\CSLBlock}[1]{\hfill\break\parbox[t]{\linewidth}{\strut\ignorespaces#1\strut}}
\newcommand{\CSLLeftMargin}[1]{\parbox[t]{\csllabelwidth}{\strut#1\strut}}
\newcommand{\CSLRightInline}[1]{\parbox[t]{\linewidth - \csllabelwidth}{\strut#1\strut}}
\newcommand{\CSLIndent}[1]{\hspace{\cslhangindent}#1}

\usepackage{tocloft}
\setlength{\cftbeforetoctitleskip}{-2em}
\renewcommand{\cftaftertoctitle}{%
  \hfill\mbox{}\newline\mbox{}\hfill}
\makeatletter
\patchcmd{\@starttoc}
  {\begingroup}
  {\begingroup\addtocontents{toc}{\protect\thispagestyle{empty}}}
  {}{}
\makeatother
\makeatletter
\@ifpackageloaded{bookmark}{}{\usepackage{bookmark}}
\makeatother
\makeatletter
\@ifpackageloaded{caption}{}{\usepackage{caption}}
\AtBeginDocument{%
\ifdefined\contentsname
  \renewcommand*\contentsname{Table of contents}
\else
  \newcommand\contentsname{Table of contents}
\fi
\ifdefined\listfigurename
  \renewcommand*\listfigurename{List of Figures}
\else
  \newcommand\listfigurename{List of Figures}
\fi
\ifdefined\listtablename
  \renewcommand*\listtablename{List of Tables}
\else
  \newcommand\listtablename{List of Tables}
\fi
\ifdefined\figurename
  \renewcommand*\figurename{Figure}
\else
  \newcommand\figurename{Figure}
\fi
\ifdefined\tablename
  \renewcommand*\tablename{Table}
\else
  \newcommand\tablename{Table}
\fi
}
\@ifpackageloaded{float}{}{\usepackage{float}}
\floatstyle{ruled}
\@ifundefined{c@chapter}{\newfloat{codelisting}{h}{lop}}{\newfloat{codelisting}{h}{lop}[chapter]}
\floatname{codelisting}{Listing}
\newcommand*\listoflistings{\listof{codelisting}{List of Listings}}
\makeatother
\makeatletter
\makeatother
\makeatletter
\@ifpackageloaded{caption}{}{\usepackage{caption}}
\@ifpackageloaded{subcaption}{}{\usepackage{subcaption}}
\makeatother

\usepackage{hyphenat}
\usepackage{ifthen}
\usepackage{calc}
\usepackage{calculator}



\usepackage{graphicx}
\usepackage{geometry}
\usepackage{afterpage}
\usepackage{tikz}
\usetikzlibrary{calc}
\usetikzlibrary{fadings}
\usepackage[pagecolor=none]{pagecolor}


% Set the titlepage font families







% Set the coverpage font families


\ifLuaTeX
  \usepackage{selnolig}  % disable illegal ligatures
\fi
\usepackage{bookmark}

\IfFileExists{xurl.sty}{\usepackage{xurl}}{} % add URL line breaks if available
\urlstyle{same} % disable monospaced font for URLs
\hypersetup{
  pdftitle={Ecological theory for the biodiversity crisis},
  pdfauthor={Henrique Miguel Pereira},
  hidelinks,
  pdfcreator={LaTeX via pandoc}}


\title{Ecological theory for the biodiversity crisis}
\author{Henrique Miguel Pereira}
\date{2025-11-04}

\begin{document}
%%%%% begin titlepage extension code

  \begin{frontmatter}

\begin{titlepage}
% This is a combination of Pandoc templating and LaTeX
% Pandoc templating https://pandoc.org/MANUAL.html#templates
% See the README for help

\thispagestyle{empty}

\newgeometry{top=-100in}

% Page color

\newcommand{\coverauthorstyle}[1]{{\fontsize{30}{36.0}\selectfont
#1}}

\begin{tikzpicture}[remember picture, overlay, inner sep=0pt, outer sep=0pt]

\tikzfading[name=fadeout, inner color=transparent!0,outer color=transparent!100]
\tikzfading[name=fadein, inner color=transparent!100,outer color=transparent!0]
\node[anchor=south west, rotate=0.0, opacity=1.0] at ($(current page.south west)+(0.0, 0.0)$) {
\includegraphics[width=\paperwidth, keepaspectratio]{images/ValeCastroPanoramaBook.jpg}};

% Title
\newcommand{\titlelocationleft}{0.2\paperwidth}
\newcommand{\titlelocationbottom}{0.9\paperheight}
\newcommand{\titlealign}{left}

\begin{scope}{%
\fontsize{50}{60.0}\selectfont
\node[anchor=north
west, align=left, rotate=0] (Title1) at ($(current page.south west)+(\titlelocationleft,\titlelocationbottom)$)  [text width = 0.7\paperwidth]  {{\bfseries{\nohyphens{Ecological
theory for the biodiversity crisis}}}};
}
\end{scope}

% Author
\newcommand{\authorlocationleft}{0.2\paperwidth}
\newcommand{\authorlocationbottom}{0.65\paperheight}
\newcommand{\authoralign}{left}

\begin{scope}
{%
\fontsize{30}{36.0}\selectfont
\node[anchor=north
west, align=left, rotate=0] (Author1) at ($(current page.south west)+(\authorlocationleft,\authorlocationbottom)$)  [text width = 0.7\paperwidth]  {
\coverauthorstyle{Henrique Miguel Pereira\\}};
}
\end{scope}

% Date
\newcommand{\datelocationleft}{0.2\paperwidth}
\newcommand{\datelocationbottom}{0.05\paperheight}
\newcommand{\datelocationalign}{left}

\begin{scope}
{%
 \node[anchor=north west, align=left, rotate=0] (Date1) at %
($(current page.south west)+(\datelocationleft,\datelocationbottom)$)  [text width = 0.7\paperwidth]  {{\nohyphens{2025-11-04}}};
}
\end{scope}

\end{tikzpicture}
\clearpage
\restoregeometry
\end{titlepage}
\setcounter{page}{1}
\end{frontmatter}

%%%%% end titlepage extension code

\renewcommand*\contentsname{Table of contents}
{
\setcounter{tocdepth}{2}
\tableofcontents
}

\mainmatter
\bookmarksetup{startatroot}

\chapter*{Preface}\label{preface}
\addcontentsline{toc}{chapter}{Preface}

\markboth{Preface}{Preface}

We live in the midst of the biodiversity crisis. Biodiversity science,
at first mostly a spin off from Ecology, has developed tremendously in
the last two decades, becoming a truly interdisciplinary field, with
contributions from Geography, Economy, Social Sciences and many other
disciplines. Big data and the development of statistical ecology has
also been remarkable. As I write these lines, the Global Biodiversity
Information Facility or GBIF (\url{https://gbif.org}) is about to hit 3
billion records. The availability of large amounts of data and the
availability of the open source R software
(\url{https://www.r-project.org}) has made statistical approaches take
the main stage in much of ecological research. However, at the same
time, as a biodiversity researcher, practitioner, and a teacher, I find
again and again the need for a good understanding of ecological theory
to be able to carry out my work. This book if for all of you that have
also felt a similar need.

Programming and mathematics are key tools of ecological theory. The
programming language I use in this book is R (version 4, available from
\url{https://www.r-project.org}). If you have familiarly with
programming in another computer language, a fast skimming of Appendix A
will be enough to get you going with the book. If you are new to
programming, I recommend you spend a few hours working through the short
course in Appendix A, but you may also wanna consult some web resources
(e.g.~\url{https://ourcodingclub.github.io/course}). With the advent of
Large Language Models (LLM) such as ChatGPT there may be the temptation
to think programming knowledge will no longer be needed. I think LLMs
will accelerate our writing of computer code, but they will not replace
the need for being able to read and understand the code.

You may wonder why I chose \textbf{R}. There are plenty other
programming languages out there, several suitable for ecological theory
and modelling (e.g.~data scientists love Pyhton). However, \textbf{R}
has become the \emph{de facto} language of ecologists with dozens of
packages\footnote{An R package is a set of functions developed by
  someone to extend the basic functionality of R.} for a wide range of
analysis, from camera trapping to species richness estimation.
\textbf{R} has also powerful packages for spatial data analysis and can
work as a Geographic Information System. So R, grew from a statistics
software (the origins of R are the comercial S-PLUS) to a general
purpose programming language with dedicated packages for different uses.

Why was R so successful, to the point of making some commercial software
companies go out of business? R was open-source and free from its
beginning, part of the free and open-source movement that gained
popularity in the 1990's, particularly with the development of the
operating system Linux. Open-source means that the code behind R (R
itself is a computer program written in C, Fortran, and other languages)
is available for anyone to study. It also allows for anyone to
contribute to improve the code in a collaborative way, including by
contributing with packages. R is also free, meaning that anyone can
download the software or any of its packages without any cost. Isn't
that a wonderful tool for science? Scientists dedicate their life to the
pursue of knowledge, and the idea that someone has to pay for accessing
that knowledge has always been a bit apocryphal. So R allowed scientists
to share their research and code for free in an integrated platform.
Another advantage of open-source software, is that we can in theory ran
programs that we wrote decades ago by downloading archived versions of
the software. At least in theory, the practice of code reproducibility
in science is a bit more complex and requires careful data and code
management by the researcher\footnote{For a guide on best practices for
  reproducible code see
  \url{https://www.britishecologicalsociety.org/wp-content/uploads/2019/06/BES-Guide-Reproducible-Code-2019.pdf}}.
Platforms such as GitHub (\url{https://github.com}) for sharing code and
Zenodo (\url{https://zenodo.org}) or Dryad (\url{https://datadryad.org})
for sharing data are a key component of reproducible science.

In contrast a lot of the code I wrote for my PhD was based on a
commercial package (Mathematica) and anyone that wants to reuse it a
couple decades later has to pay a few hundred Euros for the package
(well there are student discounts) and has to find a version of the
Mathematica that still runs this code. This latter problem can be
insurmountable, as sometimes older versions of the software are no
longer commercially available. One can argue that many programming
languages are also free and public, however for ecological theory and
modelling a high-level language like R provides a nice integrated
platform for development and is literally priceless.

But programming is not enough. In order to be able to effectively
develop theory and models of biodiversity, a good grasp of mathematics,
including of probability and statistics is essential. One challenge is
that, and this at least the case in Europe where I have lectured the
most, most biology students receive very little mathematical training.
This book cannot address by itself such gap and assumes some basic
familiarity with probability theory (e.g.~what is a probability
distribution function), some calculus training (e.g.~what area a
derivative and an integral), a little algebra (e.g.~multiplication of a
matrix by a vector). I try to take the reader forward from that level.
There are excellent books out there for those that need this basic
background (e.g.~Otto's and Day's A Biologist's Guide to Mathematical
Modelling). And Wikipedia (\url{https://www.wikipedia.org}) is often
your best friend when you wanna a fast refresh of any mathematical
concept or even for many of the models and ecological concepts presented
in this book.

This book can be used as the support for a semester course in Ecological
Theory or Ecological Modelling. I have used many of these materials over
the last decade in a similar course at iDiv/University of
Halle-Wittenberg. But the book can also be used as self-learning tool or
even as a reference tool. This book takes a lot of inspiration from the
Primer of Ecological Theory of Joan Roughgarden (the first two chapters
draw heavily on her first chapters), that I was lucky to have as a
mentor. As she used to say, one writes papers for the reviewers and
books for the readers. This book is for you.

\part{Ecology of individuals}

\chapter{Ecophysiology and the climate space: when to bask in the
sun?}\label{ecophysiology-and-the-climate-space-when-to-bask-in-the-sun}

We are now almost daily bombarded with news about climate change and its
impacts on people and ecosystems. But how does climate mechanistically
affects organisms? The basic foundations to address this question were
laid out by ecophysiology research. Interestingly, at the time that some
of seminal research on ecophysiology was carried out, back in the
1960's, climate change was not yet in the radar of most people. The
research was driven by the interest in the basic understanding of how
biophysical conditions affect organisms. Today the knowledge gained from
this research has acquired new importance. This is an area whether the
mechanistic knowledge learnt from theory and experiments is now
sometimes overlooked because of the massive datasets and machine
learning approaches available. But let's start with the basics.

\section{A model for the body temperature of an
animal}\label{a-model-for-the-body-temperature-of-an-animal}

Animals can be divided in two big groups in regard to the way they
regulate their temperature: ecoctherms and endotherms
(Figure~\ref{fig-CrociduraVipera}). Endotherms such as mammals and birds
are able to regulate their temperature by producing heat through
metabolism. Ectotherms in contrast must regulate their temperature by
obtaining heat from the environment. By obtaining energy from the
environment, ectotherms have lower energy demands than endotherms, which
have to be burning energy all the time to keep their bodies warm. Both
do have to avoid getting too hot because indeed there can bee too much
from a good thing.

\begin{figure}

\begin{minipage}{0.50\linewidth}
\includegraphics{images/Crocidura_russula.jpg}\hfill
\end{minipage}%
%
\begin{minipage}{0.50\linewidth}
\hfill
\includegraphics{images/Vipera_seonae.jpg}\end{minipage}%

\caption{\label{fig-CrociduraVipera}Shrews (left, \emph{Crocidura
russula}) are ectotherms and have high energy demands, eating almost
their body weight every day. In contrast, vipers (right, \emph{Vipera
seonae}) can go days without eating, or even hibernate or aestivate as
needed.}

\end{figure}%

To build a model of the body temperature of an animal, let's consider a
lizard that is perched in a rock basking in the sun
(Figure~\ref{fig-lizard_heat_flow}). The lizard receives heat from the
direct solar radiation than can be measured in for instance calories per
hour. It can also receive indirect solar radiation, reflected for
instance by the ground. The lizard also exchanges heat with the
surrounding air through convection. If the lizard body temperature is
higher than the surrounding air temperature it will lose heat, while if
the lizard body temperature is lower than it will gain heat. The rate at
which this exchange of heat through convection happens depends on the
body of the animal and the properties of its skin. These properties are
captured in the heat transfer coefficient, which can be measured as
calories per hour per degree Celsius. Similarly, the lizard can receive
heat by conduction from being in contact with a warm rock, or lose heat
to the rock if the rock is colder than the lizard's body temperature.
Finally the animal can lose heat through evapotranspiration, this is by
transpiring water that has a cooling effect when it evaporates. I also
like to think about this model as representing our own experience on the
beach. We can be laying on a towel receiving heat by conduction from the
sand and solar radiation from the sun. If the air temperature is warm,
we can become uncomfortable in the sun and we may seek a shade to reduce
the input from solar radiation, but if there is a cool brise we may be
able to stay in the sun a bit longer. If the air temperature is really
cold and it's really windy our skin hair may rise to reduce the
convection coefficient.

\begin{figure}

\centering{

\includegraphics[width=1\textwidth,height=\textheight]{images/Lizard_Heat_Flow.png}

}

\caption{\label{fig-lizard_heat_flow}The energy flows of a lizard
basking in the sun.}

\end{figure}%

The total flow of energy into the lizard, also known as the heat
exchange equation, can be written as
\begin{equation}\phantomsection\label{eq-heat}{f=q-k(b-a)}\end{equation}

where

\begin{itemize}
\item
  \(f\) is the energy flow (cal/h),
\item
  \(q\) is the quantity of heat in the solar radiation (cal/h),
\item
  \(k\) is the convection coefficient (cal/h/ºC),
\item
  \(b\) is the body temperature (ºC) and
\item
  \(a\) is the air temperature (ºC).
\end{itemize}

If the energy flow is positive, then the lizard is warming, while if it
is negative then the lizard is cooling. The equilibrium\footnote{The
  concept of equilibrium is very important in ecological theory. It is
  often introduced in the context of differential equations and
  corresponds to the point at which the derivative of the variable of
  interest is zero. The heat equation can also be seen as a differential
  equation where \(f\) corresponds to the derivative of the amount of
  heat \(Q\) in the lizard through time \(t\), i.e.~\(dQ/dt\).} happens
when the energy flow is zero and so the the lizard is neither cooling
nor warming. If we solve Equation~\ref{eq-heat} for equilibrium by
replacing \(f\) with zero and expanding the right-hand side of the
equation,

\[
0=q-k\ b+k\ a
\]

and rearranging for \(b\) we find the equilibrium body temperature that
we denote with a hat,

\begin{equation}\phantomsection\label{eq-bodytemp}{
\hat{b}=q/k+a.
}\end{equation}

Let's see if this equation makes sense. It says that the equilibrium
body temperature is the sum of the solar radiation divided by the
convection coefficient with the air temperature. So, at minimum the
equilibrium body temperature is equal to the air temperature when the
solar radiation is zero (e.g.~during the night). But when the solar
radiation is greater than zero then the equilibrium body temperature is
higher than the air temperature as one would expect. How much higher?
Well that depends on the convection coefficient. If the convection
coefficient is very high then the body temperature is mainly determined
by the air temperature. In contrast, if the convection coefficient is
very low then the solar radiation can contribute significantly to the
body temperature.

It's time to start using \textbf{R} to explore this model. For instance,
we can use \textbf{R} to plot the relationship between the body
temperature and the solar radiation. First we create variables for the
heat coefficient and the air temperature and assign some valuues:

\begin{Shaded}
\begin{Highlighting}[]
\NormalTok{k }\OtherTok{=} \DecValTok{50}  \CommentTok{\#convection coefficient (cal/h/ºC)}
\NormalTok{a }\OtherTok{=} \DecValTok{18}  \CommentTok{\#air temperature (ºC)}
\end{Highlighting}
\end{Shaded}

We want to plot the equilibrium body temperature for a range of solar
radiation values. So we create a vector with radiation values, for
instance ranging from 0 cal/h to 1500 cal/h in steps of 500. I am going
to use big letters to denote vectors in \textbf{R} code, in contrast
with scalars which I will denote with small letters.

\begin{Shaded}
\begin{Highlighting}[]
\NormalTok{Q }\OtherTok{=} \FunctionTok{seq}\NormalTok{(}\DecValTok{0}\NormalTok{,}\DecValTok{1500}\NormalTok{,}\AttributeTok{by=}\DecValTok{500}\NormalTok{)  }\CommentTok{\#vector with radiation values}
\end{Highlighting}
\end{Shaded}

Now we can write Equation~\ref{eq-bodytemp} in \textbf{R} to produce a
vector of the equilibrium body temperatures for each value of radiation.

\begin{Shaded}
\begin{Highlighting}[]
\NormalTok{B\_eq }\OtherTok{=}\NormalTok{ Q}\SpecialCharTok{/}\NormalTok{k}\SpecialCharTok{+}\NormalTok{a   }\CommentTok{\#vector with equilibrium body temperatures}
\end{Highlighting}
\end{Shaded}

Let's examine the values of the vector, by binding the two vectors as a
matrix,

\begin{Shaded}
\begin{Highlighting}[]
\FunctionTok{rbind}\NormalTok{(Q,B\_eq)}
\end{Highlighting}
\end{Shaded}

\begin{verbatim}
     [,1] [,2] [,3] [,4]
Q       0  500 1000 1500
B_eq   18   28   38   48
\end{verbatim}

This is nice as we can see the values of the equilibrium body
temperature for each value of solar radiation. But let's visualize this
as a graph, by plotting these vectors in \textbf{R},

\begin{Shaded}
\begin{Highlighting}[]
\FunctionTok{plot}\NormalTok{(Q,B\_eq,}\AttributeTok{type=}\StringTok{"l"}\NormalTok{)}
\end{Highlighting}
\end{Shaded}

\begin{figure}[H]

\centering{

\includegraphics{ecophysiology_files/figure-pdf/fig-bodytemp_radiation-1.pdf}

}

\caption{\label{fig-bodytemp_radiation}Equilibrium body temperature as a
function of solar radiation.}

\end{figure}%

Note that we added \texttt{type="l"} as a parameter of the plot to have
a line drawn instead of a sequence of dots in the plot.

\section{Understanding the climate
space}\label{understanding-the-climate-space}

Another way of looking at the heat exchange equation
Equation~\ref{eq-heat} is to look at what combinations of air
temperature and solar radiation values are livable for the lizard. This
is also known as the climate space of an organism. In order to explore
the climate space of the lizard we need to first assess what are the
maximum and minimum body temperature that the lizard can experience.
Let's assume those are respectively 36 and 24ºC and store them in
\textbf{R,}

\begin{Shaded}
\begin{Highlighting}[]
\NormalTok{b\_max }\OtherTok{=} \DecValTok{36}  \CommentTok{\#maximum body temperature (ºC)}
\NormalTok{b\_min }\OtherTok{=} \DecValTok{24}  \CommentTok{\#minimum body temperature (ºC)}
\end{Highlighting}
\end{Shaded}

We now want to solve Equation~\ref{eq-heat} at equilibrium for the air
temperature as a function of the solar radiation and body temperature.
We can do this by rearranging Equation~\ref{eq-bodytemp},

\begin{equation}\phantomsection\label{eq-air}{
a=b-q/k.
}\end{equation}

This equation can be then used in \textbf{R} to calculate vectors of the
maximum and minimum survivable air temperatures for each value of the
solar radiation in vector \texttt{Q},

\begin{Shaded}
\begin{Highlighting}[]
\NormalTok{A\_max }\OtherTok{=}\NormalTok{ b\_max }\SpecialCharTok{{-}}\NormalTok{ Q}\SpecialCharTok{/}\NormalTok{k  }\CommentTok{\#vector with maximum air temperatures}
\NormalTok{A\_min }\OtherTok{=}\NormalTok{ b\_min }\SpecialCharTok{{-}}\NormalTok{ Q}\SpecialCharTok{/}\NormalTok{k  }\CommentTok{\#vector with minimum air temperatures}
\end{Highlighting}
\end{Shaded}

We can now visualize the climate space, this is, the combination of
solar radiation and air temperatures in which the lizard can survive, by
plotting these two equations (i.e.~by plotting lines with x coordinates
given by the vector Q and the y coordinates given by the vectors A\_max
and A\_min),

\begin{Shaded}
\begin{Highlighting}[]
\FunctionTok{plot}\NormalTok{(Q, A\_max,}\AttributeTok{type=}\StringTok{"l"}\NormalTok{, }\CommentTok{\#plots the maximum survivable air temperature}
     \AttributeTok{xlab=}\StringTok{"Solar radiation (cal/h)"}\NormalTok{, }\CommentTok{\#adds x and y axis labels to the plot}
     \AttributeTok{ylab=}\StringTok{"Air temperature (ºC)"}\NormalTok{)    }
\FunctionTok{lines}\NormalTok{(Q,A\_min) }\CommentTok{\#adds line for the minimum survivable air temperature}
\end{Highlighting}
\end{Shaded}

\begin{figure}[H]

\centering{

\includegraphics{ecophysiology_files/figure-pdf/fig-climatespace-1.pdf}

}

\caption{\label{fig-climatespace}The climate space of a lizard.}

\end{figure}%

Let's paint the area between the two lines which corresponds to the
climate space of the lizard. We can use the \textbf{R} function
\texttt{polygon}, to which we need to give the set of \texttt{x} and
\texttt{y} coordinates delimiting the climate space as vectors. For
instance, the upper lower corner can be the first coordinate, being 0
for the \texttt{x} vector and \texttt{b\_min} for the \texttt{y} vector.
The right upper corner is 1500 for \texttt{x} and \texttt{b\_max-1500/k}
for \texttt{y}.

\begin{Shaded}
\begin{Highlighting}[]
\NormalTok{X }\OtherTok{=} \FunctionTok{c}\NormalTok{(}\DecValTok{0}\NormalTok{,}\DecValTok{0}\NormalTok{,}\DecValTok{1500}\NormalTok{,}\DecValTok{1500}\NormalTok{)}
\NormalTok{Y }\OtherTok{=} \FunctionTok{c}\NormalTok{(b\_min,b\_max,b\_max}\DecValTok{{-}1500}\SpecialCharTok{/}\NormalTok{k,b\_min}\DecValTok{{-}1500}\SpecialCharTok{/}\NormalTok{k)}
\FunctionTok{polygon}\NormalTok{(X,Y,}\AttributeTok{col=}\StringTok{"green"}\NormalTok{)}
\end{Highlighting}
\end{Shaded}

Now that we have the climate space we can plot on top of it some
empirical data of how the conditions are in the field during the day. We
can for instance consider two micro-habitats, a rock (in the sun) and a
bush (in the shade). Suppose we obtain some data from temperature
loggers that were installed in each micro-habitat , recording every
three hours the values of temperature and radiation as tabled below.

\begin{longtable}[]{@{}
  >{\raggedright\arraybackslash}p{(\columnwidth - 6\tabcolsep) * \real{0.1750}}
  >{\raggedright\arraybackslash}p{(\columnwidth - 6\tabcolsep) * \real{0.3000}}
  >{\raggedright\arraybackslash}p{(\columnwidth - 6\tabcolsep) * \real{0.3000}}
  >{\raggedright\arraybackslash}p{(\columnwidth - 6\tabcolsep) * \real{0.2250}}@{}}
\caption{Hypothetical radiation values in two micro-habitats and air
temperatures at different times of the
day.}\label{tbl-temprad}\tabularnewline
\toprule\noalign{}
\begin{minipage}[b]{\linewidth}\raggedright
Time (hh:mm)
\end{minipage} & \begin{minipage}[b]{\linewidth}\raggedright
Radiation rock (cal/h)
\end{minipage} & \begin{minipage}[b]{\linewidth}\raggedright
Radiation bush (cal/h)
\end{minipage} & \begin{minipage}[b]{\linewidth}\raggedright
Temperature (ºC)
\end{minipage} \\
\midrule\noalign{}
\endfirsthead
\toprule\noalign{}
\begin{minipage}[b]{\linewidth}\raggedright
Time (hh:mm)
\end{minipage} & \begin{minipage}[b]{\linewidth}\raggedright
Radiation rock (cal/h)
\end{minipage} & \begin{minipage}[b]{\linewidth}\raggedright
Radiation bush (cal/h)
\end{minipage} & \begin{minipage}[b]{\linewidth}\raggedright
Temperature (ºC)
\end{minipage} \\
\midrule\noalign{}
\endhead
\bottomrule\noalign{}
\endlastfoot
00:00 & 150 & 150 & 18 \\
03:00 & 150 & 150 & 13 \\
06:00 & 800 & 450 & 10 \\
09:00 & 1100 & 600 & 14 \\
12:00 & 1300 & 650 & 21 \\
15:00 & 1200 & 650 & 24 \\
18:00 & 800 & 350 & 22 \\
21:00 & 400 & 200 & 20 \\
\end{longtable}

We can overlay these values of temperatures and radiations on the plot
to understand which microhabitat should the lizard choose at each time
of the day. First we create a vector for each column of
Table~\ref{tbl-temprad}. Note that we must enclose the times of the day
in commas as they are strings. We also append at the end of the vector
the values for 0:00 in order to close the lines (otherwise there would
be a gap between 21:00 and 0:00).

\begin{Shaded}
\begin{Highlighting}[]
\CommentTok{\#Times of the day}
\NormalTok{T }\OtherTok{=} \FunctionTok{c}\NormalTok{(}\StringTok{"00:00"}\NormalTok{,}\StringTok{"03:00"}\NormalTok{,}\StringTok{"06:00"}\NormalTok{,}\StringTok{"09:00"}\NormalTok{,}
      \StringTok{"12:00"}\NormalTok{,}\StringTok{"15:00"}\NormalTok{,}\StringTok{"18:00"}\NormalTok{,}\StringTok{"21:00"}\NormalTok{,}\StringTok{"00:00"}\NormalTok{)}

\CommentTok{\#Solar radiation in the rock habitat}
\NormalTok{Rock\_q }\OtherTok{=} \FunctionTok{c}\NormalTok{(}\DecValTok{150}\NormalTok{,}\DecValTok{150}\NormalTok{,}\DecValTok{800}\NormalTok{,}\DecValTok{1100}\NormalTok{,}\DecValTok{1300}\NormalTok{,}\DecValTok{1200}\NormalTok{,}\DecValTok{800}\NormalTok{,}\DecValTok{400}\NormalTok{,}\DecValTok{150}\NormalTok{)}

\CommentTok{\#Air temperature in the rock habitat}
\NormalTok{Rock\_a }\OtherTok{=} \FunctionTok{c}\NormalTok{(}\DecValTok{18}\NormalTok{,}\DecValTok{13}\NormalTok{,}\DecValTok{10}\NormalTok{,}\DecValTok{14}\NormalTok{,}\DecValTok{21}\NormalTok{,}\DecValTok{24}\NormalTok{,}\DecValTok{22}\NormalTok{,}\DecValTok{20}\NormalTok{,}\DecValTok{18}\NormalTok{)}

\CommentTok{\#Solar radiation in the bush habitat}
\NormalTok{Bush\_q }\OtherTok{=} \FunctionTok{c}\NormalTok{(}\DecValTok{150}\NormalTok{,}\DecValTok{150}\NormalTok{,}\DecValTok{450}\NormalTok{,}\DecValTok{600}\NormalTok{,}\DecValTok{650}\NormalTok{,}\DecValTok{650}\NormalTok{,}\DecValTok{350}\NormalTok{,}\DecValTok{200}\NormalTok{,}\DecValTok{150}\NormalTok{)}

\CommentTok{\#   Air temperature in the bush habitat}
\NormalTok{Bush\_a }\OtherTok{=} \FunctionTok{c}\NormalTok{(}\DecValTok{18}\NormalTok{,}\DecValTok{13}\NormalTok{,}\DecValTok{10}\NormalTok{,}\DecValTok{14}\NormalTok{,}\DecValTok{21}\NormalTok{,}\DecValTok{24}\NormalTok{,}\DecValTok{22}\NormalTok{,}\DecValTok{20}\NormalTok{,}\DecValTok{18}\NormalTok{)}
\end{Highlighting}
\end{Shaded}

We overlay these vectors in the figure by invoking the function
\texttt{lines(Xcoor,Ycoord)} for each micro-habitat. We add labels to
each point to show the correspondence between each point and the time of
the day.

\begin{Shaded}
\begin{Highlighting}[]
\FunctionTok{lines}\NormalTok{(Rock\_q,Rock\_a,}\AttributeTok{col=}\StringTok{"orange"}\NormalTok{)}
\FunctionTok{text}\NormalTok{(Rock\_q,Rock\_a,T)}

\FunctionTok{lines}\NormalTok{(Bush\_q,Bush\_a,}\AttributeTok{col=}\StringTok{"blue"}\NormalTok{)}
\FunctionTok{text}\NormalTok{(Bush\_q,Bush\_a,T)}
\end{Highlighting}
\end{Shaded}

\begin{figure}

\centering{

\includegraphics{ecophysiology_files/figure-pdf/fig-climatespace2-1.pdf}

}

\caption{\label{fig-climatespace2}The climate space of a lizard with the
conditions of two microhabitats ploted at different times of the day:
rock (orange), bush (blue).}

\end{figure}%

\section{From ecophysiological models to species distribution
models}\label{from-ecophysiological-models-to-species-distribution-models}

In the previous section we developed a simple mechanistic model for the
climate space of an organism. This is a micro-habitat level climate
space. But one can also infer the climate space from the distribution of
a species at the macro-habitat level. They are different but they are
conceptually similar. The inference of such macro climate space is an
area where species distribution models excel (for more on species
distribution models check for example (Guisan, Thuiller, and Zimmermann
2017)). The idea is relatively simple. One starts with a bunch of
locations of a species in geographical space. Then, one obtains the
climate characteristics for those locations, and then plots those
locations in climate space. The convex hull in climate space delimited
by those locations is the macro-habitat climate space. What species
distribution models (SDMs) learn to do is to delineate that climate
space, this is, SDMs try to predict the probability of a species
occurring somewhere in the climate space by learning from presences and
absences from species. Then the SDM can extrapolate again what happens
in geographical space, by predicting the species occurence probability
for any point based on the climate data for that point. Without getting
all the way into a full blown SDM, let's try to follow some of the
basics.

\begin{figure}

\centering{

\includegraphics{images/Lacerta_schreiberi.JPG}

}

\caption{\label{fig-schreiberi}The Iberian emerald lizard, \emph{Lacerta
schreiberi}.}

\end{figure}%

I will illustrate this with one my favorite species, the Iberian emerald
lizard, \emph{Lacerta schreiberi}, a species endemic to the Iberian
peninsula, often found near water streams in mountain landscapes, but
also in other habitats. First we obtain the presences for a species from
GBIF (Global Biodiversity Information Facility,
\url{http://www.gbif.org}). GBIF is a repository for biodiversity
observations, and has over three billions of occurrences of over a
million species all over the world at the time I am writing these words.
Natural history museums, citizen science platforms such as iNaturalist (
\url{http://www.inaturalist.org}) and eBird
(\url{http://www.ebird.org}), and scientists, publish their species
records in GBIF using a data format called Darwin Core. There is a R
package that allows one to download observations from GBIF and also
other relevant geodata, the \textbf{geodata} package. We will be doing
some geospatial processing, so let's load also the \textbf{terra}
package.

\begin{Shaded}
\begin{Highlighting}[]
\FunctionTok{library}\NormalTok{(geodata)}
\FunctionTok{library}\NormalTok{(terra)}
\FunctionTok{library}\NormalTok{(predicts)}
\FunctionTok{library}\NormalTok{(ggplot2)}
\end{Highlighting}
\end{Shaded}

We download the observations of \emph{Lacerta schreiberi} from GBIF with
the function \texttt{sp\_occurence(genus,species)}. We also download a
world map with the function \texttt{world(path)}, where path should
point towards a local directory where we want to store the data.

\begin{Shaded}
\begin{Highlighting}[]
\NormalTok{occ }\OtherTok{\textless{}{-}} \FunctionTok{sp\_occurrence}\NormalTok{(}\StringTok{"Lacerta"}\NormalTok{,}\StringTok{"schreiberi"}\NormalTok{) }
\NormalTok{world\_data }\OtherTok{\textless{}{-}} \FunctionTok{world}\NormalTok{(}\AttributeTok{path =} \StringTok{"."}\NormalTok{) }\CommentTok{\#download world countries limits}
\NormalTok{ext\_eur }\OtherTok{\textless{}{-}} \FunctionTok{c}\NormalTok{(}\SpecialCharTok{{-}}\DecValTok{15}\NormalTok{,}\DecValTok{45}\NormalTok{,}\DecValTok{35}\NormalTok{,}\DecValTok{72}\NormalTok{) }\CommentTok{\# coordinates of European extent}

\CommentTok{\#download bioclim data}
\NormalTok{bioclim }\OtherTok{\textless{}{-}} \FunctionTok{worldclim\_global}\NormalTok{(}\AttributeTok{var =} \StringTok{\textquotesingle{}bioc\textquotesingle{}}\NormalTok{, }\AttributeTok{res =} \DecValTok{10}\NormalTok{, }\AttributeTok{path =} \StringTok{"."}\NormalTok{)}

\CommentTok{\# plot the map of Europe}
\FunctionTok{plot}\NormalTok{(world\_data,  }
     \AttributeTok{xlab =} \StringTok{"Longitude"}\NormalTok{, }\AttributeTok{ylab =} \StringTok{"Latitude"}\NormalTok{, }\AttributeTok{axes =} \ConstantTok{TRUE}\NormalTok{, }\AttributeTok{ext=}\NormalTok{ext\_eur)}

\CommentTok{\# add the species occurrences}
\FunctionTok{points}\NormalTok{(occ}\SpecialCharTok{$}\NormalTok{lon,occ}\SpecialCharTok{$}\NormalTok{lat,}\AttributeTok{cex=}\NormalTok{.}\DecValTok{1}\NormalTok{,}\AttributeTok{col=}\StringTok{"blue"}\NormalTok{)}

\CommentTok{\# create pseudo{-}absences}
\NormalTok{eumask}\OtherTok{\textless{}{-}}\FunctionTok{crop}\NormalTok{(bioclim[[}\DecValTok{1}\NormalTok{]], }\FunctionTok{ext}\NormalTok{(ext\_eur))}
\NormalTok{absences}\OtherTok{\textless{}{-}}\FunctionTok{backgroundSample}\NormalTok{(eumask,}\FunctionTok{nrow}\NormalTok{(occ),}\FunctionTok{vect}\NormalTok{(occ),}\AttributeTok{tryf=}\DecValTok{10}\NormalTok{)}
\FunctionTok{points}\NormalTok{(absences,}\AttributeTok{cex=}\NormalTok{.}\DecValTok{1}\NormalTok{,}\AttributeTok{col=}\StringTok{"red"}\NormalTok{)}


\CommentTok{\# plot the climate variables in geographic space}
\FunctionTok{plot}\NormalTok{(bioclim[[}\DecValTok{19}\NormalTok{]],}\AttributeTok{ext=}\NormalTok{ext\_eur)}
\FunctionTok{plot}\NormalTok{(bioclim[[}\DecValTok{3}\NormalTok{]],}\AttributeTok{ext=}\NormalTok{ext\_eur)}

\CommentTok{\# plot the climate space figure}
\CommentTok{\# first merge occurences and absences into a single matrix}
\NormalTok{allpts }\OtherTok{\textless{}{-}} \FunctionTok{rbind}\NormalTok{(}\FunctionTok{cbind}\NormalTok{(occ}\SpecialCharTok{$}\NormalTok{lon,occ}\SpecialCharTok{$}\NormalTok{lat,}\FunctionTok{rep}\NormalTok{(}\DecValTok{1}\NormalTok{,}\FunctionTok{nrow}\NormalTok{(occ))),}
                \FunctionTok{cbind}\NormalTok{(absences,}\FunctionTok{rep}\NormalTok{(}\DecValTok{0}\NormalTok{,}\FunctionTok{nrow}\NormalTok{(absences))))}
\FunctionTok{colnames}\NormalTok{(allpts)[}\DecValTok{3}\NormalTok{]}\OtherTok{\textless{}{-}}\StringTok{"pres"}

\CommentTok{\# extract the climate variable values for all points}
\NormalTok{bioclim\_sample}\OtherTok{\textless{}{-}}\FunctionTok{extract}\NormalTok{(bioclim,allpts[,}\DecValTok{1}\SpecialCharTok{:}\DecValTok{2}\NormalTok{])}
\NormalTok{allpts\_bioclim }\OtherTok{\textless{}{-}} \FunctionTok{cbind}\NormalTok{(allpts,bioclim\_sample)}

\NormalTok{allpts\_bioclim\_plot}\OtherTok{\textless{}{-}}\NormalTok{allpts\_bioclim}
\NormalTok{allpts\_bioclim\_plot}\SpecialCharTok{$}\NormalTok{pres}\OtherTok{\textless{}{-}}\FunctionTok{as.factor}\NormalTok{(allpts\_bioclim\_plot}\SpecialCharTok{$}\NormalTok{pres)}
\FunctionTok{ggplot}\NormalTok{(allpts\_bioclim\_plot, }\FunctionTok{aes}\NormalTok{(}\AttributeTok{x=}\NormalTok{wc2}\FloatTok{.1}\NormalTok{\_10m\_bio\_19,}\AttributeTok{y=}\NormalTok{wc2}\FloatTok{.1}\NormalTok{\_10m\_bio\_3,}\AttributeTok{shape=}\NormalTok{pres, }\AttributeTok{col=}\NormalTok{pres)) }\SpecialCharTok{+}
  \FunctionTok{geom\_point}\NormalTok{(}\AttributeTok{size =} \DecValTok{3}\NormalTok{, }\AttributeTok{alpha=}\FloatTok{0.5}\NormalTok{) }\SpecialCharTok{+} \FunctionTok{scale\_color\_manual}\NormalTok{(}\AttributeTok{values =} \FunctionTok{c}\NormalTok{(}\StringTok{"0"} \OtherTok{=} \StringTok{"red"}\NormalTok{, }\StringTok{"1"} \OtherTok{=} \StringTok{"blue"}\NormalTok{)) }
\end{Highlighting}
\end{Shaded}

\begin{figure}

\begin{minipage}{0.50\linewidth}

\centering{

\includegraphics{ecophysiology_files/figure-pdf/fig-geog-vs-climate-1.pdf}

}

\subcaption{\label{fig-geog-vs-climate-1}Geographic space}

\end{minipage}%
%
\begin{minipage}{0.50\linewidth}

\centering{

\includegraphics{ecophysiology_files/figure-pdf/fig-geog-vs-climate-2.pdf}

}

\subcaption{\label{fig-geog-vs-climate-2}Bioclim 19}

\end{minipage}%
\newline
\begin{minipage}{0.50\linewidth}

\centering{

\includegraphics{ecophysiology_files/figure-pdf/fig-geog-vs-climate-3.pdf}

}

\subcaption{\label{fig-geog-vs-climate-3}Bioclim 13}

\end{minipage}%
%
\begin{minipage}{0.50\linewidth}

\centering{

\includegraphics{ecophysiology_files/figure-pdf/fig-geog-vs-climate-4.pdf}

}

\subcaption{\label{fig-geog-vs-climate-4}Climate space}

\end{minipage}%

\caption{\label{fig-geog-vs-climate}The geographic space and climate
space of Lacerta schreiberi.}

\end{figure}%

\section{Statistical confrontation: regressing the climate
space}\label{statistical-confrontation-regressing-the-climate-space}

\emph{By Andres Marmol}

Why anoles? Anoles are among the most studied reptile organisms. They
are vastly diverse with more than 488 species described until today ---
circa 11\% of all the squamates without considering snakes --- due to
their outstanding adaptive radiation, which has inspired many to
dive-in-depth into understanding how evolution has operated within this
group
(\href{https://doi.org/10.1525/california/9780520255913.001.0001}{Lizards
in an Evolutionary Tree} by J. B. Losos is a highly recommended
instructive reading for more info). As a result, very detailed data sets
on many species of anoles, including information on micro habitat, field
body temperature, activity patterns and morphological traits,
performance, etc., can be found online in relatively few time.

In the following steps we will be downloading from Dryad and using a
data set created by Winchell et al.~(2016)
(https://doi.org/10.5061/dryad.h234n) on \emph{Anolis cristatellus}
making a pair-comparison of several morphological and thermal traits of
individuals living in urban and forested areas at Puerto Rico. \emph{A.
cristatellus} is widely distributed across Puerto Rico and can be found
in natural and human-intervened spaces easily. It can be seen over
ground and laying on the trunks of trees in natural settings, and on
walls and metal fences in more urban areas.

\subsection{Calling the dataset from
Dryad}\label{calling-the-dataset-from-dryad}

To download the dataset, it is necessary to install the R package
\texttt{rdryad}.

\begin{Shaded}
\begin{Highlighting}[]
\FunctionTok{install.packages}\NormalTok{(}\StringTok{"rdryad"}\NormalTok{)}
\end{Highlighting}
\end{Shaded}

Then we open the newly installed package and download the dataset by
specifying the DOI associated to this dataset.

\begin{Shaded}
\begin{Highlighting}[]
\CommentTok{\#calling rdryad}
\FunctionTok{library}\NormalTok{(rdryad)}

\CommentTok{\# Specify the DOI}
\NormalTok{doi }\OtherTok{\textless{}{-}} \StringTok{"10.5061/dryad.h234n"}

\CommentTok{\# Download the dataset using the dryad\_download function}
\NormalTok{downloaded\_file }\OtherTok{\textless{}{-}} \FunctionTok{dryad\_download}\NormalTok{(}\AttributeTok{dois =}\NormalTok{ doi)}
\end{Highlighting}
\end{Shaded}

The data sets are now downloaded on the paths indicated in the box
below, although for this exercise only table {[}2{]}
winchell\_evol\_phenshifts.csv'' is needed.

\begin{Shaded}
\begin{Highlighting}[]
\CommentTok{\# The downloaded\_file will contain the file path where the dataset is saved}
\FunctionTok{print}\NormalTok{(downloaded\_file)}
\end{Highlighting}
\end{Shaded}

\begin{verbatim}
$`10.5061/dryad.h234n`
[1] "/Users/henrique/Library/Caches/R/rdryad/10_5061_dryad_h234n/winchell_evol_CG.csv"        
[2] "/Users/henrique/Library/Caches/R/rdryad/10_5061_dryad_h234n/winchell_evol_phenshifts.csv"
\end{verbatim}

Once this is done, it is always good practice to check if the recently
open data frame works by checking the columns within the data frame.

\begin{Shaded}
\begin{Highlighting}[]
\CommentTok{\#calling the dataset of interest as df}
\NormalTok{df }\OtherTok{\textless{}{-}} \FunctionTok{read.csv}\NormalTok{(downloaded\_file[[}\DecValTok{1}\NormalTok{]][}\DecValTok{2}\NormalTok{])}

\CommentTok{\# checking the columns (variables) available in the data frame df using the str function}
\FunctionTok{names}\NormalTok{(df)}
\end{Highlighting}
\end{Shaded}

\begin{verbatim}
 [1] "ID"                 "date"               "Site"              
 [4] "context"            "perch"              "bodytemp.C"        
 [7] "perch.temp.C"       "ambient.temp.C"     "humidity.percent"  
[10] "perch.height.cm"    "perch.diam.cm"      "weight.g"          
[13] "head.height.mm"     "svl.mm"             "local.time.decimal"
[16] "flags"              "JL"                 "JW"                
[19] "METC"               "RAD"                "ULN"               
[22] "HUM"                "FEM"                "TIB"               
[25] "FIB"                "METT1"              "METT2"             
[28] "FL"                 "HL"                
\end{verbatim}

There are 29 different variables within \texttt{df}, but this exercise
will focus only in a few of them including \texttt{context},
\texttt{perch}, \texttt{bodytemp.C}, \texttt{perch.temp.C},
\texttt{ambient.temp.C}, and \texttt{local.time.decimal} where
\texttt{context} indicates if the lizard was captured in natural or
urban spaces; \texttt{perch} refers to the microhabitat where the lizard
was found; \texttt{bodytemp.C} is body temperature \(b\);
\texttt{perch.temp.C} is substrate temperature \(T_s\);
\texttt{ambient.temp.C} is air temperature \(a\); and
\texttt{local.time.decimal} is the time of the day in hours when the
lizards was captured and measured.

\subsection{\texorpdfstring{Exploring the thermophysiological data of
\emph{A.
cristatellus.}}{Exploring the thermophysiological data of A. cristatellus.}}\label{sec-anolis_cristatellus}

As covered in the first chapter, \(b\) increases proportionally to \(a\)
as indicated in equation (1.2). Let's check if the data confirms this
equation.

\begin{Shaded}
\begin{Highlighting}[]
\CommentTok{\# to call a variable directly from a data frame one must use $, indicating the name of the dataframe at the left and the name of the variable (columnt on the right).}

\FunctionTok{plot}\NormalTok{(df}\SpecialCharTok{$}\NormalTok{ambient.temp.C, df}\SpecialCharTok{$}\NormalTok{bodytemp.C,}
     \AttributeTok{xlab=}\StringTok{"air temperature (°C)"}\NormalTok{,}
     \AttributeTok{ylab=}\StringTok{"body temperature (°C)"}\NormalTok{)}
\end{Highlighting}
\end{Shaded}

\includegraphics{ecophysiology_files/figure-pdf/unnamed-chunk-16-1.pdf}

In the plot we can observe that indeed body temperature increases with
increasing temperature. However, we cannot quantitatively know how much
is this increase. For that, we can model how air temperature affects
body temperature modelling them a a line with the equation:

\[
\hat{y} = α + β*x + e 
\] where \(\hat{y}\) is the predicted value of \(y\) for any given
\(x\); \(α\) is point of intersection of the line when \(x=0\), \(β\) is
the slope of the regression line (i.e.~how much increases \(\hat{y}\)
with a unit increase in \(x\)) and \(e\) is the random error.

Placing our variables in formula, it would look like this:

\[
b = b_{0} + β*a + e
\]

where \(b\) is body temperature, \(a\) is ambient temperature, and
\(b_0\) is the body temperature when \(a\) is zero.

Now lets model it using a least-squares linear regression using the
\texttt{lm} function in R.

\begin{Shaded}
\begin{Highlighting}[]
\CommentTok{\#define the linear model}
\CommentTok{\#the variable at the left of the tilde (\textasciitilde{}) is the dependent variable}
\CommentTok{\#data calls the specific dataframe to be used.}

\NormalTok{model1 }\OtherTok{\textless{}{-}} \FunctionTok{lm}\NormalTok{(bodytemp.C }\SpecialCharTok{\textasciitilde{}}\NormalTok{ ambient.temp.C, }\AttributeTok{data=}\NormalTok{df)}

\CommentTok{\# print model outputs}
\FunctionTok{print}\NormalTok{(model1)}
\end{Highlighting}
\end{Shaded}

\begin{verbatim}

Call:
lm(formula = bodytemp.C ~ ambient.temp.C, data = df)

Coefficients:
   (Intercept)  ambient.temp.C  
        8.7004          0.7237  
\end{verbatim}

\subsection{Interpreting the results of our
model.}\label{interpreting-the-results-of-our-model.}

The results of our model indicates that if the air temperature (\(a\))
is zero, \emph{A. cristatellus} would show a body temperature (\(b\)) of
8.7 °C, which increases in approximately 0.7 °C for each 1 °C increase
in air temperature.

Now it is also possible to draw a few predictions

\begin{Shaded}
\begin{Highlighting}[]
\CommentTok{\# a number of possible ambient temperatures that can be found in Puerto Rico during the day}

\NormalTok{amb\_temps }\OtherTok{=} \FunctionTok{data.frame}\NormalTok{(}\AttributeTok{ambient.temp.C =} \FunctionTok{seq}\NormalTok{(}\DecValTok{27}\NormalTok{,}\DecValTok{38}\NormalTok{, }\DecValTok{1}\NormalTok{))}\CommentTok{\# defines a vector with values from 27 to 38, with one unit increase}

\FunctionTok{predict}\NormalTok{(model1, }\AttributeTok{newdata=}\NormalTok{amb\_temps)}
\end{Highlighting}
\end{Shaded}

\begin{verbatim}
       1        2        3        4        5        6        7        8 
28.23967 28.96335 29.68702 30.41070 31.13437 31.85805 32.58172 33.30540 
       9       10       11       12 
34.02907 34.75275 35.47642 36.20010 
\end{verbatim}

\subsection{How good our model is
performing?}\label{how-good-our-model-is-performing}

Graphically the linear model fits the data as follows:

\begin{Shaded}
\begin{Highlighting}[]
\CommentTok{\# Create the scatter plot}
\FunctionTok{plot}\NormalTok{(df}\SpecialCharTok{$}\NormalTok{ambient.temp.C, df}\SpecialCharTok{$}\NormalTok{bodytemp.C,}
     \AttributeTok{xlab=}\StringTok{"air temperature (°C)"}\NormalTok{,}
     \AttributeTok{ylab=}\StringTok{"body temperature (°C)"}\NormalTok{)}

\CommentTok{\# add the line of fit from our recently created linear model "model1"}
\FunctionTok{abline}\NormalTok{(model1, }\AttributeTok{col =} \StringTok{"red"}\NormalTok{, }\AttributeTok{lwd=} \DecValTok{2}\NormalTok{)}
\end{Highlighting}
\end{Shaded}

\includegraphics{ecophysiology_files/figure-pdf/unnamed-chunk-19-1.pdf}

The red line in the plot follows very well the pattern of increase shown
by our data. It is noticeable, however, that some points fall farther
from the line of best fit than others. In other words, many \(\hat{y}\)
values (body temperature) are predicted with larger errors than others,
for a given \(x\) value (ambient temperature). These errors are also
known as residuals, and are the vertical distance between the observed
values and the line of best fit. They are graphically represented in
green in our plot:

\includegraphics{ecophysiology_files/figure-pdf/unnamed-chunk-20-1.pdf}

What a least square linear regression does is to find the lowest sum of
squares of the residuals (every green line in the plot) so that the
least amount of error is kept in the linear model.

Finally, how well the model fits the data? In our example, the body
temperatures do not perfectly fall on the line of best fit, and many of
them are scattered around it. A common metric assess this question in a
simple regression model is the \(r^2\). \(r^2\) values indicates how
much of the variance of body temperature (our dependent variable) is
explained by ambient temperature (our independent variable). A perfect
fit is indicated by an \(r^2\) of 1, while no fit would be zero. To
check the \(r^2\) in our model we can simply use the \texttt{summary}
function in R.

\begin{Shaded}
\begin{Highlighting}[]
\CommentTok{\# obtaining the r{-}squared of our model}

\FunctionTok{summary}\NormalTok{(model1)}\SpecialCharTok{$}\NormalTok{r.squared}
\end{Highlighting}
\end{Shaded}

\begin{verbatim}
[1] 0.5736088
\end{verbatim}

\(r^2\) equals 0.574 indicating that almost 60\% of the variance in body
temperature is explained by ambient temperature.

In summary, our expectation that lizards body temperature increases
depending on how the temperature in their environment changes has been
confirmed. Specifically for \emph{A. cristatellus} in this study, body
temperature increases \textasciitilde{} 0.7 °C every 1 degree increase
in air temperature. A few questions remain. For example, does habitat
has an effect in the body temperature in \emph{A. cristatellus}, and if
it does, how different could the body temperatures of the lizards in
habitat A are from those in habitat B?

\section{References}\label{references}

\phantomsection\label{refs}
\begin{CSLReferences}{1}{0}
\bibitem[\citeproctext]{ref-guisan2017}
Guisan, Antoine, Wilfried Thuiller, and Niklaus E. Zimmermann. 2017.
\emph{Habitat Suitability and Distribution Models: With Applications in
r}. Cambridge University Press.

\end{CSLReferences}

\chapter{Economic models of behavior: do animals
optimize?}\label{economic-models-of-behavior-do-animals-optimize}

Animals have to make decisions all the time: where and when to move,
when and what to eat, whom to mate, etc. How do animals make those
decisions? One approach to study this question is to assume that animals
take decisions that maximize their fitness. This approach is rooted in
the idea that natural selection, under certain conditions, maximize the
average fitness of individuals in a population. It underpins many of the
studies in behavioral ecology, and lead to the development during the
1980's.of a sub-field dedicated to the study of foraging decisions,
known as optimal foraging. The idea of optimal foraging is to develop
economic models of foraging behavior, assessing the costs and benefits
of different actions, and identifying in any particular circumstance the
action that maximizes benefits and minimizes costs. This is, animals are
seen as optimizer of decision-making. Here we consider models for two
``idealized'' types of foragers: searching predator and sit-and-wait
predators.

\section{Searching predator}\label{searching-predator}

The searching predator is typified as an animal that actively searchs
for food. For instance, wolfs roam the landscape constantly searching
for their prey (Figure~\ref{fig-Canis_lupus}). The question is then, any
time that a forager comes across a prey whether to spend the time
catching and eating it, or continuing to search for another prey, for
instance a larger or easier prey item. Joan Roughgarden also liked this
type of foraging to the sushi-bar problem where dishes are coming one
after the other on a moving tray, and a person needs to decide whether
to pick a given dish. This decision is particularly akin to the
searching predator situation when one assumes that a person can have
only one dish at a time in her or his table, We consider two types of
searching predators: time minimizers and energy maximizers.

\begin{figure}

\centering{

\includegraphics{images/Canis_lupus.jpg}

}

\caption{\label{fig-Canis_lupus}A pack of wolf (\emph{Canis lupus})
roaming around the landscape of Peneda-Gerês National Park, Portugal.}

\end{figure}%

\subsection{Time minimizer}\label{time-minimizer}

Many searching predators are themselves preys to other animals. For
instance shrews search for insects to eat, but can be themselves eaten
by carnivores or owls. Therefore a reasonable assumption is that they
are making foraging decisions that minimize the time foraging. Let's
assume there are two types of prey items, type 1 and 2. These two preys
occur at different abundances in the environment, let's name them
\(a_1\) and \(a_2\). These abundances can be measured as encounter rates
from the predator perspective, this is the number of prey items found
per unit time. The two types of prey also have different handling times,
this is the amount of time required to chase and process the prey,
\(h_1\) and \(h_2\) . We also convention to call the type 1 prey the
prey with the lowest handling time, i.e.~\(h_1<h_2\),

A searching predator can adopt one of the following three strategies:

\begin{itemize}
\item
  Strategy 1: to consume only prey items of type 1
\item
  Strategy 2: to consumer only prey items of type 2
\item
  Strategy 1\&2: to consumer both types of prey
\end{itemize}

In order to find out which strategy should be adopted by the forager,
one needs to calculate the average spent per food item in each of the
strategies. For Strategy 1 the average time per item, \(T_1\) is the sum
of the amount of time the predator needs to encounter a prey with the
amount of time that it takes to process that prey. As the abundance is
measured in encounter rates, i.e.~prey items per unit time, the inverse
of that is the waiting time for a prey. Therefore we have for Strategy
1,

\begin{equation}\phantomsection\label{eq-T1_timeminimizer}{T_1=1/a_1+h_1}\end{equation}

Similarly, for Strategy 2, the average time per item \(T_2\) is given by

\begin{equation}\phantomsection\label{eq-T2_timeminimizer}{T_2=1/a_2+h_2.}\end{equation}

A more interesting case is Strategy 3. Here the waiting time is the
inverse of the sum of the abundances of both times of prey, while the
handling time is average of the handling times of both ypes of prey
weighted by their relative abundance,

\begin{equation}\phantomsection\label{eq-T3_timeminimizer}{
T_{3}=\frac{1}{a_1+a_2}+\frac{a_1 h_1+a_2 h_2}{a_1+a_2}
}\end{equation}

We can now define the problem that the forager has to solve as to find
the strategy \(i\) that minimizes \(T_i\):

\[
min(T_i) \quad \text{for} \quad i=1, 2, 3
\]

It is easy to demonstrate that Strategy 2 is always worse than Strategy
3, independently of the abundances of the two types of prey. This
happens because Strategy 3 always implies a longer waiting time than the
two other specialized strategies (i.e.~one has to wait less time to find
any item of an prey than items of a given prey time,
\(1/(a_1+a_2)<1/a_2\)) and the handling time of strategy 3 can never be
higher than the handling time of strategy 2 (it's always a value betwen
\(h_1\) and \(h_2\)). So we can exclude Strategy 2 from our analysis.
The choice is then between Strategy 1, just taking items of the
preferred prey item, and 3, taking items of both types of prey. Let's
assess this two strategies with a little bit of help from R. We start by
assuming that the handling time of prey type 1 is 1 second while prey
type 2 takes 60 seconds. Let's also assume that the abundance of prey
type 2 is 0.05 individuals per second, i.e.~one individuals needs to
wait in average 20 seconds to find prey type 2.

\begin{Shaded}
\begin{Highlighting}[]
\NormalTok{h1}\OtherTok{=}\DecValTok{1}           \CommentTok{\#Handling time of prey type 1 (s)}
\NormalTok{h2}\OtherTok{=}\DecValTok{60}          \CommentTok{\#Handling time of prey type 2 (s)}
\NormalTok{a2}\OtherTok{=}\FloatTok{0.05}        \CommentTok{\#abundance of prey type 2 (ind/s)}
\end{Highlighting}
\end{Shaded}

Let's now plot the time per item of each of the strategies as a function
of the abundance of the preferred prey.

\begin{Shaded}
\begin{Highlighting}[]
\NormalTok{a1}\OtherTok{\textless{}{-}}\FunctionTok{seq}\NormalTok{(}\FloatTok{0.005}\NormalTok{,}\FloatTok{0.1}\NormalTok{,}\FloatTok{0.001}\NormalTok{)   }\CommentTok{\#abundance of prey time 1 (ind/s)}
\NormalTok{t1}\OtherTok{=}\DecValTok{1}\SpecialCharTok{/}\NormalTok{a1}\SpecialCharTok{+}\NormalTok{h1                 }\CommentTok{\#time per item of Strategy 1 (s)}
\NormalTok{t3}\OtherTok{=}\DecValTok{1}\SpecialCharTok{/}\NormalTok{(a1}\SpecialCharTok{+}\NormalTok{a2)}\SpecialCharTok{+}\NormalTok{h1}\SpecialCharTok{*}\NormalTok{a1}\SpecialCharTok{/}\NormalTok{(a1}\SpecialCharTok{+}\NormalTok{a2)}\SpecialCharTok{+}\NormalTok{h2}\SpecialCharTok{*}\NormalTok{a2}\SpecialCharTok{/}\NormalTok{(a1}\SpecialCharTok{+}\NormalTok{a2) }\CommentTok{\#time per item of Strategy 2 (s)}
\FunctionTok{plot}\NormalTok{(a1,t1, }\AttributeTok{type=}\StringTok{"l"}\NormalTok{, }\AttributeTok{col=}\StringTok{"blue"}\NormalTok{)}
\FunctionTok{lines}\NormalTok{(a1,t3, }\AttributeTok{type=}\StringTok{"l"}\NormalTok{, }\AttributeTok{col=}\StringTok{"red"}\NormalTok{)}
\FunctionTok{legend}\NormalTok{(}\StringTok{"topright"}\NormalTok{,                    }
       \AttributeTok{legend =} \FunctionTok{c}\NormalTok{(}\StringTok{"Strategy 1"}\NormalTok{, }\StringTok{"Strategy 3"}\NormalTok{),  }\CommentTok{\# Labels}
       \AttributeTok{col =} \FunctionTok{c}\NormalTok{(}\StringTok{"red"}\NormalTok{, }\StringTok{"blue"}\NormalTok{),             }\CommentTok{\# Line colors}
       \AttributeTok{lwd =} \DecValTok{2}\NormalTok{,                            }\CommentTok{\# Line width}
       \AttributeTok{lty =} \DecValTok{1}\NormalTok{)      }
\end{Highlighting}
\end{Shaded}

\includegraphics{optimalforaging_files/figure-pdf/unnamed-chunk-2-1.pdf}

There is a critical threshold of the abundance of prey type 1 above
which strategy 1 is preferrable, while below that threshold strategy 3
is the best strategy. Interestingly this threshold does not depend on
the abundance of the less preferable prey. For instance, if we assume a
low abundance of prey time 2 at 0.01, the resulting plot is:

\begin{Shaded}
\begin{Highlighting}[]
\NormalTok{a2 }\OtherTok{=} \FloatTok{0.01}
\NormalTok{t1}\OtherTok{=}\DecValTok{1}\SpecialCharTok{/}\NormalTok{a1}\SpecialCharTok{+}\NormalTok{h1                 }\CommentTok{\#time per item of Strategy 1 (s)}
\NormalTok{t3}\OtherTok{=}\DecValTok{1}\SpecialCharTok{/}\NormalTok{(a1}\SpecialCharTok{+}\NormalTok{a2)}\SpecialCharTok{+}\NormalTok{h1}\SpecialCharTok{*}\NormalTok{a1}\SpecialCharTok{/}\NormalTok{(a1}\SpecialCharTok{+}\NormalTok{a2)}\SpecialCharTok{+}\NormalTok{h2}\SpecialCharTok{*}\NormalTok{a2}\SpecialCharTok{/}\NormalTok{(a1}\SpecialCharTok{+}\NormalTok{a2) }\CommentTok{\#time per item of Strategy 2 (s)}
\FunctionTok{plot}\NormalTok{(a1,t1, }\AttributeTok{type=}\StringTok{"l"}\NormalTok{, }\AttributeTok{col=}\StringTok{"blue"}\NormalTok{)}
\FunctionTok{lines}\NormalTok{(a1,t3, }\AttributeTok{type=}\StringTok{"l"}\NormalTok{, }\AttributeTok{col=}\StringTok{"red"}\NormalTok{)}
\end{Highlighting}
\end{Shaded}

\includegraphics{optimalforaging_files/figure-pdf/unnamed-chunk-3-1.pdf}

To determine this critical threshold one can compare the two vectors, T1
and T3, and find the first position at which T1 becomes smaller than T3,

\begin{Shaded}
\begin{Highlighting}[]
\NormalTok{pos}\OtherTok{=}\FunctionTok{which}\NormalTok{(t1}\SpecialCharTok{\textless{}}\NormalTok{t3)[}\DecValTok{1}\NormalTok{]}
\NormalTok{a1[pos]}
\end{Highlighting}
\end{Shaded}

\begin{verbatim}
[1] 0.017
\end{verbatim}

So the critical threshold for these handling times occurs when
\(a_1=0.017\) individuals per second.

\subsection{Energy per time maximizer}\label{energy-per-time-maximizer}

Perhaps more often, animals try to maximize their energy yield
(benefits) while minimizing the time foraging (costs). Or in another way
of looking at it, they try to maximize their energy yield per unit time.
We already know the time per item associated to each of the three
strategies of the searching predator. We now need to caiculate the
average energy yield per item. Consider now that the energy content of
the prey items are \(e_1\) and \(e_2\) for prey of type 1 and 2,
respectively. We define prey 1 as the preferred type of prey, so we
assume that the ratio of the energic content (measured for instance in
calories) to the handling time is higher for type 1 prey,
i.e.~\(e_1/h_1>e_2/h_2\). Now we calculate the energy yield per item for
each strategy. We start with the energy content of the prey, but need to
subtract the energy spent while waiting the prey and the energy spent
chasing and processing the prey,

\begin{equation}\phantomsection\label{eq-E1_energymaximizer}{
E_1=e_1-ew*tw_1-eh*h_1
}\end{equation}

where \(ew\) and \(eh\) are the energy spent per unit time while waiting
for the prey and the energy spent per unit time while handling,
respectively. They can both be measured for instance in cal/s. We
already know from the time minimizer that the waiting time for the prey
is the inverse of the abundance, \(tw_1=1/a_1\). Therefore substituting
in Equation~\ref{eq-E1_energymaximizer} we have

\begin{equation}\phantomsection\label{eq-E1_energymaximizer2}{
E_1=e_1-\frac{ew}{a_1}-eh*h_1.
}\end{equation}

A similar expression can be written for Strategy 2, replacing 1 with 2
in Equation~\ref{eq-E1_energymaximizer}.

\begin{equation}\phantomsection\label{eq-E2_energymaximizer}{
E_2=e_2-\frac{ew}{a_2}-eh*h_2.
}\end{equation}

More interesting is to derive the expression for Strategy 3, where the
foragers takes both types of prey. The energy content of the prey is the
average of the energetic contents of each prey type, weighted by their
abundances, \((e_1*a_1+e_2*a_2)/(a_1+a_2)\). The waiting time is the
inverse of the sums of the abundances of preys of both types, as in
Equation~\ref{eq-T3_timeminimizer}. The handling time is the average of
the handling times of each prey type, weighted by their abundances. So
we have,

\begin{equation}\phantomsection\label{eq-E3_energymaximizer}{
E_3=\frac{e_1*a_1+e_2*a_2}{a_1+a_2}-\frac{ew}{a_1+a_2}-eh\frac{h_1*a_1+h_2*a_2}{a_1+a_2}.
}\end{equation}

Finally we can calculate the energy per unit time for each strategy by
dividing Equation~\ref{eq-E1_energymaximizer2} by
Equation~\ref{eq-T1_timeminimizer} for Strategy 1, dividing
Equation~\ref{eq-E2_energymaximizer} by
Equation~\ref{eq-T2_timeminimizer} for Strategy 2, and dividing
Equation~\ref{eq-E3_energymaximizer} by
Equation~\ref{eq-T3_timeminimizer} for strategy 3,

\[
ET_i=\frac{E_i}{T_i}
\]

Similarly to the time minimizer, it's possible to show mathematically
that strategy 2 is never an optimal strategy. So the interesting
comparison is again between strategy 1 and strategy 3. Let's use R to
plot the energy per time yield for both strategies. We start by setting
the parameter values of our model with some realistic numbers.

\begin{Shaded}
\begin{Highlighting}[]
\NormalTok{e1}\OtherTok{\textless{}{-}}\DecValTok{10}                       \CommentTok{\#Caloric content of prey 1}
\NormalTok{e2}\OtherTok{\textless{}{-}}\DecValTok{100}                      \CommentTok{\#Caloric content of prey 2}
\NormalTok{h1}\OtherTok{\textless{}{-}}\DecValTok{1}                        \CommentTok{\#Handling time of prey 1}
\NormalTok{h2}\OtherTok{\textless{}{-}}\DecValTok{60}                       \CommentTok{\#Handling time of prey 2}
\NormalTok{ew}\OtherTok{\textless{}{-}}\DecValTok{1}                        \CommentTok{\#Energy spend per unit time while waiting (cal/s)}
\NormalTok{eh}\OtherTok{\textless{}{-}}\DecValTok{1}                        \CommentTok{\#energy spend per unit time handling the prey (cal/s)}
\end{Highlighting}
\end{Shaded}

We will examine the energy yields for strategy 1 and strategy 3 for a
fixed abundance of prey type 2 and a sequence of abundances of prey type
1 from 0.005 individuals per second to 0.5 individuals per second

\begin{Shaded}
\begin{Highlighting}[]
\NormalTok{a1}\OtherTok{\textless{}{-}} \FunctionTok{seq}\NormalTok{(}\FloatTok{0.005}\NormalTok{,}\FloatTok{0.5}\NormalTok{,}\FloatTok{0.001}\NormalTok{)    }\CommentTok{\#Sequence of abundances of prey 1}
\NormalTok{a2}\OtherTok{\textless{}{-}}\FloatTok{0.05}
\end{Highlighting}
\end{Shaded}

With the parameter and abundance values defined, we can calculate
\(E_1\), \(E_3\), \(T_1\), \(T_3\), and then \(ET_1\) and \(ET_3\) based
on the equations above, resulting in vectors for these variables with
each entry in the vector corresponding to an abundance value in the
vector of abundances \texttt{a1.}

\begin{Shaded}
\begin{Highlighting}[]
\NormalTok{E1 }\OtherTok{=}\NormalTok{ e1 }\SpecialCharTok{{-}}\NormalTok{(eh}\SpecialCharTok{*}\NormalTok{h1) }\SpecialCharTok{{-}}\NormalTok{ ew}\SpecialCharTok{/}\NormalTok{a1    }\CommentTok{\#energy per item of Strategy 1 (s)}
\NormalTok{E3 }\OtherTok{=}\NormalTok{ (e1}\SpecialCharTok{*}\NormalTok{a1}\SpecialCharTok{+}\NormalTok{e2}\SpecialCharTok{*}\NormalTok{a2)}\SpecialCharTok{/}\NormalTok{(a1}\SpecialCharTok{+}\NormalTok{a2)}\SpecialCharTok{{-}}
\NormalTok{     eh}\SpecialCharTok{*}\NormalTok{(h1}\SpecialCharTok{*}\NormalTok{a1}\SpecialCharTok{+}\NormalTok{h2}\SpecialCharTok{*}\NormalTok{a2)}\SpecialCharTok{/}\NormalTok{(a1}\SpecialCharTok{+}\NormalTok{a2)}\SpecialCharTok{{-}}\NormalTok{ew}\SpecialCharTok{/}\NormalTok{(a1}\SpecialCharTok{+}\NormalTok{a2) }\CommentTok{\#energy per item of Strategy 3 (s)}

\NormalTok{T1}\OtherTok{=}\DecValTok{1}\SpecialCharTok{/}\NormalTok{a1}\SpecialCharTok{+}\NormalTok{h1                 }\CommentTok{\#time per item of Strategy 1 (s)}
\NormalTok{T3}\OtherTok{=}\DecValTok{1}\SpecialCharTok{/}\NormalTok{(a1}\SpecialCharTok{+}\NormalTok{a2)}\SpecialCharTok{+}\NormalTok{h1}\SpecialCharTok{*}\NormalTok{a1}\SpecialCharTok{/}\NormalTok{(a1}\SpecialCharTok{+}\NormalTok{a2)}\SpecialCharTok{+}\NormalTok{h2}\SpecialCharTok{*}\NormalTok{a2}\SpecialCharTok{/}\NormalTok{(a1}\SpecialCharTok{+}\NormalTok{a2) }\CommentTok{\#time per item of Strategy 3 (s)}

\NormalTok{ET1 }\OtherTok{=}\NormalTok{ E1}\SpecialCharTok{/}\NormalTok{T1 }\CommentTok{\#energy per time of strategy 1}
\NormalTok{ET3 }\OtherTok{=}\NormalTok{ E3}\SpecialCharTok{/}\NormalTok{T3 }\CommentTok{\#energy per time of strategy 3}
\end{Highlighting}
\end{Shaded}

Next we plot the energy yields against the values of abundance of prey
of type 1, and add a nice legend:

\begin{Shaded}
\begin{Highlighting}[]
\FunctionTok{plot}\NormalTok{(a1,ET1,}\AttributeTok{type=}\StringTok{"l"}\NormalTok{,}\AttributeTok{xlab=}\StringTok{"Abundance of prey 1"}\NormalTok{, }\AttributeTok{ylab=}\StringTok{"Energy per time"}\NormalTok{, }\AttributeTok{col=}\StringTok{"blue"}\NormalTok{)}
\FunctionTok{lines}\NormalTok{(a1,ET3,}\AttributeTok{type=}\StringTok{"l"}\NormalTok{,}\AttributeTok{col=}\StringTok{"red"}\NormalTok{)}
\FunctionTok{legend}\NormalTok{(}\StringTok{"topright"}\NormalTok{,                    }
       \AttributeTok{legend =} \FunctionTok{c}\NormalTok{(}\StringTok{"Strategy 1"}\NormalTok{, }\StringTok{"Strategy 3"}\NormalTok{),  }\CommentTok{\# Labels}
       \AttributeTok{col =} \FunctionTok{c}\NormalTok{(}\StringTok{"red"}\NormalTok{, }\StringTok{"blue"}\NormalTok{),             }\CommentTok{\# Line colors}
       \AttributeTok{lwd =} \DecValTok{2}\NormalTok{,                            }\CommentTok{\# Line width}
       \AttributeTok{lty =} \DecValTok{1}\NormalTok{) }
\end{Highlighting}
\end{Shaded}

\includegraphics{optimalforaging_files/figure-pdf/unnamed-chunk-8-1.pdf}

Similarly to the the time minimizer, for the energy maximizer there is
also a critical threshold of the abundance of prey type 1 above which
strategy 1 is preferrable, while below that threshold strategy 3 is the
best strategy.

\section{Sit-and-wait predator}\label{sit-and-wait-predator}

In contrast to the searching predator, the sit-and-wait predator forages
by patiently ambushing its prey. For instance, the lizard \emph{Anolis
gingivinus} (Figure~\ref{fig-Anolis_gingivinus}) waits in a perch, often
a tree trunk, for a prey to come into its reach, sprinting then down the
trunk or into the ground to capture its prey, returning then to its
perch again. So here the decision that the predator has to take after
seeing the prey is whether it should run an capture the prey or if it
should ignore it.

\begin{figure}

\centering{

\includegraphics{images/Anolis_gingivinus.jpg}

}

\caption{\label{fig-Anolis_gingivinus}An Anolis gingivinus, a
sit-and-wait predator, captures a prey after running down from its
perch. Photo taken in the island of Saint Martin, Leward Islands,
Caribbean.}

\end{figure}%

We will build on the model we developed for the energy per time
maximizer searching predator to develop a model for the decisions of the
sit-and-wait predator. The energy per prey item consumed is,

\begin{equation}\phantomsection\label{eq-energy_sit_and_wait}{
E=e-ep t_p-e_w t_w
}\end{equation}

where \(e_p\) is the energy per unit time while pursuing the prey,
\(t_p\) is the average time it takes to sprint to the prey and come back
to the perch, \(e_w\) is the energy per unit time while waiting for the
prey, and \(t_w\) is the average time it takes to wait for a prey item
to show up within the home-range or territory of the predator. We will
use these two terms interchangeably, although they are sometimes used in
the literature differently.

Let's consider that the home-range of the lizard is shaped as a
semi-circle centered in the perch. This is the lizard has a viewing
angle of \(180\deg\) from its perch (Figure~\ref{fig-feedingterritory}).
Then the total abundance of the the prey in the territory is the
integral of the prey point density over the territory. So, using polar
coordinates for the integral this can be written as,

\[
A=\int_{0}^{r_c}a \pi r dr
\]

The waiting time is just the inverse of the total abundance.

\[t_w=1/A=\frac{1}{\int_{0}^{r_c}a \pi r dr}\]

\begin{figure}

\centering{

\includegraphics{images/FeedingTerritory.jpg}

}

\caption{\label{fig-feedingterritory}Feeding territory of a sit-and-wait
predator with radius \(r_c\).}

\end{figure}%

The average pursuit time is a bit more complicated. For a prey landing
at distance \(r\) from the predator, the pursuit time is the time the
lizards needs to run to that location and back. This is can be
calculated by dividing the double of the distance by the speed of the
lizard, \(2r/v\). We have now to average these sprint times over the
range of possibile distance that the lizard can run, from 0 to the
radius of its territory \(r_c\), weighted by the probability of a prey
appearing at each distance. The probability of a prey appearing at
distance \(r\) is the length of the semi-circle with radius \(r\)
multiplied by the point density of prey divided by the total density of
prey in the territory of the lizard, \(a \pi r / A\). So we now need to
calculate the integral of the sprint times at each distance times the
probability of running to a prey at that distance over the range of
possible radius values,

\[
t_p=\int_{0}^{r_c} \frac{2r}{v} \frac{a \pi r}{A} \ dr
\]

The average time per prey item consumed is simply the sum of the average
pursuit time, \(t_p\), and the average waiting time \(t_w\),

\begin{equation}\phantomsection\label{eq-time_sit_and_wait}{
T=t_p+t_w
}\end{equation}

Therefore the energy per time that the optimal sit-and-wait predator
wants to maximize is obtained by dividing
Equation~\ref{eq-energy_sit_and_wait} by
Equation~\ref{eq-time_sit_and_wait},

\[
ET(r_c)=\frac{e-e_p t_p(r_c)-e_w t_w(r_c)}{t_p(r_c)+t_w(r_c)}
\]

where we highlighted that the wating time and the pursuit time are
functions of the territory size or cut-off radius \(r_c\).

The optimal decision is to find the cut-off radius \(r_c\) that
maximizes the energy per time,

\[
\max_{r_c}(ET)
\]

Let's plot the energy per time as a function of the territory size,

\begin{Shaded}
\begin{Highlighting}[]
\NormalTok{ew}\OtherTok{\textless{}{-}}\FloatTok{0.1}                  \CommentTok{\#Energy per unit of time of waiting}
\NormalTok{ep}\OtherTok{\textless{}{-}}\DecValTok{1}                    \CommentTok{\#Energy per unit of time of pursuing the prey}
\NormalTok{v}\OtherTok{\textless{}{-}}\FloatTok{0.5}                   \CommentTok{\#Velocity of the predator in chase}
\NormalTok{a}\OtherTok{\textless{}{-}}\FloatTok{0.005}                 \CommentTok{\#Abundance of prey}
\NormalTok{e}\OtherTok{\textless{}{-}}\DecValTok{10}                    \CommentTok{\#Energy value of the prey}

\NormalTok{rc}\OtherTok{\textless{}{-}}\FunctionTok{seq}\NormalTok{(}\FloatTok{0.001}\NormalTok{,}\DecValTok{4}\NormalTok{,}\FloatTok{0.001}\NormalTok{)   }\CommentTok{\#Radius of the hunting ground }

\NormalTok{tw }\OtherTok{\textless{}{-}} \DecValTok{1}\SpecialCharTok{/}\NormalTok{(a}\SpecialCharTok{*}\NormalTok{pi}\SpecialCharTok{*}\NormalTok{((rc}\SpecialCharTok{\^{}}\DecValTok{2}\NormalTok{)}\SpecialCharTok{/}\DecValTok{2}\NormalTok{))   }\CommentTok{\#Waiting time }
\NormalTok{tp}\OtherTok{\textless{}{-}}\DecValTok{4}\SpecialCharTok{/}\DecValTok{3}\SpecialCharTok{*}\NormalTok{rc}\SpecialCharTok{/}\NormalTok{v                }\CommentTok{\#Pursuing time}
\NormalTok{ti}\OtherTok{\textless{}{-}}\NormalTok{tw}\SpecialCharTok{+}\NormalTok{tp                   }\CommentTok{\#Time per item }
\NormalTok{ei}\OtherTok{\textless{}{-}}\NormalTok{e}\SpecialCharTok{{-}}\NormalTok{ew}\SpecialCharTok{*}\NormalTok{tw}\SpecialCharTok{{-}}\NormalTok{ep}\SpecialCharTok{*}\NormalTok{tp           }\CommentTok{\#Energy per item}
\NormalTok{et}\OtherTok{\textless{}{-}}\NormalTok{ei}\SpecialCharTok{/}\NormalTok{ti                   }\CommentTok{\#Energy per unit of time}

\NormalTok{et }\OtherTok{\textless{}{-}}\ControlFlowTok{function}\NormalTok{()}
\NormalTok{\{}
\NormalTok{  (e}\SpecialCharTok{{-}}\NormalTok{ew}\SpecialCharTok{*}\NormalTok{(}\DecValTok{1}\SpecialCharTok{/}\NormalTok{(a}\SpecialCharTok{*}\NormalTok{pi}\SpecialCharTok{*}\NormalTok{((rc}\SpecialCharTok{\^{}}\DecValTok{2}\NormalTok{)}\SpecialCharTok{/}\DecValTok{2}\NormalTok{)))}\SpecialCharTok{{-}}\NormalTok{ep}\SpecialCharTok{*}\NormalTok{(}\DecValTok{4}\SpecialCharTok{/}\DecValTok{3}\SpecialCharTok{*}\NormalTok{rc}\SpecialCharTok{/}\NormalTok{v))}\SpecialCharTok{/}\NormalTok{(}\DecValTok{1}\SpecialCharTok{/}\NormalTok{(a}\SpecialCharTok{*}\NormalTok{pi}\SpecialCharTok{*}\NormalTok{((rc}\SpecialCharTok{\^{}}\DecValTok{2}\NormalTok{)}\SpecialCharTok{/}\DecValTok{2}\NormalTok{))}\SpecialCharTok{+}\DecValTok{4}\SpecialCharTok{/}\DecValTok{3}\SpecialCharTok{*}\NormalTok{rc}\SpecialCharTok{/}\NormalTok{v)}
\NormalTok{\}}

\FunctionTok{plot}\NormalTok{(rc,}\FunctionTok{et}\NormalTok{(),}\AttributeTok{type=}\StringTok{"l"}\NormalTok{,}
     \AttributeTok{xlab=}\StringTok{"Radius of the territory"}\NormalTok{,}
     \AttributeTok{ylab=}\StringTok{"Energy per unit of time (cal/sec) "}\NormalTok{)}
 \FunctionTok{abline}\NormalTok{(}\DecValTok{0}\NormalTok{,}\DecValTok{0}\NormalTok{)}
\end{Highlighting}
\end{Shaded}

\includegraphics{optimalforaging_files/figure-pdf/unnamed-chunk-9-1.pdf}

\section{Statistical confrontation: finding the
maximum}\label{statistical-confrontation-finding-the-maximum}

\chapter{Conflict resolution in ecology and evolution: war and
peace}\label{conflict-resolution-in-ecology-and-evolution-war-and-peace}

Frequency dependent selection. Game theory and evolutionary stable
strategies. Non-cooperative games: payoffs, Nash equilibrium, genetic
population model. 2 player games, discrete strategies: Hawk-Dove;
Prisioner's dillema`ndefinedndefinedndefinedSearching predator

\chapter{\texorpdfstring{Movement and dispersal\textbf{:} should I stay
or should I
go?}{Movement and dispersal: should I stay or should I go?}}\label{movement-and-dispersal-should-i-stay-or-should-i-go}

Diffusion dispersal models as random-walk. Interactions between neighbor
individuals (e.g.~competition for space or resources). The ideal free
distribution; The ideal despotic distribution

\part{Ecology of populations}

\chapter{Demography: boom or bust}\label{sec-demography}

\part{Ecology of communities}

\chapter{Modelling biodiversity change: winners and
losers}\label{modelling-biodiversity-change-winners-and-losers}

Every minute about 2 football fields of native forest are cleared
somewhere in the world. What are the consequences of land-use change for
biodiversity?

\cleardoublepage
\phantomsection
\addcontentsline{toc}{part}{Appendices}
\appendix

\chapter{A brief R tutorial}\label{a-brief-r-tutorial}

If you are new to \textbf{R} you can have a short dive into its main
features by working through this tutorial. If you had learnt programming
in another computer language, you will be able to skim over this
tutorial to find the main differences from what you have learnt to how
things are done in \textbf{R.}

\section{Variables, vectors and
matrices}\label{variables-vectors-and-matrices}

\begin{Shaded}
\begin{Highlighting}[]
\CommentTok{\# Introduction to variables}
\CommentTok{\# Variables can be any sequence of letter and numbers, but }
\CommentTok{\# it cannot start with a number}
\NormalTok{x }\OtherTok{=} \DecValTok{2}
\NormalTok{x }\OtherTok{\textless{}{-}} \DecValTok{4}
\DecValTok{2}\SpecialCharTok{+}\DecValTok{2} 
\end{Highlighting}
\end{Shaded}

\begin{verbatim}
[1] 4
\end{verbatim}

\begin{Shaded}
\begin{Highlighting}[]
\NormalTok{y }\OtherTok{\textless{}{-}}\NormalTok{ x}\SpecialCharTok{\^{}}\DecValTok{5}
\NormalTok{y}
\end{Highlighting}
\end{Shaded}

\begin{verbatim}
[1] 1024
\end{verbatim}

\begin{Shaded}
\begin{Highlighting}[]
\CommentTok{\# Introduction to vectors}
\NormalTok{v1 }\OtherTok{\textless{}{-}} \FunctionTok{c}\NormalTok{(}\DecValTok{2}\NormalTok{,}\DecValTok{3}\NormalTok{,}\DecValTok{6}\NormalTok{,}\DecValTok{12}\NormalTok{)}
\NormalTok{v2 }\OtherTok{\textless{}{-}} \DecValTok{1}\SpecialCharTok{:}\DecValTok{100}
\FunctionTok{length}\NormalTok{(v2)}
\end{Highlighting}
\end{Shaded}

\begin{verbatim}
[1] 100
\end{verbatim}

\begin{Shaded}
\begin{Highlighting}[]
\NormalTok{v2}
\end{Highlighting}
\end{Shaded}

\begin{verbatim}
  [1]   1   2   3   4   5   6   7   8   9  10  11  12  13  14  15  16  17  18
 [19]  19  20  21  22  23  24  25  26  27  28  29  30  31  32  33  34  35  36
 [37]  37  38  39  40  41  42  43  44  45  46  47  48  49  50  51  52  53  54
 [55]  55  56  57  58  59  60  61  62  63  64  65  66  67  68  69  70  71  72
 [73]  73  74  75  76  77  78  79  80  81  82  83  84  85  86  87  88  89  90
 [91]  91  92  93  94  95  96  97  98  99 100
\end{verbatim}

\begin{Shaded}
\begin{Highlighting}[]
\NormalTok{v3 }\OtherTok{\textless{}{-}} \FunctionTok{seq}\NormalTok{(}\DecValTok{1}\NormalTok{,}\DecValTok{100}\NormalTok{,}\DecValTok{5}\NormalTok{)  }\CommentTok{\# call without naming arguments}
\NormalTok{v3}
\end{Highlighting}
\end{Shaded}

\begin{verbatim}
 [1]  1  6 11 16 21 26 31 36 41 46 51 56 61 66 71 76 81 86 91 96
\end{verbatim}

\begin{Shaded}
\begin{Highlighting}[]
\NormalTok{v3 }\OtherTok{\textless{}{-}} \FunctionTok{seq}\NormalTok{(}\AttributeTok{from=}\DecValTok{1}\NormalTok{,}\AttributeTok{to=}\DecValTok{100}\NormalTok{,}\AttributeTok{by=}\DecValTok{5}\NormalTok{) }\CommentTok{\# call with names of arguments}
\NormalTok{v3}
\end{Highlighting}
\end{Shaded}

\begin{verbatim}
 [1]  1  6 11 16 21 26 31 36 41 46 51 56 61 66 71 76 81 86 91 96
\end{verbatim}

\begin{Shaded}
\begin{Highlighting}[]
\NormalTok{v3 }\OtherTok{\textless{}{-}} \FunctionTok{seq}\NormalTok{(}\AttributeTok{to=}\DecValTok{100}\NormalTok{,}\AttributeTok{by=}\DecValTok{5}\NormalTok{) }\CommentTok{\# call skipping the first argument}
\CommentTok{\#and using the default value 1 {-} see help(seq)}
\NormalTok{v3}
\end{Highlighting}
\end{Shaded}

\begin{verbatim}
 [1]  1  6 11 16 21 26 31 36 41 46 51 56 61 66 71 76 81 86 91 96
\end{verbatim}

\begin{Shaded}
\begin{Highlighting}[]
\NormalTok{v3 }\OtherTok{\textless{}{-}} \FunctionTok{seq}\NormalTok{(}\AttributeTok{by=}\DecValTok{5}\NormalTok{,}\AttributeTok{to=}\DecValTok{100}\NormalTok{) }\CommentTok{\# call by arguments and change order or arguments}

\CommentTok{\# Indexing vectors}
\NormalTok{v3[}\DecValTok{3}\NormalTok{] }\CommentTok{\#uses square brackets to obtain the third element of the vector}
\end{Highlighting}
\end{Shaded}

\begin{verbatim}
[1] 11
\end{verbatim}

\begin{Shaded}
\begin{Highlighting}[]
\NormalTok{v3}\SpecialCharTok{\textgreater{}}\DecValTok{20} \CommentTok{\# produce a vector of boolean values that are TRUE when}
\end{Highlighting}
\end{Shaded}

\begin{verbatim}
 [1] FALSE FALSE FALSE FALSE  TRUE  TRUE  TRUE  TRUE  TRUE  TRUE  TRUE  TRUE
[13]  TRUE  TRUE  TRUE  TRUE  TRUE  TRUE  TRUE  TRUE
\end{verbatim}

\begin{Shaded}
\begin{Highlighting}[]
      \CommentTok{\#v3 is greater than 20}
\NormalTok{v3[v3}\SpecialCharTok{\textgreater{}}\DecValTok{20}\NormalTok{] }\CommentTok{\# select from v3 all the values that are greater than 20}
\end{Highlighting}
\end{Shaded}

\begin{verbatim}
 [1] 21 26 31 36 41 46 51 56 61 66 71 76 81 86 91 96
\end{verbatim}

\begin{Shaded}
\begin{Highlighting}[]
\NormalTok{v4}\OtherTok{\textless{}{-}}\FunctionTok{c}\NormalTok{(}\DecValTok{1}\NormalTok{,}\DecValTok{2}\NormalTok{,}\DecValTok{3}\NormalTok{,}\DecValTok{4}\NormalTok{,}\DecValTok{5}\NormalTok{)}
\NormalTok{v4[}\FunctionTok{c}\NormalTok{(}\ConstantTok{FALSE}\NormalTok{,}\ConstantTok{TRUE}\NormalTok{,}\ConstantTok{FALSE}\NormalTok{,}\ConstantTok{TRUE}\NormalTok{,}\ConstantTok{FALSE}\NormalTok{)] }\CommentTok{\#select from v4 the second and fourth element}
\end{Highlighting}
\end{Shaded}

\begin{verbatim}
[1] 2 4
\end{verbatim}

\begin{Shaded}
\begin{Highlighting}[]
\NormalTok{v3[}\DecValTok{1}\SpecialCharTok{:}\DecValTok{10}\NormalTok{] }\CommentTok{\# first ten elements}
\end{Highlighting}
\end{Shaded}

\begin{verbatim}
 [1]  1  6 11 16 21 26 31 36 41 46
\end{verbatim}

\begin{Shaded}
\begin{Highlighting}[]
\NormalTok{v3[}\SpecialCharTok{{-}}\DecValTok{1}\NormalTok{] }\CommentTok{\# dropping first element}
\end{Highlighting}
\end{Shaded}

\begin{verbatim}
 [1]  6 11 16 21 26 31 36 41 46 51 56 61 66 71 76 81 86 91 96
\end{verbatim}

\begin{Shaded}
\begin{Highlighting}[]
\FunctionTok{head}\NormalTok{(v2) }\CommentTok{\# prints the first few elements of v2}
\end{Highlighting}
\end{Shaded}

\begin{verbatim}
[1] 1 2 3 4 5 6
\end{verbatim}

\begin{Shaded}
\begin{Highlighting}[]
\FunctionTok{tail}\NormalTok{(v2) }\CommentTok{\# prints the last few elements of v2}
\end{Highlighting}
\end{Shaded}

\begin{verbatim}
[1]  95  96  97  98  99 100
\end{verbatim}

\begin{Shaded}
\begin{Highlighting}[]
\FunctionTok{which}\NormalTok{(v3 }\SpecialCharTok{==} \DecValTok{26}\NormalTok{) }\CommentTok{\# returns the position of v3 that equals 26}
\end{Highlighting}
\end{Shaded}

\begin{verbatim}
[1] 6
\end{verbatim}

\begin{Shaded}
\begin{Highlighting}[]
\CommentTok{\#Numerical operations with vectors}
\DecValTok{2}\SpecialCharTok{\^{}}\NormalTok{v2}
\end{Highlighting}
\end{Shaded}

\begin{verbatim}
  [1] 2.000000e+00 4.000000e+00 8.000000e+00 1.600000e+01 3.200000e+01
  [6] 6.400000e+01 1.280000e+02 2.560000e+02 5.120000e+02 1.024000e+03
 [11] 2.048000e+03 4.096000e+03 8.192000e+03 1.638400e+04 3.276800e+04
 [16] 6.553600e+04 1.310720e+05 2.621440e+05 5.242880e+05 1.048576e+06
 [21] 2.097152e+06 4.194304e+06 8.388608e+06 1.677722e+07 3.355443e+07
 [26] 6.710886e+07 1.342177e+08 2.684355e+08 5.368709e+08 1.073742e+09
 [31] 2.147484e+09 4.294967e+09 8.589935e+09 1.717987e+10 3.435974e+10
 [36] 6.871948e+10 1.374390e+11 2.748779e+11 5.497558e+11 1.099512e+12
 [41] 2.199023e+12 4.398047e+12 8.796093e+12 1.759219e+13 3.518437e+13
 [46] 7.036874e+13 1.407375e+14 2.814750e+14 5.629500e+14 1.125900e+15
 [51] 2.251800e+15 4.503600e+15 9.007199e+15 1.801440e+16 3.602880e+16
 [56] 7.205759e+16 1.441152e+17 2.882304e+17 5.764608e+17 1.152922e+18
 [61] 2.305843e+18 4.611686e+18 9.223372e+18 1.844674e+19 3.689349e+19
 [66] 7.378698e+19 1.475740e+20 2.951479e+20 5.902958e+20 1.180592e+21
 [71] 2.361183e+21 4.722366e+21 9.444733e+21 1.888947e+22 3.777893e+22
 [76] 7.555786e+22 1.511157e+23 3.022315e+23 6.044629e+23 1.208926e+24
 [81] 2.417852e+24 4.835703e+24 9.671407e+24 1.934281e+25 3.868563e+25
 [86] 7.737125e+25 1.547425e+26 3.094850e+26 6.189700e+26 1.237940e+27
 [91] 2.475880e+27 4.951760e+27 9.903520e+27 1.980704e+28 3.961408e+28
 [96] 7.922816e+28 1.584563e+29 3.169127e+29 6.338253e+29 1.267651e+30
\end{verbatim}

\begin{Shaded}
\begin{Highlighting}[]
\FunctionTok{log}\NormalTok{(v2)}
\end{Highlighting}
\end{Shaded}

\begin{verbatim}
  [1] 0.0000000 0.6931472 1.0986123 1.3862944 1.6094379 1.7917595 1.9459101
  [8] 2.0794415 2.1972246 2.3025851 2.3978953 2.4849066 2.5649494 2.6390573
 [15] 2.7080502 2.7725887 2.8332133 2.8903718 2.9444390 2.9957323 3.0445224
 [22] 3.0910425 3.1354942 3.1780538 3.2188758 3.2580965 3.2958369 3.3322045
 [29] 3.3672958 3.4011974 3.4339872 3.4657359 3.4965076 3.5263605 3.5553481
 [36] 3.5835189 3.6109179 3.6375862 3.6635616 3.6888795 3.7135721 3.7376696
 [43] 3.7612001 3.7841896 3.8066625 3.8286414 3.8501476 3.8712010 3.8918203
 [50] 3.9120230 3.9318256 3.9512437 3.9702919 3.9889840 4.0073332 4.0253517
 [57] 4.0430513 4.0604430 4.0775374 4.0943446 4.1108739 4.1271344 4.1431347
 [64] 4.1588831 4.1743873 4.1896547 4.2046926 4.2195077 4.2341065 4.2484952
 [71] 4.2626799 4.2766661 4.2904594 4.3040651 4.3174881 4.3307333 4.3438054
 [78] 4.3567088 4.3694479 4.3820266 4.3944492 4.4067192 4.4188406 4.4308168
 [85] 4.4426513 4.4543473 4.4659081 4.4773368 4.4886364 4.4998097 4.5108595
 [92] 4.5217886 4.5325995 4.5432948 4.5538769 4.5643482 4.5747110 4.5849675
 [99] 4.5951199 4.6051702
\end{verbatim}

\begin{Shaded}
\begin{Highlighting}[]
\NormalTok{v5 }\OtherTok{\textless{}{-}} \DecValTok{101}\SpecialCharTok{:}\DecValTok{200}
\NormalTok{v5}\SpecialCharTok{{-}}\NormalTok{v2}
\end{Highlighting}
\end{Shaded}

\begin{verbatim}
  [1] 100 100 100 100 100 100 100 100 100 100 100 100 100 100 100 100 100 100
 [19] 100 100 100 100 100 100 100 100 100 100 100 100 100 100 100 100 100 100
 [37] 100 100 100 100 100 100 100 100 100 100 100 100 100 100 100 100 100 100
 [55] 100 100 100 100 100 100 100 100 100 100 100 100 100 100 100 100 100 100
 [73] 100 100 100 100 100 100 100 100 100 100 100 100 100 100 100 100 100 100
 [91] 100 100 100 100 100 100 100 100 100 100
\end{verbatim}

\begin{Shaded}
\begin{Highlighting}[]
\NormalTok{v5}\SpecialCharTok{/}\NormalTok{v2}
\end{Highlighting}
\end{Shaded}

\begin{verbatim}
  [1] 101.000000  51.000000  34.333333  26.000000  21.000000  17.666667
  [7]  15.285714  13.500000  12.111111  11.000000  10.090909   9.333333
 [13]   8.692308   8.142857   7.666667   7.250000   6.882353   6.555556
 [19]   6.263158   6.000000   5.761905   5.545455   5.347826   5.166667
 [25]   5.000000   4.846154   4.703704   4.571429   4.448276   4.333333
 [31]   4.225806   4.125000   4.030303   3.941176   3.857143   3.777778
 [37]   3.702703   3.631579   3.564103   3.500000   3.439024   3.380952
 [43]   3.325581   3.272727   3.222222   3.173913   3.127660   3.083333
 [49]   3.040816   3.000000   2.960784   2.923077   2.886792   2.851852
 [55]   2.818182   2.785714   2.754386   2.724138   2.694915   2.666667
 [61]   2.639344   2.612903   2.587302   2.562500   2.538462   2.515152
 [67]   2.492537   2.470588   2.449275   2.428571   2.408451   2.388889
 [73]   2.369863   2.351351   2.333333   2.315789   2.298701   2.282051
 [79]   2.265823   2.250000   2.234568   2.219512   2.204819   2.190476
 [85]   2.176471   2.162791   2.149425   2.136364   2.123596   2.111111
 [91]   2.098901   2.086957   2.075269   2.063830   2.052632   2.041667
 [97]   2.030928   2.020408   2.010101   2.000000
\end{verbatim}

\begin{Shaded}
\begin{Highlighting}[]
\NormalTok{v1}\SpecialCharTok{/}\NormalTok{v2}
\end{Highlighting}
\end{Shaded}

\begin{verbatim}
  [1] 2.00000000 1.50000000 2.00000000 3.00000000 0.40000000 0.50000000
  [7] 0.85714286 1.50000000 0.22222222 0.30000000 0.54545455 1.00000000
 [13] 0.15384615 0.21428571 0.40000000 0.75000000 0.11764706 0.16666667
 [19] 0.31578947 0.60000000 0.09523810 0.13636364 0.26086957 0.50000000
 [25] 0.08000000 0.11538462 0.22222222 0.42857143 0.06896552 0.10000000
 [31] 0.19354839 0.37500000 0.06060606 0.08823529 0.17142857 0.33333333
 [37] 0.05405405 0.07894737 0.15384615 0.30000000 0.04878049 0.07142857
 [43] 0.13953488 0.27272727 0.04444444 0.06521739 0.12765957 0.25000000
 [49] 0.04081633 0.06000000 0.11764706 0.23076923 0.03773585 0.05555556
 [55] 0.10909091 0.21428571 0.03508772 0.05172414 0.10169492 0.20000000
 [61] 0.03278689 0.04838710 0.09523810 0.18750000 0.03076923 0.04545455
 [67] 0.08955224 0.17647059 0.02898551 0.04285714 0.08450704 0.16666667
 [73] 0.02739726 0.04054054 0.08000000 0.15789474 0.02597403 0.03846154
 [79] 0.07594937 0.15000000 0.02469136 0.03658537 0.07228916 0.14285714
 [85] 0.02352941 0.03488372 0.06896552 0.13636364 0.02247191 0.03333333
 [91] 0.06593407 0.13043478 0.02150538 0.03191489 0.06315789 0.12500000
 [97] 0.02061856 0.03061224 0.06060606 0.12000000
\end{verbatim}

\begin{Shaded}
\begin{Highlighting}[]
\CommentTok{\#Using strings in R}
\NormalTok{mystring }\OtherTok{\textless{}{-}} \StringTok{"Ecology"}
\NormalTok{vstrg }\OtherTok{\textless{}{-}} \FunctionTok{c}\NormalTok{(}\StringTok{"Anna"}\NormalTok{, }\StringTok{"Peter"}\NormalTok{, }\StringTok{"Xavier"}\NormalTok{)}
\NormalTok{vstrg[}\DecValTok{2}\NormalTok{]}
\end{Highlighting}
\end{Shaded}

\begin{verbatim}
[1] "Peter"
\end{verbatim}

\begin{Shaded}
\begin{Highlighting}[]
\CommentTok{\#making plots in R}
\FunctionTok{plot}\NormalTok{(v2,v2)}
\end{Highlighting}
\end{Shaded}

\includegraphics{intro_to_R_files/figure-pdf/unnamed-chunk-1-1.pdf}

\begin{Shaded}
\begin{Highlighting}[]
\FunctionTok{plot}\NormalTok{(v2,v2}\SpecialCharTok{\^{}}\DecValTok{2}\NormalTok{)}
\end{Highlighting}
\end{Shaded}

\includegraphics{intro_to_R_files/figure-pdf/unnamed-chunk-1-2.pdf}

\begin{Shaded}
\begin{Highlighting}[]
\FunctionTok{plot}\NormalTok{(v2,v2}\SpecialCharTok{\^{}}\DecValTok{2}\NormalTok{,}\AttributeTok{type=}\StringTok{"l"}\NormalTok{)}
\end{Highlighting}
\end{Shaded}

\includegraphics{intro_to_R_files/figure-pdf/unnamed-chunk-1-3.pdf}

\begin{Shaded}
\begin{Highlighting}[]
\FunctionTok{plot}\NormalTok{(v2,v2}\SpecialCharTok{\^{}}\DecValTok{2}\NormalTok{,}\AttributeTok{type=}\StringTok{"l"}\NormalTok{,}\AttributeTok{col=}\StringTok{"red"}\NormalTok{)}
\end{Highlighting}
\end{Shaded}

\includegraphics{intro_to_R_files/figure-pdf/unnamed-chunk-1-4.pdf}

\begin{Shaded}
\begin{Highlighting}[]
\FunctionTok{plot}\NormalTok{(v2,v2}\SpecialCharTok{\^{}}\DecValTok{2}\NormalTok{,}\AttributeTok{type=}\StringTok{"l"}\NormalTok{,}\AttributeTok{col=}\StringTok{"red"}\NormalTok{,}\AttributeTok{main=}\StringTok{"My beautiful plot"}\NormalTok{)}
\end{Highlighting}
\end{Shaded}

\includegraphics{intro_to_R_files/figure-pdf/unnamed-chunk-1-5.pdf}

\begin{Shaded}
\begin{Highlighting}[]
\FunctionTok{plot}\NormalTok{(v2,v2}\SpecialCharTok{\^{}}\DecValTok{2}\NormalTok{,}\AttributeTok{type=}\StringTok{"l"}\NormalTok{,}\AttributeTok{col=}\StringTok{"red"}\NormalTok{,}\AttributeTok{main=}\StringTok{"My beautiful plot"}\NormalTok{,}\AttributeTok{xlab=}\StringTok{"x"}\NormalTok{,}
     \AttributeTok{ylab=}\StringTok{"x\^{}2"}\NormalTok{)}
\FunctionTok{lines}\NormalTok{(v2,v2}\SpecialCharTok{\^{}}\DecValTok{3}\NormalTok{,}\AttributeTok{col=}\StringTok{"blue"}\NormalTok{)}
\end{Highlighting}
\end{Shaded}

\includegraphics{intro_to_R_files/figure-pdf/unnamed-chunk-1-6.pdf}

\begin{Shaded}
\begin{Highlighting}[]
\CommentTok{\# Matrices in R}
\NormalTok{m }\OtherTok{\textless{}{-}} \FunctionTok{matrix}\NormalTok{(}\DecValTok{5}\NormalTok{,}\DecValTok{3}\NormalTok{,}\DecValTok{2}\NormalTok{)}
\NormalTok{m}
\end{Highlighting}
\end{Shaded}

\begin{verbatim}
     [,1] [,2]
[1,]    5    5
[2,]    5    5
[3,]    5    5
\end{verbatim}

\begin{Shaded}
\begin{Highlighting}[]
\NormalTok{m2 }\OtherTok{\textless{}{-}} \FunctionTok{matrix}\NormalTok{(}\DecValTok{1}\SpecialCharTok{:}\DecValTok{6}\NormalTok{,}\DecValTok{3}\NormalTok{,}\DecValTok{2}\NormalTok{)}
\NormalTok{m2}
\end{Highlighting}
\end{Shaded}

\begin{verbatim}
     [,1] [,2]
[1,]    1    4
[2,]    2    5
[3,]    3    6
\end{verbatim}

\begin{Shaded}
\begin{Highlighting}[]
\FunctionTok{t}\NormalTok{(m2) }\CommentTok{\# transposes matrix}
\end{Highlighting}
\end{Shaded}

\begin{verbatim}
     [,1] [,2] [,3]
[1,]    1    2    3
[2,]    4    5    6
\end{verbatim}

\begin{Shaded}
\begin{Highlighting}[]
\NormalTok{x }\OtherTok{\textless{}{-}} \DecValTok{1}\SpecialCharTok{:}\DecValTok{4}
\NormalTok{y }\OtherTok{\textless{}{-}} \DecValTok{5}\SpecialCharTok{:}\DecValTok{8}

\NormalTok{m3}\OtherTok{\textless{}{-}}\FunctionTok{cbind}\NormalTok{(x,y)}
\NormalTok{m3}
\end{Highlighting}
\end{Shaded}

\begin{verbatim}
     x y
[1,] 1 5
[2,] 2 6
[3,] 3 7
[4,] 4 8
\end{verbatim}

\begin{Shaded}
\begin{Highlighting}[]
\NormalTok{m4}\OtherTok{\textless{}{-}}\FunctionTok{rbind}\NormalTok{(x,y)}
\NormalTok{m4}
\end{Highlighting}
\end{Shaded}

\begin{verbatim}
  [,1] [,2] [,3] [,4]
x    1    2    3    4
y    5    6    7    8
\end{verbatim}

\begin{Shaded}
\begin{Highlighting}[]
\CommentTok{\# Indexing matrices}
\NormalTok{m3[}\DecValTok{3}\NormalTok{,}\DecValTok{2}\NormalTok{] }\CommentTok{\#element in row 3 and column 2}
\end{Highlighting}
\end{Shaded}

\begin{verbatim}
y 
7 
\end{verbatim}

\begin{Shaded}
\begin{Highlighting}[]
\NormalTok{m3[}\DecValTok{1}\NormalTok{,] }\CommentTok{\#entire first row}
\end{Highlighting}
\end{Shaded}

\begin{verbatim}
x y 
1 5 
\end{verbatim}

\begin{Shaded}
\begin{Highlighting}[]
\NormalTok{m3[,}\DecValTok{1}\NormalTok{] }\CommentTok{\#entire first column}
\end{Highlighting}
\end{Shaded}

\begin{verbatim}
[1] 1 2 3 4
\end{verbatim}

\begin{Shaded}
\begin{Highlighting}[]
\FunctionTok{colnames}\NormalTok{(m3)}\OtherTok{\textless{}{-}}\FunctionTok{c}\NormalTok{(}\StringTok{"col1"}\NormalTok{,}\StringTok{"col2"}\NormalTok{)}
\NormalTok{m3}
\end{Highlighting}
\end{Shaded}

\begin{verbatim}
     col1 col2
[1,]    1    5
[2,]    2    6
[3,]    3    7
[4,]    4    8
\end{verbatim}

\begin{Shaded}
\begin{Highlighting}[]
\NormalTok{m3[,}\StringTok{"col2"}\NormalTok{]}
\end{Highlighting}
\end{Shaded}

\begin{verbatim}
[1] 5 6 7 8
\end{verbatim}

\begin{Shaded}
\begin{Highlighting}[]
\CommentTok{\# Lists in R}
\NormalTok{mylist }\OtherTok{\textless{}{-}} \FunctionTok{list}\NormalTok{(}\AttributeTok{elem1=}\NormalTok{m,}\AttributeTok{elem2=}\NormalTok{v2,}\AttributeTok{elem3=}\StringTok{"my list"}\NormalTok{)}
\NormalTok{mylist}\SpecialCharTok{$}\NormalTok{elem2}
\end{Highlighting}
\end{Shaded}

\begin{verbatim}
  [1]   1   2   3   4   5   6   7   8   9  10  11  12  13  14  15  16  17  18
 [19]  19  20  21  22  23  24  25  26  27  28  29  30  31  32  33  34  35  36
 [37]  37  38  39  40  41  42  43  44  45  46  47  48  49  50  51  52  53  54
 [55]  55  56  57  58  59  60  61  62  63  64  65  66  67  68  69  70  71  72
 [73]  73  74  75  76  77  78  79  80  81  82  83  84  85  86  87  88  89  90
 [91]  91  92  93  94  95  96  97  98  99 100
\end{verbatim}

\begin{Shaded}
\begin{Highlighting}[]
\CommentTok{\# Dataframes}
\NormalTok{df }\OtherTok{\textless{}{-}} \FunctionTok{as.data.frame}\NormalTok{(m3)}
\NormalTok{df}\SpecialCharTok{$}\NormalTok{col1}
\end{Highlighting}
\end{Shaded}

\begin{verbatim}
[1] 1 2 3 4
\end{verbatim}

\section{Iterations and conditional
expressions}\label{iterations-and-conditional-expressions}

\begin{Shaded}
\begin{Highlighting}[]
\CommentTok{\# FOR loops}

\ControlFlowTok{for}\NormalTok{ (k }\ControlFlowTok{in} \DecValTok{1}\SpecialCharTok{:}\DecValTok{10}\NormalTok{)  }\CommentTok{\# for k =1, 2, 3, 4, 5,...10}
  \FunctionTok{print}\NormalTok{ (k}\SpecialCharTok{\^{}}\DecValTok{2}\NormalTok{)   }\CommentTok{\#do this}
\end{Highlighting}
\end{Shaded}

\begin{verbatim}
[1] 1
[1] 4
[1] 9
[1] 16
[1] 25
[1] 36
[1] 49
[1] 64
[1] 81
[1] 100
\end{verbatim}

\begin{Shaded}
\begin{Highlighting}[]
\NormalTok{R }\OtherTok{\textless{}{-}} \FloatTok{1.2}
\NormalTok{n }\OtherTok{\textless{}{-}} \DecValTok{1}
\FunctionTok{print}\NormalTok{(n[}\DecValTok{1}\NormalTok{])}
\end{Highlighting}
\end{Shaded}

\begin{verbatim}
[1] 1
\end{verbatim}

\begin{Shaded}
\begin{Highlighting}[]
\ControlFlowTok{for}\NormalTok{ (t }\ControlFlowTok{in} \DecValTok{1}\SpecialCharTok{:}\DecValTok{100}\NormalTok{)}
\NormalTok{\{}
\NormalTok{  n[t}\SpecialCharTok{+}\DecValTok{1}\NormalTok{] }\OtherTok{\textless{}{-}}\NormalTok{ R}\SpecialCharTok{*}\NormalTok{n[t]}
  \FunctionTok{print}\NormalTok{(n[t}\SpecialCharTok{+}\DecValTok{1}\NormalTok{])}
\NormalTok{\}}
\end{Highlighting}
\end{Shaded}

\begin{verbatim}
[1] 1.2
[1] 1.44
[1] 1.728
[1] 2.0736
[1] 2.48832
[1] 2.985984
[1] 3.583181
[1] 4.299817
[1] 5.15978
[1] 6.191736
[1] 7.430084
[1] 8.9161
[1] 10.69932
[1] 12.83918
[1] 15.40702
[1] 18.48843
[1] 22.18611
[1] 26.62333
[1] 31.948
[1] 38.3376
[1] 46.00512
[1] 55.20614
[1] 66.24737
[1] 79.49685
[1] 95.39622
[1] 114.4755
[1] 137.3706
[1] 164.8447
[1] 197.8136
[1] 237.3763
[1] 284.8516
[1] 341.8219
[1] 410.1863
[1] 492.2235
[1] 590.6682
[1] 708.8019
[1] 850.5622
[1] 1020.675
[1] 1224.81
[1] 1469.772
[1] 1763.726
[1] 2116.471
[1] 2539.765
[1] 3047.718
[1] 3657.262
[1] 4388.714
[1] 5266.457
[1] 6319.749
[1] 7583.698
[1] 9100.438
[1] 10920.53
[1] 13104.63
[1] 15725.56
[1] 18870.67
[1] 22644.8
[1] 27173.76
[1] 32608.52
[1] 39130.22
[1] 46956.26
[1] 56347.51
[1] 67617.02
[1] 81140.42
[1] 97368.5
[1] 116842.2
[1] 140210.6
[1] 168252.8
[1] 201903.3
[1] 242284
[1] 290740.8
[1] 348889
[1] 418666.7
[1] 502400.1
[1] 602880.1
[1] 723456.1
[1] 868147.4
[1] 1041777
[1] 1250132
[1] 1500159
[1] 1800190
[1] 2160228
[1] 2592274
[1] 3110729
[1] 3732875
[1] 4479450
[1] 5375340
[1] 6450408
[1] 7740489
[1] 9288587
[1] 11146304
[1] 13375565
[1] 16050678
[1] 19260814
[1] 23112977
[1] 27735572
[1] 33282687
[1] 39939224
[1] 47927069
[1] 57512482
[1] 69014979
[1] 82817975
\end{verbatim}

\begin{Shaded}
\begin{Highlighting}[]
\NormalTok{R }\OtherTok{\textless{}{-}} \FloatTok{1.2}
\NormalTok{n }\OtherTok{\textless{}{-}} \DecValTok{1}
\ControlFlowTok{for}\NormalTok{ (t }\ControlFlowTok{in} \DecValTok{1}\SpecialCharTok{:}\DecValTok{100}\NormalTok{)}
\NormalTok{  n[t}\SpecialCharTok{+}\DecValTok{1}\NormalTok{] }\OtherTok{\textless{}{-}}\NormalTok{ R}\SpecialCharTok{*}\NormalTok{n[t]}

\CommentTok{\# IF conditional statement}

\CommentTok{\# logical operators}
\CommentTok{\# == equal to}
\CommentTok{\# \textgreater{} greater than}
\CommentTok{\# \textless{} smaller than}
\CommentTok{\# \textgreater{}= greater or equal}
\CommentTok{\# \textless{}= smaller or equal}
\CommentTok{\# != different from}
\CommentTok{\# \&\& and}
\CommentTok{\# || or}

\ControlFlowTok{if}\NormalTok{ (}\DecValTok{3}\SpecialCharTok{\textgreater{}}\DecValTok{2}\NormalTok{) }\FunctionTok{print}\NormalTok{ (}\StringTok{"yes"}\NormalTok{)}
\end{Highlighting}
\end{Shaded}

\begin{verbatim}
[1] "yes"
\end{verbatim}

\begin{Shaded}
\begin{Highlighting}[]
\ControlFlowTok{if}\NormalTok{ (}\DecValTok{3}\SpecialCharTok{==}\DecValTok{2}\NormalTok{) }\FunctionTok{print}\NormalTok{ (}\StringTok{"yes"}\NormalTok{) }\ControlFlowTok{else} \FunctionTok{print}\NormalTok{(}\StringTok{"no"}\NormalTok{)}
\end{Highlighting}
\end{Shaded}

\begin{verbatim}
[1] "no"
\end{verbatim}

\begin{Shaded}
\begin{Highlighting}[]
\ControlFlowTok{if}\NormalTok{ ((}\DecValTok{3}\SpecialCharTok{\textgreater{}}\DecValTok{2}\NormalTok{)}\SpecialCharTok{\&\&}\NormalTok{(}\DecValTok{4}\SpecialCharTok{\textgreater{}}\DecValTok{5}\NormalTok{)) }\FunctionTok{print}\NormalTok{ (}\StringTok{"yes"}\NormalTok{)}

\ControlFlowTok{for}\NormalTok{ (k }\ControlFlowTok{in} \DecValTok{1}\SpecialCharTok{:}\DecValTok{10}\NormalTok{)  }\CommentTok{\# for k =1, 2, 3, 4, 5,...10}
  \ControlFlowTok{if}\NormalTok{ (k}\SpecialCharTok{\^{}}\DecValTok{2}\SpecialCharTok{\textgreater{}}\DecValTok{20}\NormalTok{) }\FunctionTok{print}\NormalTok{ (k}\SpecialCharTok{\^{}}\DecValTok{2}\NormalTok{)  }
\end{Highlighting}
\end{Shaded}

\begin{verbatim}
[1] 25
[1] 36
[1] 49
[1] 64
[1] 81
[1] 100
\end{verbatim}

\section{Writing functions}\label{writing-functions}

\begin{Shaded}
\begin{Highlighting}[]
\CommentTok{\# creating FUNCTIONS in r}

\NormalTok{pythagoras }\OtherTok{\textless{}{-}} \ControlFlowTok{function}\NormalTok{ (c1,c2)}
\NormalTok{\{}
\NormalTok{  h }\OtherTok{\textless{}{-}} \FunctionTok{sqrt}\NormalTok{ (c1}\SpecialCharTok{\^{}}\DecValTok{2} \SpecialCharTok{+}\NormalTok{ c2}\SpecialCharTok{\^{}}\DecValTok{2}\NormalTok{)}
  \FunctionTok{return}\NormalTok{ (h)}
\NormalTok{\}}

\FunctionTok{pythagoras}\NormalTok{(}\DecValTok{1}\NormalTok{,}\DecValTok{1}\NormalTok{)}
\end{Highlighting}
\end{Shaded}

\begin{verbatim}
[1] 1.414214
\end{verbatim}

\begin{Shaded}
\begin{Highlighting}[]
\FunctionTok{pythagoras}\NormalTok{(}\DecValTok{5}\NormalTok{,}\DecValTok{5}\NormalTok{)}
\end{Highlighting}
\end{Shaded}

\begin{verbatim}
[1] 7.071068
\end{verbatim}

\begin{Shaded}
\begin{Highlighting}[]
\FunctionTok{pythagoras}\NormalTok{(}\DecValTok{10}\NormalTok{,}\DecValTok{1}\NormalTok{)}
\end{Highlighting}
\end{Shaded}

\begin{verbatim}
[1] 10.04988
\end{verbatim}

\begin{Shaded}
\begin{Highlighting}[]
\CommentTok{\# regression in R}

\FunctionTok{help}\NormalTok{(lm)}
\NormalTok{x }\OtherTok{\textless{}{-}} \FunctionTok{c}\NormalTok{(}\DecValTok{1}\NormalTok{,}\DecValTok{2}\NormalTok{,}\DecValTok{3}\NormalTok{,}\DecValTok{4}\NormalTok{)}
\NormalTok{y }\OtherTok{\textless{}{-}} \FunctionTok{c}\NormalTok{(}\FloatTok{1.1}\NormalTok{,}\FloatTok{2.3}\NormalTok{,}\FloatTok{2.9}\NormalTok{,}\FloatTok{4.1}\NormalTok{)}
\FunctionTok{plot}\NormalTok{(x,y)}
\NormalTok{myreg}\OtherTok{\textless{}{-}}\FunctionTok{lm}\NormalTok{(y }\SpecialCharTok{\textasciitilde{}}\NormalTok{ x)}
\FunctionTok{summary}\NormalTok{(myreg)}
\end{Highlighting}
\end{Shaded}

\begin{verbatim}

Call:
lm(formula = y ~ x)

Residuals:
    1     2     3     4 
-0.06  0.18 -0.18  0.06 

Coefficients:
            Estimate Std. Error t value Pr(>|t|)   
(Intercept)  0.20000    0.23238   0.861  0.48012   
x            0.96000    0.08485  11.314  0.00772 **
---
Signif. codes:  0 '***' 0.001 '**' 0.01 '*' 0.05 '.' 0.1 ' ' 1

Residual standard error: 0.1897 on 2 degrees of freedom
Multiple R-squared:  0.9846,    Adjusted R-squared:  0.9769 
F-statistic:   128 on 1 and 2 DF,  p-value: 0.007722
\end{verbatim}

\begin{Shaded}
\begin{Highlighting}[]
\FunctionTok{abline}\NormalTok{(myreg)}
\end{Highlighting}
\end{Shaded}

\includegraphics{intro_to_R_files/figure-pdf/unnamed-chunk-3-1.pdf}

\section{Random numbers and statistical
distributions}\label{random-numbers-and-statistical-distributions}

\begin{Shaded}
\begin{Highlighting}[]
\NormalTok{random1d}\OtherTok{\textless{}{-}}\ControlFlowTok{function}\NormalTok{(tmax)}
\NormalTok{\{}
\NormalTok{x}\OtherTok{\textless{}{-}}\DecValTok{0}
\ControlFlowTok{for}\NormalTok{ (t }\ControlFlowTok{in} \DecValTok{1}\SpecialCharTok{:}\NormalTok{tmax)}
\NormalTok{\{}
\NormalTok{  r}\OtherTok{\textless{}{-}}\FunctionTok{runif}\NormalTok{(}\DecValTok{1}\NormalTok{)}
  \ControlFlowTok{if}\NormalTok{ (r}\SpecialCharTok{\textless{}}\DecValTok{1}\SpecialCharTok{/}\DecValTok{2}\NormalTok{)}
\NormalTok{    x[t}\SpecialCharTok{+}\DecValTok{1}\NormalTok{]}\OtherTok{\textless{}{-}}\NormalTok{x[t]}\SpecialCharTok{+}\DecValTok{1} \ControlFlowTok{else}
\NormalTok{      x[t}\SpecialCharTok{+}\DecValTok{1}\NormalTok{]}\OtherTok{\textless{}{-}}\NormalTok{x[t]}\SpecialCharTok{{-}}\DecValTok{1}
\NormalTok{\}}
\FunctionTok{return}\NormalTok{(x)}
\NormalTok{\}}

\FunctionTok{plot}\NormalTok{(}\FunctionTok{random1d}\NormalTok{(}\DecValTok{100}\NormalTok{))}
\end{Highlighting}
\end{Shaded}

\includegraphics{intro_to_R_files/figure-pdf/unnamed-chunk-4-1.pdf}

\begin{Shaded}
\begin{Highlighting}[]
\NormalTok{tmax}\OtherTok{\textless{}{-}}\DecValTok{10000}
\NormalTok{lastx}\OtherTok{\textless{}{-}}\DecValTok{0}
\ControlFlowTok{for}\NormalTok{ (i }\ControlFlowTok{in} \DecValTok{1}\SpecialCharTok{:}\DecValTok{1000}\NormalTok{)}
\NormalTok{\{}
\NormalTok{  x}\OtherTok{\textless{}{-}}\FunctionTok{random1d}\NormalTok{(tmax)}
\NormalTok{  lastx[i]}\OtherTok{\textless{}{-}}\NormalTok{x[tmax]}
\NormalTok{\}}

\FunctionTok{hist}\NormalTok{(lastx)}
\end{Highlighting}
\end{Shaded}

\includegraphics{intro_to_R_files/figure-pdf/unnamed-chunk-4-2.pdf}

\begin{Shaded}
\begin{Highlighting}[]
\FunctionTok{mean}\NormalTok{(lastx)}
\end{Highlighting}
\end{Shaded}

\begin{verbatim}
[1] 2.556
\end{verbatim}

\begin{Shaded}
\begin{Highlighting}[]
\NormalTok{d}\OtherTok{\textless{}{-}}\FunctionTok{sqrt}\NormalTok{(lastx}\SpecialCharTok{\^{}}\DecValTok{2}\NormalTok{)}
\FunctionTok{hist}\NormalTok{(d)}
\end{Highlighting}
\end{Shaded}

\includegraphics{intro_to_R_files/figure-pdf/unnamed-chunk-4-3.pdf}

\begin{Shaded}
\begin{Highlighting}[]
\FunctionTok{mean}\NormalTok{(d)}
\end{Highlighting}
\end{Shaded}

\begin{verbatim}
[1] 84.246
\end{verbatim}

\begin{Shaded}
\begin{Highlighting}[]
\FunctionTok{median}\NormalTok{(d)}
\end{Highlighting}
\end{Shaded}

\begin{verbatim}
[1] 73
\end{verbatim}

\begin{Shaded}
\begin{Highlighting}[]
\FunctionTok{max}\NormalTok{(d)}
\end{Highlighting}
\end{Shaded}

\begin{verbatim}
[1] 313
\end{verbatim}

\begin{Shaded}
\begin{Highlighting}[]
\FunctionTok{hist}\NormalTok{(d[d}\SpecialCharTok{\textless{}}\DecValTok{20}\NormalTok{],}\AttributeTok{breaks=}\FunctionTok{c}\NormalTok{(}\DecValTok{1}\SpecialCharTok{:}\DecValTok{20}\NormalTok{))}
\end{Highlighting}
\end{Shaded}

\includegraphics{intro_to_R_files/figure-pdf/unnamed-chunk-4-4.pdf}

\section{Spatial analysis}\label{spatial-analysis}

\subsection{Shapefiles and Raster (Isabel
Rosa)}\label{shapefiles-and-raster-isabel-rosa}

When you work with spatial data, essentially you use two types of data:

\begin{enumerate}
\def\labelenumi{\arabic{enumi})}
\tightlist
\item
  vector data (i.e., shapefiles): stores the geometric location and
  attribute information of geographic features. These can be represented
  by points, lines, or polygons (areas).
\item
  matricial data (i.e., raster): consists of a matrix of cells (or
  pixels) organized into rows and columns (or a grid) where each cell
  contains a value representing information. They can be categorical or
  continuous and have multiple bands.
\end{enumerate}

For more information on the tree cover datasets, please see:
\url{https://earthenginepartners.appspot.com/science-2013-global-forest/download_v1.2.html}

\begin{Shaded}
\begin{Highlighting}[]
\CommentTok{\# read in shapefile using rgdal}
\NormalTok{sc }\OtherTok{\textless{}{-}} \FunctionTok{readOGR}\NormalTok{(}\StringTok{"."}\NormalTok{, }\StringTok{"SantaCatarina"}\NormalTok{)}

\CommentTok{\# import municipalities and settlements shapefiles}
\NormalTok{sc\_mun }\OtherTok{\textless{}{-}} \FunctionTok{readOGR}\NormalTok{(}\StringTok{"."}\NormalTok{, }\StringTok{"SantaCatarina\_mun"}\NormalTok{)}
\NormalTok{br\_sett }\OtherTok{\textless{}{-}} \FunctionTok{readOGR}\NormalTok{(}\StringTok{"."}\NormalTok{, }\StringTok{"Brazil\_settlements"}\NormalTok{)}

\CommentTok{\# always good to check the contents of your dat}
\CommentTok{\#str(br\_sett)}

\CommentTok{\# visualize one of the variables}
\FunctionTok{spplot}\NormalTok{(sc\_mun, }\AttributeTok{z=}\StringTok{"Shape\_Area"}\NormalTok{, }\AttributeTok{main =} \StringTok{"Municipality Area (km2)"}\NormalTok{)}

\CommentTok{\# read in raster}
\NormalTok{tc}\OtherTok{\textless{}{-}}\FunctionTok{raster}\NormalTok{(}\StringTok{"tree\_cover.tif"}\NormalTok{)}

\CommentTok{\# import loss and gain rasters here}
\NormalTok{tl}\OtherTok{\textless{}{-}}\FunctionTok{raster}\NormalTok{(}\StringTok{"loss.tif"}\NormalTok{)}
\NormalTok{tg}\OtherTok{\textless{}{-}}\FunctionTok{raster}\NormalTok{(}\StringTok{"gain.tif"}\NormalTok{)}

\CommentTok{\# for multiple band rasters, you can choose to import just one or all bands}
\CommentTok{\#r2 \textless{}{-} raster("tree\_cover\_multi.tif", band=2)}

\CommentTok{\# note that the value 255, which is Hansen\textquotesingle{}s nodata value was not recognized as such}
\FunctionTok{NAvalue}\NormalTok{(tg) }\CommentTok{\# check first}
\FunctionTok{NAvalue}\NormalTok{(tc)}\OtherTok{\textless{}{-}}\DecValTok{255} \CommentTok{\#fix it by forcing 255 to be the nodata}
\FunctionTok{NAvalue}\NormalTok{(tl)}\OtherTok{\textless{}{-}}\DecValTok{255} \CommentTok{\#fix it by forcing 255 to be the nodata}
\FunctionTok{NAvalue}\NormalTok{(tg)}\OtherTok{\textless{}{-}}\DecValTok{255} \CommentTok{\#fix it by forcing 255 to be the nodata}

\CommentTok{\# visualize one of the rasters}
\FunctionTok{par}\NormalTok{(}\AttributeTok{mfrow=}\FunctionTok{c}\NormalTok{(}\DecValTok{1}\NormalTok{,}\DecValTok{3}\NormalTok{))}
\FunctionTok{plot}\NormalTok{(tc, }\AttributeTok{main =} \StringTok{"Tree Cover (\%)"}\NormalTok{)}
\FunctionTok{plot}\NormalTok{(tl, }\AttributeTok{main =} \StringTok{"Tree Cover Loss (binary)"}\NormalTok{)}
\FunctionTok{plot}\NormalTok{(tg, }\AttributeTok{main =} \StringTok{"Tree Cover Gain (binary)"}\NormalTok{)}
\end{Highlighting}
\end{Shaded}

\subsection{Reference systems}\label{reference-systems}

Coordinate systems are essential to understand when working with spatial
data. Some reading material on this can be found here: Essentially, if
one wants to know which position of the Earth we refer to, coordinates
of geospatial data require a reference system:

\begin{enumerate}
\def\labelenumi{\arabic{enumi})}
\tightlist
\item
  geodesic/geographic coordinates need an order (lat/long), a unit
  (e.g., degrees) and a datum (a reference ellipsoid: e.g.~WGS84)
\item
  cartesian/projected coordinates (e.g.~UTM, web Mercator) need also
  measurement units (e.g., meters), and some way of encoding how they
  relate to geodesic coordinates, in which datum (this is handled by the
  GIS system)
\end{enumerate}

\subsection{Operations with
Shapefiles}\label{operations-with-shapefiles}

Clip: in R you can clip using the command ``intersect'', so that
intersect(feature to be clipped, clip feature) Select: you can use a
boolean selection to subset the features of your shapefile, for instance
if you just want to look at settlements with a mininum number of
habitants, so that Population \textgreater{} median(Population) There
are several options, have a look at this great tutorial:
\url{http://www.rspatial.org/spatial/rst/7-vectmanip.html}

\begin{Shaded}
\begin{Highlighting}[]
\CommentTok{\# Clip the settlement features using the Santa Catarina shapefile}
\NormalTok{sc\_sett}\OtherTok{\textless{}{-}}\FunctionTok{intersect}\NormalTok{(br\_sett, sc)}

\CommentTok{\#sc\_sett$med \textless{}{-} sc\_sett$population \textgreater{} median(sc\_sett$population) \# oops! annoyingly our population values have been stored as factors}

\CommentTok{\# convert to original numerical values}
\NormalTok{sc\_sett}\SpecialCharTok{$}\NormalTok{population}\OtherTok{\textless{}{-}}\FunctionTok{as.numeric}\NormalTok{(}\FunctionTok{as.vector}\NormalTok{(sc\_sett}\SpecialCharTok{$}\NormalTok{population))}
\CommentTok{\# careful! applying as.numeric alone it will not work!!}

\CommentTok{\# visualize results}
\FunctionTok{plot}\NormalTok{(sc\_sett, }\AttributeTok{main =} \StringTok{"Settlements in Santa Catarina"}\NormalTok{)}
\FunctionTok{spplot}\NormalTok{(sc\_sett, }\AttributeTok{z=}\StringTok{"population"}\NormalTok{, }\AttributeTok{main =} \StringTok{"Population per Settlement (people)"}\NormalTok{)}

\CommentTok{\# select settlements larger then the median value}
\NormalTok{sc\_sett}\SpecialCharTok{$}\NormalTok{med }\OtherTok{\textless{}{-}}\NormalTok{ sc\_sett}\SpecialCharTok{$}\NormalTok{population }\SpecialCharTok{\textgreater{}} \FunctionTok{median}\NormalTok{(sc\_sett}\SpecialCharTok{$}\NormalTok{population)}
\NormalTok{sc\_largesett }\OtherTok{\textless{}{-}}\NormalTok{ sc\_sett[sc\_sett}\SpecialCharTok{$}\NormalTok{med }\SpecialCharTok{==} \DecValTok{1}\NormalTok{, ]}

\CommentTok{\# visualize results}
\FunctionTok{par}\NormalTok{(}\AttributeTok{mfrow=}\FunctionTok{c}\NormalTok{(}\DecValTok{1}\NormalTok{,}\DecValTok{2}\NormalTok{))}
\FunctionTok{plot}\NormalTok{(sc\_sett, }\AttributeTok{main =} \StringTok{"All Settlements"}\NormalTok{)}
\FunctionTok{plot}\NormalTok{(sc\_largesett, }\AttributeTok{main =} \StringTok{"Largest Settlements"}\NormalTok{)}
\end{Highlighting}
\end{Shaded}

\subsection{Operations with Rasters}\label{operations-with-rasters}

There are many operations you can do with rasters, and these are more
frequently used in spatial analyses than shapefiles. Here I will just
illustrate a couple of simple operations: - Global/Raster statistics -
obtain a value that summarizes the whole raster layer - Cell statistics
(pixel-by-pixel operation): obtains a value per pixel - Focal statistics
(operation that takes into account neighborhood of central cell) -
results in raster of same of different size - Zonal statistics -
calculates summary statistics of a give raster (e.g., elevation) based
on pre-defined zones (e.g., admnistrative boundaries, biomes). Outputs a
table with the values per zone. For more great examples, have a look
here: \url{http://www.rspatial.org/spatial/rst/8-rastermanip.html}

\begin{Shaded}
\begin{Highlighting}[]
\CommentTok{\# sum the loss and gain rasters to know where there was simultaneous loss and gain in Santa Catarina}
\NormalTok{tclg}\OtherTok{\textless{}{-}}\NormalTok{tl}\SpecialCharTok{+}\NormalTok{tg }
\FunctionTok{par}\NormalTok{(}\AttributeTok{mfrow=}\FunctionTok{c}\NormalTok{(}\DecValTok{1}\NormalTok{,}\DecValTok{3}\NormalTok{))}
\FunctionTok{plot}\NormalTok{(tl, }\AttributeTok{main =} \StringTok{"Forest Loss"}\NormalTok{)}
\FunctionTok{plot}\NormalTok{(tg, }\AttributeTok{main =} \StringTok{"Forest Gain"}\NormalTok{)}
\FunctionTok{plot}\NormalTok{(tclg, }\AttributeTok{main =} \StringTok{"Forest Loss and Gain"}\NormalTok{)}

\CommentTok{\# you can also try to create three new rasters and work with them}
\CommentTok{\# create a new raster}
\NormalTok{r }\OtherTok{\textless{}{-}} \FunctionTok{raster}\NormalTok{(}\AttributeTok{ncol=}\DecValTok{10}\NormalTok{, }\AttributeTok{nrow=}\DecValTok{10}\NormalTok{, }\AttributeTok{xmx=}\SpecialCharTok{{-}}\DecValTok{80}\NormalTok{, }\AttributeTok{xmn=}\SpecialCharTok{{-}}\DecValTok{150}\NormalTok{, }\AttributeTok{ymn=}\DecValTok{20}\NormalTok{, }\AttributeTok{ymx=}\DecValTok{60}\NormalTok{)}
\FunctionTok{values}\NormalTok{(r) }\OtherTok{\textless{}{-}} \FunctionTok{runif}\NormalTok{(}\FunctionTok{ncell}\NormalTok{(r)) }\CommentTok{\# assign random values}
\CommentTok{\#plot(r)}

\CommentTok{\# create two more rasters based on the first one}
\NormalTok{r2 }\OtherTok{\textless{}{-}}\NormalTok{ r }\SpecialCharTok{*}\NormalTok{ r}
\NormalTok{r3  }\OtherTok{\textless{}{-}} \FunctionTok{sqrt}\NormalTok{(r)}

\CommentTok{\# either stack or brick them}
\NormalTok{s }\OtherTok{\textless{}{-}} \FunctionTok{stack}\NormalTok{(r, r2, r3)}
\CommentTok{\#b \textless{}{-} brick(s)}

\CommentTok{\# Raster statistics {-} calculate several statistics per raster layer (i.e., sum, mean, median)}
\FunctionTok{cellStats}\NormalTok{(s, }\StringTok{"sum"}\NormalTok{) }\CommentTok{\# outputs a value per raster}

\CommentTok{\# Cell statistics {-} calculate several statistics per pixel  (i.e., sum, mean, median)}
\FunctionTok{par}\NormalTok{(}\AttributeTok{mfrow=}\FunctionTok{c}\NormalTok{(}\DecValTok{2}\NormalTok{,}\DecValTok{2}\NormalTok{))}
\FunctionTok{plot}\NormalTok{(r, }\AttributeTok{main =}\StringTok{"Random 1"}\NormalTok{)}
\FunctionTok{plot}\NormalTok{(r2, }\AttributeTok{main =}\StringTok{"Random 2"}\NormalTok{)}
\FunctionTok{plot}\NormalTok{(r3, }\AttributeTok{main =}\StringTok{"Random 3"}\NormalTok{)}
\FunctionTok{plot}\NormalTok{(}\FunctionTok{overlay}\NormalTok{(s, }\AttributeTok{fun=}\StringTok{"mean"}\NormalTok{), }\AttributeTok{main=}\StringTok{"Average Values"}\NormalTok{) }\CommentTok{\# outputs a new raster}

\CommentTok{\# Focal statistics {-} calculate mean accounting for the neighborhood values, compare with previous outcome }
\NormalTok{f1 }\OtherTok{\textless{}{-}} \FunctionTok{focal}\NormalTok{(tc, }\AttributeTok{w=}\FunctionTok{matrix}\NormalTok{(}\DecValTok{1}\NormalTok{,}\AttributeTok{nrow=}\DecValTok{5}\NormalTok{,}\AttributeTok{ncol=}\DecValTok{5}\NormalTok{) , }\AttributeTok{fun=}\NormalTok{mean)}
\FunctionTok{plot}\NormalTok{(f1, }\AttributeTok{main =} \StringTok{"Average forest cover 5x5"}\NormalTok{)}
\CommentTok{\# sum the loss, vary window size}
\NormalTok{f2 }\OtherTok{\textless{}{-}} \FunctionTok{focal}\NormalTok{(tl, }\AttributeTok{w=}\FunctionTok{matrix}\NormalTok{(}\DecValTok{1}\NormalTok{,}\AttributeTok{nrow=}\DecValTok{5}\NormalTok{,}\AttributeTok{ncol=}\DecValTok{5}\NormalTok{) , }\AttributeTok{fun=}\NormalTok{sum)}
\FunctionTok{plot}\NormalTok{(f2, }\AttributeTok{main =} \StringTok{"Total forest loss 5x5"}\NormalTok{)}
\CommentTok{\# sum the gain, vary window size}
\NormalTok{f3 }\OtherTok{\textless{}{-}} \FunctionTok{focal}\NormalTok{(tg, }\AttributeTok{w=}\FunctionTok{matrix}\NormalTok{(}\DecValTok{1}\NormalTok{,}\AttributeTok{nrow=}\DecValTok{5}\NormalTok{,}\AttributeTok{ncol=}\DecValTok{5}\NormalTok{) , }\AttributeTok{fun=}\NormalTok{sum)}
\FunctionTok{plot}\NormalTok{(f3, }\AttributeTok{main =} \StringTok{"Total forest gain 5x5"}\NormalTok{)}

\CommentTok{\# plot 4 maps with different window sizes}
\FunctionTok{par}\NormalTok{(}\AttributeTok{mfrow=}\FunctionTok{c}\NormalTok{(}\DecValTok{2}\NormalTok{,}\DecValTok{2}\NormalTok{))}
\ControlFlowTok{for}\NormalTok{(i }\ControlFlowTok{in} \FunctionTok{c}\NormalTok{(}\DecValTok{5}\NormalTok{,}\DecValTok{15}\NormalTok{,}\DecValTok{25}\NormalTok{,}\DecValTok{55}\NormalTok{))\{}
\NormalTok{  f\_w }\OtherTok{\textless{}{-}} \FunctionTok{focal}\NormalTok{(tc, }\AttributeTok{w=}\FunctionTok{matrix}\NormalTok{(}\DecValTok{1}\NormalTok{,}\AttributeTok{nrow=}\NormalTok{i,}\AttributeTok{ncol=}\NormalTok{i) , }\AttributeTok{fun=}\NormalTok{sum)}
  \FunctionTok{plot}\NormalTok{(f\_w, }\AttributeTok{main =} \FunctionTok{paste0}\NormalTok{(}\StringTok{"Window size: "}\NormalTok{, i))}
\NormalTok{\}}

\CommentTok{\# Zonal Statistics {-} using two rasters}
\NormalTok{sc\_tc\_mean\_loss }\OtherTok{\textless{}{-}} \FunctionTok{zonal}\NormalTok{(tc, tl, }\AttributeTok{fun=}\NormalTok{mean) }\CommentTok{\#average tree cover in loss areas}
\NormalTok{sc\_tc\_mean\_gain }\OtherTok{\textless{}{-}} \FunctionTok{zonal}\NormalTok{(tc, tg, }\AttributeTok{fun=}\NormalTok{mean) }\CommentTok{\#average tree cover in gain areas}

\CommentTok{\# average tree cover loss}
\NormalTok{sc\_tc\_mean\_loss}

\CommentTok{\# average tree cover gain}
\NormalTok{sc\_tc\_mean\_gain}
\end{Highlighting}
\end{Shaded}

\subsection{Operations with both Rasters and
Shapefiles}\label{operations-with-both-rasters-and-shapefiles}

Here I'll show a couple of examples of operation that use feature data
as inputs and output rasters: Distance to features - calculates the
euclidean distance from each cell/pixel to the closest feature (e.g.,
roads, settlements). Outputs a raster file with these distances.
Interpolation: a world in itself! Very vey short example provided here
(based on a single method, IDW), please see more here:
\url{http://www.rspatial.org/analysis/rst/4-interpolation.html} To
better understand interpolation I advise you to read first about spatial
autocorrelation:
\url{http://www.rspatial.org/analysis/rst/3-spauto.html}

To use interpolation metrics you need to load another packaged called
gstat Inverse distance weighted (IDW) - See more also here:
\url{http://desktop.arcgis.com/en/arcmap/10.3/tools/3d-analyst-toolbox/how-idw-works.htm}

\begin{Shaded}
\begin{Highlighting}[]
\CommentTok{\# create an empty raster (little trick using existing raster)}
\NormalTok{dist\_sett}\OtherTok{\textless{}{-}}\NormalTok{tc}\SpecialCharTok{*}\DecValTok{0}
\CommentTok{\# or you can create an empty one like before}
\CommentTok{\# dist\_sett \textless{}{-} raster(ncol=ncol(tc), nrow=nrow(tc), xmx=extent(tc)@xmax, xmn=extent(tc)@xmin, ymn=extent(tc)@ymin, ymx=extent(tc)@ymax)}

\CommentTok{\# Distance to points}
\NormalTok{dist\_sett }\OtherTok{\textless{}{-}} \FunctionTok{distanceFromPoints}\NormalTok{(dist\_sett, sc\_sett)}

\CommentTok{\# you can then mask the outside area of Santa Catarina}
\NormalTok{dist\_sett }\OtherTok{\textless{}{-}} \FunctionTok{mask}\NormalTok{(dist\_sett, tc)}

\CommentTok{\# plot results}
\FunctionTok{plot}\NormalTok{(dist\_sett, }\AttributeTok{main =} \StringTok{"Distance to settlements (m)"}\NormalTok{)}

\CommentTok{\# load gstat}
\FunctionTok{library}\NormalTok{(gstat)}
\NormalTok{idw\_sett}\OtherTok{\textless{}{-}}\NormalTok{tc}\SpecialCharTok{*}\DecValTok{0}

\CommentTok{\# compute the model, see reference for more detail}
\NormalTok{gs }\OtherTok{\textless{}{-}} \FunctionTok{gstat}\NormalTok{(}\AttributeTok{formula=}\NormalTok{population}\SpecialCharTok{\textasciitilde{}}\DecValTok{1}\NormalTok{, }\AttributeTok{locations=}\NormalTok{sc\_sett, }\AttributeTok{nmax=}\DecValTok{5}\NormalTok{, }\AttributeTok{set=}\FunctionTok{list}\NormalTok{(}\AttributeTok{idp =} \DecValTok{2}\NormalTok{))}
\NormalTok{idw\_out }\OtherTok{\textless{}{-}} \FunctionTok{interpolate}\NormalTok{(idw\_sett, gs)}

\DocumentationTok{\#\# [inverse distance weighted interpolation]}
\NormalTok{sc\_pop }\OtherTok{\textless{}{-}} \FunctionTok{mask}\NormalTok{(idw\_out, tc)}
\FunctionTok{plot}\NormalTok{(sc\_pop, }\AttributeTok{main =} \StringTok{"Santa Catarina Population"}\NormalTok{)}
\end{Highlighting}
\end{Shaded}

\subsection{Export Shapefiles and
Rasters}\label{export-shapefiles-and-rasters}

It's very easy to export both shapefiles and rasters from R to be
visualized in QGIS or ArcMap.

\begin{Shaded}
\begin{Highlighting}[]
\CommentTok{\# Save feature layers (point, polygon, polyline) to shapefile }
\FunctionTok{writeOGR}\NormalTok{(sc\_largesett, }\AttributeTok{dsn=}\StringTok{"."}\NormalTok{, }\AttributeTok{layer=}\StringTok{"SC\_largeSett"}\NormalTok{, }\AttributeTok{driver=}\StringTok{"ESRI Shapefile"}\NormalTok{ )}

\CommentTok{\# or }
\CommentTok{\#shapefile(sc\_largesett, "SC\_largeSett.shp", overwrite=TRUE) }

\CommentTok{\#Exporting raster}
\FunctionTok{writeRaster}\NormalTok{(sc\_pop, }\AttributeTok{filename=}\StringTok{"SC\_popmap"}\NormalTok{, }\AttributeTok{format=}\StringTok{"GTiff"}\NormalTok{ )}
\end{Highlighting}
\end{Shaded}

\section{Working with biodiversity data: GBIF, EBV Portal (Corey
Callaghan, Luise Quoss)First we load the library
rgbif.}\label{working-with-biodiversity-data-gbif-ebv-portal-corey-callaghan-luise-quossfirst-we-load-the-library-rgbif.}

\begin{Shaded}
\begin{Highlighting}[]
\FunctionTok{library}\NormalTok{(rgbif)}
\FunctionTok{library}\NormalTok{(tidyverse)}
\end{Highlighting}
\end{Shaded}

Now we will download observations of a species. Let's download
observations of the common toad ``Bufo bufo''.

\begin{Shaded}
\begin{Highlighting}[]
\NormalTok{matbufobufo}\OtherTok{\textless{}{-}}\FunctionTok{occ\_search}\NormalTok{(}\AttributeTok{scientificName=}\StringTok{"Bufo bufo"}\NormalTok{, }\AttributeTok{limit=}\DecValTok{500}\NormalTok{, }\AttributeTok{hasCoordinate =} \ConstantTok{TRUE}\NormalTok{, }\AttributeTok{hasGeospatialIssue =} \ConstantTok{FALSE}\NormalTok{)}
\end{Highlighting}
\end{Shaded}

Let's examine the object \emph{matbufobufo}

\begin{Shaded}
\begin{Highlighting}[]
\FunctionTok{class}\NormalTok{(matbufobufo)}
\NormalTok{matbufobufo}
\end{Highlighting}
\end{Shaded}

Let's download data about octupusses. They are in the order
``Octopoda''. First we need to find the GBIF search key for Octopoda.

\begin{Shaded}
\begin{Highlighting}[]
\NormalTok{a}\OtherTok{\textless{}{-}}\FunctionTok{name\_suggest}\NormalTok{(}\AttributeTok{q=}\StringTok{"Octopoda"}\NormalTok{,}\AttributeTok{rank=}\StringTok{"Order"}\NormalTok{)}
\NormalTok{key}\OtherTok{\textless{}{-}}\NormalTok{a}\SpecialCharTok{$}\NormalTok{data}\SpecialCharTok{$}\NormalTok{key}
\end{Highlighting}
\end{Shaded}

\begin{Shaded}
\begin{Highlighting}[]
\NormalTok{octopusses}\OtherTok{\textless{}{-}}\FunctionTok{occ\_search}\NormalTok{(}\AttributeTok{orderKey=}\NormalTok{key,}\AttributeTok{limit=}\DecValTok{2000}\NormalTok{, }\AttributeTok{hasCoordinate =} \ConstantTok{TRUE}\NormalTok{, }\AttributeTok{hasGeospatialIssue =} \ConstantTok{FALSE}\NormalTok{)}
\end{Highlighting}
\end{Shaded}

Show the result

\begin{Shaded}
\begin{Highlighting}[]
\NormalTok{octmat}\OtherTok{\textless{}{-}}\NormalTok{octopusses}\SpecialCharTok{$}\NormalTok{data}
\FunctionTok{head}\NormalTok{(octmat)}
\end{Highlighting}
\end{Shaded}

Count the number of observations per species using tidyverse and pipes

\begin{Shaded}
\begin{Highlighting}[]
\CommentTok{\#class(octmat)}
\NormalTok{octmat }\SpecialCharTok{\%\textgreater{}\%} 
  \FunctionTok{group\_by}\NormalTok{(scientificName) }\SpecialCharTok{\%\textgreater{}\%} 
  \FunctionTok{summarise}\NormalTok{(}\AttributeTok{sample\_size=}\FunctionTok{n}\NormalTok{()) }\SpecialCharTok{\%\textgreater{}\%}
  \FunctionTok{arrange}\NormalTok{(}\FunctionTok{desc}\NormalTok{(sample\_size)) }\SpecialCharTok{\%\textgreater{}\%} 
  \FunctionTok{mutate}\NormalTok{(}\AttributeTok{sample\_size\_log=}\FunctionTok{log}\NormalTok{(sample\_size,}\DecValTok{2}\NormalTok{)) }\SpecialCharTok{\%\textgreater{}\%} 
  \FunctionTok{ggplot}\NormalTok{(}\FunctionTok{aes}\NormalTok{(}\AttributeTok{x =}\NormalTok{ sample\_size\_log)) }\SpecialCharTok{+} \FunctionTok{geom\_histogram}\NormalTok{() }
\end{Highlighting}
\end{Shaded}

Plot the records on an interactive map. First load the leaflet package.

\begin{Shaded}
\begin{Highlighting}[]
\FunctionTok{library}\NormalTok{(leaflet)}
\FunctionTok{leaflet}\NormalTok{(}\AttributeTok{data=}\NormalTok{octmat) }\SpecialCharTok{\%\textgreater{}\%} \FunctionTok{addTiles}\NormalTok{() }\SpecialCharTok{\%\textgreater{}\%}
  \FunctionTok{addCircleMarkers}\NormalTok{(}\AttributeTok{lat=} \SpecialCharTok{\textasciitilde{}}\NormalTok{decimalLatitude, }\AttributeTok{lng =} \SpecialCharTok{\textasciitilde{}}\NormalTok{decimalLongitude,}\AttributeTok{popup=}\SpecialCharTok{\textasciitilde{}}\NormalTok{scientificName)}
\end{Highlighting}
\end{Shaded}

\subsection{Version 2 (Isabel Rosa)}\label{version-2-isabel-rosa}

Here are the packages we'll need.

\begin{Shaded}
\begin{Highlighting}[]
\FunctionTok{library}\NormalTok{(rgbif)}
\FunctionTok{library}\NormalTok{(tidyverse)}
\FunctionTok{library}\NormalTok{(raster)}
\FunctionTok{library}\NormalTok{(maps)}
\FunctionTok{library}\NormalTok{(leaflet)}
\FunctionTok{library}\NormalTok{(sdmpredictors)}
\end{Highlighting}
\end{Shaded}

First let's pick an example species to download data for. We will only
download 500 observations to keep it simple for now. If you were doing
this for real, you would download all data for that species (see notes
at the end). I will choose the European Robin:
\url{https://en.wikipedia.org/wiki/European_robin.}

\begin{Shaded}
\begin{Highlighting}[]
\NormalTok{species }\OtherTok{\textless{}{-}} \FunctionTok{occ\_search}\NormalTok{(}\AttributeTok{scientificName=}\StringTok{"Erithacus rubecula"}\NormalTok{, }\AttributeTok{limit=}\DecValTok{500}\NormalTok{, }\AttributeTok{hasCoordinate =} \ConstantTok{TRUE}\NormalTok{, }\AttributeTok{hasGeospatialIssue=}\ConstantTok{FALSE}\NormalTok{)}
\end{Highlighting}
\end{Shaded}

What does this object look like?

\begin{Shaded}
\begin{Highlighting}[]
\FunctionTok{class}\NormalTok{(species)}

\NormalTok{species}
\end{Highlighting}
\end{Shaded}

It is a special object of class \texttt{gbif} which allows for the
metadata and the actual data to all be included, as well as taxonomic
hierarchy data, and media metadata. We won't worry too much about the
details of this object now. But we do want to get a dataframe we can
work with! To do this, we have one extra step.

\begin{Shaded}
\begin{Highlighting}[]
\NormalTok{sp\_dat }\OtherTok{\textless{}{-}}\NormalTok{ species}\SpecialCharTok{$}\NormalTok{data}

\FunctionTok{class}\NormalTok{(sp\_dat)}

\FunctionTok{head}\NormalTok{(sp\_dat)}
\end{Highlighting}
\end{Shaded}

So this was just for one species. Lets broaden this out a little bit.
What if we were interested in many species of a given order/class? Here,
we will choose an entire order to download. I will choose owls!
\url{https://en.wikipedia.org/wiki/Owl.} First, we need to find the
`key' that gbif uses for that order and then pass it to our GBIF
download function. Again, we are only getting a small number of records
for illustration purposes.

\begin{Shaded}
\begin{Highlighting}[]
\NormalTok{a }\OtherTok{\textless{}{-}} \FunctionTok{name\_suggest}\NormalTok{(}\AttributeTok{q=}\StringTok{\textquotesingle{}Strigiformes\textquotesingle{}}\NormalTok{)}

\NormalTok{key }\OtherTok{\textless{}{-}}\NormalTok{ a}\SpecialCharTok{$}\NormalTok{data}\SpecialCharTok{$}\NormalTok{key}

\NormalTok{order }\OtherTok{\textless{}{-}} \FunctionTok{occ\_search}\NormalTok{(}\AttributeTok{orderKey=}\NormalTok{key, }\AttributeTok{limit=}\DecValTok{1000}\NormalTok{, }\AttributeTok{hasCoordinate =} \ConstantTok{TRUE}\NormalTok{, }\AttributeTok{hasGeospatialIssue=}\ConstantTok{FALSE}\NormalTok{) }
\end{Highlighting}
\end{Shaded}

What kind of object is `order'? As with species, we need to turn it into
a dataframe to work with.

\begin{Shaded}
\begin{Highlighting}[]
\NormalTok{order\_dat }\OtherTok{\textless{}{-}}\NormalTok{ order}\SpecialCharTok{$}\NormalTok{data}

\FunctionTok{class}\NormalTok{(order\_dat)}

\FunctionTok{head}\NormalTok{(order\_dat)}
\end{Highlighting}
\end{Shaded}

Count the number of observations by species

\begin{Shaded}
\begin{Highlighting}[]
\NormalTok{order\_dat }\SpecialCharTok{\%\textgreater{}\%}
  \FunctionTok{group\_by}\NormalTok{(scientificName) }\SpecialCharTok{\%\textgreater{}\%}
  \FunctionTok{summarize}\NormalTok{(}\AttributeTok{sample\_size=}\FunctionTok{n}\NormalTok{()) }\SpecialCharTok{\%\textgreater{}\%}
  \FunctionTok{arrange}\NormalTok{(}\FunctionTok{desc}\NormalTok{(sample\_size))}
\end{Highlighting}
\end{Shaded}

Plot the records on an interactive map. First for our chosen species.

\begin{Shaded}
\begin{Highlighting}[]
\FunctionTok{leaflet}\NormalTok{(}\AttributeTok{data =}\NormalTok{ sp\_dat) }\SpecialCharTok{\%\textgreater{}\%}
    \FunctionTok{addTiles}\NormalTok{() }\SpecialCharTok{\%\textgreater{}\%}
    \FunctionTok{addCircleMarkers}\NormalTok{(}\AttributeTok{lat =} \SpecialCharTok{\textasciitilde{}}\NormalTok{decimalLatitude, }\AttributeTok{lng =} \SpecialCharTok{\textasciitilde{}}\NormalTok{decimalLongitude,}\AttributeTok{popup =} \SpecialCharTok{\textasciitilde{}}\NormalTok{scientificName)}
\end{Highlighting}
\end{Shaded}

Then for the order we chose.

\begin{Shaded}
\begin{Highlighting}[]
\FunctionTok{leaflet}\NormalTok{(}\AttributeTok{data =}\NormalTok{ order\_dat) }\SpecialCharTok{\%\textgreater{}\%}
    \FunctionTok{addTiles}\NormalTok{() }\SpecialCharTok{\%\textgreater{}\%}
    \FunctionTok{addCircleMarkers}\NormalTok{(}\AttributeTok{lat =} \SpecialCharTok{\textasciitilde{}}\NormalTok{decimalLatitude, }\AttributeTok{lng =} \SpecialCharTok{\textasciitilde{}}\NormalTok{decimalLongitude, }\AttributeTok{popup =} \SpecialCharTok{\textasciitilde{}}\NormalTok{scientificName)}
\end{Highlighting}
\end{Shaded}

\section{Climate data}\label{climate-data}

Let's play with some global climate data and overlay that with our GBIF
observations.

\begin{Shaded}
\begin{Highlighting}[]
\NormalTok{mean\_temp\_map }\OtherTok{\textless{}{-}} \FunctionTok{getData}\NormalTok{(}\AttributeTok{name=}\StringTok{"worldclim"}\NormalTok{, }\AttributeTok{res=}\DecValTok{10}\NormalTok{, }\AttributeTok{var=}\StringTok{"tmean"}\NormalTok{)}
\FunctionTok{plot}\NormalTok{(mean\_temp\_map)}
\end{Highlighting}
\end{Shaded}

Each month has separate values for each cell. To combine to a yearly
value, we just take the mean.

\begin{Shaded}
\begin{Highlighting}[]
\NormalTok{annual\_mean\_temp }\OtherTok{\textless{}{-}} \FunctionTok{mean}\NormalTok{(mean\_temp\_map)}\SpecialCharTok{/}\DecValTok{10} \CommentTok{\#data comes as degrees * 10}
\end{Highlighting}
\end{Shaded}

Now plot this.

\begin{Shaded}
\begin{Highlighting}[]
\FunctionTok{plot}\NormalTok{(annual\_mean\_temp)}
\end{Highlighting}
\end{Shaded}

To get out the values for the organism of interest, we use
\texttt{extract}.

\begin{Shaded}
\begin{Highlighting}[]
\NormalTok{org\_temp }\OtherTok{\textless{}{-}} \FunctionTok{extract}\NormalTok{(annual\_mean\_temp, }\FunctionTok{cbind}\NormalTok{(}\AttributeTok{x=}\NormalTok{sp\_dat}\SpecialCharTok{$}\NormalTok{decimalLongitude, }\AttributeTok{y=}\NormalTok{sp\_dat}\SpecialCharTok{$}\NormalTok{decimalLatitude))}
\end{Highlighting}
\end{Shaded}

Now we will visualize how the global distribution of temperature values
compares with the species' distribution of temperature values. This
shows the distribution of temps where robins are found versus the global
distribution of temps.

\begin{Shaded}
\begin{Highlighting}[]
\NormalTok{temp }\OtherTok{\textless{}{-}} \FunctionTok{tibble}\NormalTok{(}\AttributeTok{mean\_temp=}\FunctionTok{getValues}\NormalTok{(annual\_mean\_temp)) }\SpecialCharTok{\%\textgreater{}\%}
  \FunctionTok{filter}\NormalTok{(}\SpecialCharTok{!}\FunctionTok{is.na}\NormalTok{(mean\_temp))}

\NormalTok{temp\_org }\OtherTok{\textless{}{-}} \FunctionTok{tibble}\NormalTok{(}\AttributeTok{organism\_temp=}\NormalTok{org\_temp) }\SpecialCharTok{\%\textgreater{}\%}
  \FunctionTok{filter}\NormalTok{(}\SpecialCharTok{!}\FunctionTok{is.na}\NormalTok{(organism\_temp))}

\FunctionTok{ggplot}\NormalTok{(temp, }\FunctionTok{aes}\NormalTok{(}\AttributeTok{x=}\NormalTok{mean\_temp))}\SpecialCharTok{+}
  \FunctionTok{geom\_density}\NormalTok{(}\AttributeTok{fill=}\StringTok{"blue"}\NormalTok{)}\SpecialCharTok{+}
  \FunctionTok{geom\_density}\NormalTok{(}\AttributeTok{data=}\NormalTok{temp\_org, }\FunctionTok{aes}\NormalTok{(}\AttributeTok{x=}\NormalTok{organism\_temp), }\AttributeTok{fill=}\StringTok{"red"}\NormalTok{)}\SpecialCharTok{+}
  \FunctionTok{theme\_bw}\NormalTok{()}
\end{Highlighting}
\end{Shaded}

\section{Intro to apply, pipes, ggplot2,
tidyverse.}\label{intro-to-apply-pipes-ggplot2-tidyverse.}

\subsection{Introduction to gpplot}\label{introduction-to-gpplot}

\begin{Shaded}
\begin{Highlighting}[]
\NormalTok{cars}
\FunctionTok{library}\NormalTok{(ggplot2)}

\FunctionTok{ggplot}\NormalTok{(}\AttributeTok{data=}\NormalTok{cars, }\AttributeTok{mapping=}\FunctionTok{aes}\NormalTok{(}\AttributeTok{x=}\NormalTok{speed,}\AttributeTok{y=}\NormalTok{dist)) }\SpecialCharTok{+} \FunctionTok{geom\_point}\NormalTok{(}\AttributeTok{colour=}\StringTok{"red"}\NormalTok{)}
\end{Highlighting}
\end{Shaded}

\begin{Shaded}
\begin{Highlighting}[]
\FunctionTok{plot}\NormalTok{(cars}\SpecialCharTok{$}\NormalTok{speed,cars}\SpecialCharTok{$}\NormalTok{dist)}
\end{Highlighting}
\end{Shaded}

\begin{Shaded}
\begin{Highlighting}[]
\CommentTok{\#Introduction to R {-} 9}


\CommentTok{\# a recursive function that calculates a factorial}
\NormalTok{myfun }\OtherTok{\textless{}{-}} \ControlFlowTok{function}\NormalTok{(x)}
\NormalTok{\{}
  \ControlFlowTok{if}\NormalTok{ (x}\SpecialCharTok{==}\DecValTok{1}\NormalTok{)}
    \FunctionTok{return}\NormalTok{ (}\DecValTok{1}\NormalTok{)}
  \ControlFlowTok{else} \FunctionTok{return}\NormalTok{(x}\SpecialCharTok{*}\FunctionTok{myfun}\NormalTok{(x}\DecValTok{{-}1}\NormalTok{))}
\NormalTok{\}}

\FunctionTok{myfun}\NormalTok{(}\DecValTok{1}\SpecialCharTok{:}\DecValTok{10}\NormalTok{) }\CommentTok{\# does not work}

\CommentTok{\#option1 {-} with a for loop}
\NormalTok{start\_time }\OtherTok{\textless{}{-}} \FunctionTok{Sys.time}\NormalTok{()}
\NormalTok{y}\OtherTok{\textless{}{-}}\DecValTok{0}
\ControlFlowTok{for}\NormalTok{ (i }\ControlFlowTok{in} \DecValTok{1}\SpecialCharTok{:}\DecValTok{100}\NormalTok{)}
\NormalTok{  y[i]}\OtherTok{\textless{}{-}}\FunctionTok{myfun}\NormalTok{(i)}
\NormalTok{end\_time }\OtherTok{\textless{}{-}} \FunctionTok{Sys.time}\NormalTok{()}
\NormalTok{end\_time}\SpecialCharTok{{-}}\NormalTok{start\_time}
\NormalTok{y}
\FunctionTok{plot}\NormalTok{(y,}\AttributeTok{type=}\StringTok{"l"}\NormalTok{)}

\CommentTok{\#option 2 {-} with apply}
\NormalTok{start\_time }\OtherTok{\textless{}{-}} \FunctionTok{Sys.time}\NormalTok{()}
\NormalTok{y}\OtherTok{\textless{}{-}}\FunctionTok{sapply}\NormalTok{(}\DecValTok{1}\SpecialCharTok{:}\DecValTok{100}\NormalTok{,myfun)}
\NormalTok{end\_time }\OtherTok{\textless{}{-}} \FunctionTok{Sys.time}\NormalTok{()}
\NormalTok{end\_time}\SpecialCharTok{{-}}\NormalTok{start\_time}
\NormalTok{y}

\CommentTok{\# selecting a subset from a matrix and applying a function to a column of that subset}

\FunctionTok{setwd}\NormalTok{(}\StringTok{"\textasciitilde{}/iDiv Dropbox/Henrique Pereira/Teaching/Spatial Ecology/Spatial Ecology 2022/2\_Lab\_assignments"}\NormalTok{)}
\NormalTok{Florida }\OtherTok{\textless{}{-}} \FunctionTok{read.csv}\NormalTok{(}\StringTok{"Florida.csv"}\NormalTok{)}

\CommentTok{\# number of species for year 1970 and route 20}
\FunctionTok{tapply}\NormalTok{(Florida}\SpecialCharTok{$}\NormalTok{Abundance,Florida}\SpecialCharTok{$}\NormalTok{Route}\SpecialCharTok{==}\DecValTok{20} \SpecialCharTok{\&}\NormalTok{ Florida}\SpecialCharTok{$}\NormalTok{Year}\SpecialCharTok{==}\DecValTok{1970}\NormalTok{, length)}

\CommentTok{\# matrix with number of species per route and per year}
\NormalTok{out}\OtherTok{\textless{}{-}}\FunctionTok{tapply}\NormalTok{(Florida}\SpecialCharTok{$}\NormalTok{Abundance,}\FunctionTok{list}\NormalTok{(Florida}\SpecialCharTok{$}\NormalTok{Route,Florida}\SpecialCharTok{$}\NormalTok{Year), length)}

\FunctionTok{names}\NormalTok{(out[,}\DecValTok{1}\NormalTok{])}
\FunctionTok{plot}\NormalTok{(out[}\DecValTok{10}\NormalTok{,])}
\FunctionTok{plot}\NormalTok{(out[}\DecValTok{20}\NormalTok{,])}


\NormalTok{shannon}\OtherTok{\textless{}{-}}\ControlFlowTok{function}\NormalTok{(x)}
\NormalTok{\{}
\NormalTok{  p}\OtherTok{\textless{}{-}}\NormalTok{x}\SpecialCharTok{/}\FunctionTok{sum}\NormalTok{(x)}
  \SpecialCharTok{{-}} \FunctionTok{sum}\NormalTok{(p}\SpecialCharTok{*}\FunctionTok{log}\NormalTok{(p))}
\NormalTok{\}}

\NormalTok{out}\OtherTok{\textless{}{-}}\FunctionTok{tapply}\NormalTok{(Florida}\SpecialCharTok{$}\NormalTok{Abundance,}\FunctionTok{list}\NormalTok{(Florida}\SpecialCharTok{$}\NormalTok{Route,Florida}\SpecialCharTok{$}\NormalTok{Year), shannon)}
\FunctionTok{plot}\NormalTok{(out[}\DecValTok{10}\NormalTok{,])}

\FunctionTok{library}\NormalTok{(tidyverse)}

\CommentTok{\#our first pipe}
\NormalTok{x}\OtherTok{\textless{}{-}}\FunctionTok{rnorm}\NormalTok{(}\DecValTok{1000}\NormalTok{)}
\FunctionTok{hist}\NormalTok{(x)}

\FunctionTok{rnorm}\NormalTok{(}\DecValTok{1000}\NormalTok{) }\SpecialCharTok{\%\textgreater{}\%}\NormalTok{  hist}

\NormalTok{t}\OtherTok{\textless{}{-}}\DecValTok{1}\SpecialCharTok{:}\FunctionTok{ncol}\NormalTok{(out)}
\NormalTok{myreg}\OtherTok{\textless{}{-}}\FunctionTok{lm}\NormalTok{(out[}\DecValTok{10}\NormalTok{,]}\SpecialCharTok{\textasciitilde{}}\NormalTok{t)}
\FunctionTok{summary}\NormalTok{(myreg)}
\FunctionTok{plot}\NormalTok{(out[}\DecValTok{10}\NormalTok{,])}
\FunctionTok{abline}\NormalTok{(myreg)}

\FunctionTok{lm}\NormalTok{(out[}\DecValTok{10}\NormalTok{,]}\SpecialCharTok{\textasciitilde{}}\NormalTok{t) }\SpecialCharTok{\%\textgreater{}\%}\NormalTok{ summary }
\FunctionTok{plot}\NormalTok{(out[}\DecValTok{10}\NormalTok{,])}
\FunctionTok{lm}\NormalTok{(out[}\DecValTok{10}\NormalTok{,]}\SpecialCharTok{\textasciitilde{}}\NormalTok{t) }\SpecialCharTok{\%\textgreater{}\%}\NormalTok{ abline }

\CommentTok{\#ggplot}
\NormalTok{mat}\OtherTok{=}\FunctionTok{cbind}\NormalTok{(t,out[}\DecValTok{10}\NormalTok{,])}
\FunctionTok{data}\NormalTok{(cars)}
\FunctionTok{colnames}\NormalTok{(mat)}\OtherTok{\textless{}{-}}\FunctionTok{c}\NormalTok{(}\StringTok{"time"}\NormalTok{,}\StringTok{"shannon"}\NormalTok{)}
\NormalTok{mat}\OtherTok{\textless{}{-}}\FunctionTok{as.data.frame}\NormalTok{(mat)}

\NormalTok{myplot }\OtherTok{\textless{}{-}}  \FunctionTok{ggplot}\NormalTok{(mat, }\FunctionTok{aes}\NormalTok{(time,shannon))}\SpecialCharTok{+}
  \FunctionTok{geom\_point}\NormalTok{()}
\NormalTok{myplot}

\NormalTok{myplot }\OtherTok{\textless{}{-}}  \FunctionTok{ggplot}\NormalTok{(mat, }\FunctionTok{aes}\NormalTok{(time,shannon))}\SpecialCharTok{+}
  \FunctionTok{geom\_line}\NormalTok{()}
\NormalTok{myplot}

\FunctionTok{data}\NormalTok{(cars)}
\NormalTok{myplot }\OtherTok{\textless{}{-}}  \FunctionTok{ggplot}\NormalTok{(cars, }\FunctionTok{aes}\NormalTok{(speed,dist))}\SpecialCharTok{+}
  \FunctionTok{geom\_point}\NormalTok{()}\SpecialCharTok{+}\FunctionTok{geom\_line}\NormalTok{()}
\NormalTok{myplot}

\FunctionTok{data}\NormalTok{(cars)}
\NormalTok{myplot }\OtherTok{\textless{}{-}}  \FunctionTok{ggplot}\NormalTok{(cars, }\FunctionTok{aes}\NormalTok{(speed,dist))}\SpecialCharTok{+}
  \FunctionTok{geom\_point}\NormalTok{()}\SpecialCharTok{+}\FunctionTok{geom\_smooth}\NormalTok{(}\AttributeTok{method=}\StringTok{"lm"}\NormalTok{)}
\NormalTok{myplot}

\FunctionTok{data}\NormalTok{(cars)}
\NormalTok{myplot }\OtherTok{\textless{}{-}}  \FunctionTok{ggplot}\NormalTok{(cars, }\FunctionTok{aes}\NormalTok{(speed,dist))}\SpecialCharTok{+}
  \FunctionTok{geom\_point}\NormalTok{()}\SpecialCharTok{+}\FunctionTok{geom\_smooth}\NormalTok{(}\AttributeTok{method=}\StringTok{"lm"}\NormalTok{)}\SpecialCharTok{+}\FunctionTok{scale\_x\_log10}\NormalTok{()}\SpecialCharTok{+}\FunctionTok{scale\_y\_log10}\NormalTok{()}
\NormalTok{myplot}
\end{Highlighting}
\end{Shaded}

\chapter{Computational labs}\label{computational-labs}

\section{Climate space of an
ectotherm}\label{climate-space-of-an-ectotherm}

Solar radiation and convection are the two main pathways for animals
such as small lizards. In this case, the total heat flux (\emph{f}) into
the animal is given by the \textbf{heat flux equation}:

\[f = q\ –h*(b - a)\]

\begin{itemize}
\item
  \(q\) = solar radiation (cal/h)
\item
  \(h*(b - a)\) = heat loss through convection
\item
  \(b\) = body temperature of the animal (ºC)
\item
  \(a\) = air temperature (ºC)
\item
  \(h\) = convection heat transfer coefficient (cal/h/ ºC)
\end{itemize}

\begin{enumerate}
\def\labelenumi{\arabic{enumi}.}
\tightlist
\item
  \textbf{\emph{Equilibrium body temperature}}

  \begin{enumerate}
  \def\labelenumii{\alph{enumii}.}
  \tightlist
  \item
    Assuming that air temperature is 18ºC, and \textbf{h} = 50 cal/h,
    plot the lizard's equilibrium body temperature as a function of
    solar radiation. Assume that solar radiation varies between 0 and
    1500 cal/h
  \end{enumerate}
\item
  \textbf{\emph{Climate space of a Lizard}}

  \begin{enumerate}
  \def\labelenumii{\alph{enumii}.}
  \item
    Draw the climate space of the lizard by drawing the polygon that is
    limited by the minimum and maximum temperatures that the lizard can
    support, as a function of solar radiation. Consider that the upper
    lethal limit for the body temperature (\textbf{\emph{bmax}}) is 36ºC
    and the lower lethal limit for the body temperature
    (\textbf{\emph{bmin}}) is 24ºC.

    \textbf{Hint:} \texttt{polygon} - draws the polygons whose vertices
    are given in x and y.
  \end{enumerate}
\item
  \textbf{\emph{Air Temperature and solar radiation throughout the day
  in two locations}}

  \begin{enumerate}
  \def\labelenumii{\alph{enumii}.}
  \item
    Plot the following values of air temperature and solar radiation
    that are available throughout a day in two locations. For:

    \begin{itemize}
    \item
      Times of the day

\begin{Shaded}
\begin{Highlighting}[]
\NormalTok{t }\OtherTok{=} \FunctionTok{c}\NormalTok{(}\StringTok{"00:00"}\NormalTok{,}\StringTok{"03:00"}\NormalTok{,}\StringTok{"06:00"}\NormalTok{,}\StringTok{"09:00"}\NormalTok{,}\StringTok{"12:00"}\NormalTok{,}
  \StringTok{"15:00"}\NormalTok{,}\StringTok{"18:00"}\NormalTok{,}\StringTok{"21:00"}\NormalTok{,}\StringTok{"00:00"}\NormalTok{)}
\end{Highlighting}
\end{Shaded}
    \item
      Solar radiation in the rock habitat

\begin{Shaded}
\begin{Highlighting}[]
\NormalTok{rock\_q }\OtherTok{=} \FunctionTok{c}\NormalTok{(}\DecValTok{150}\NormalTok{,}\DecValTok{150}\NormalTok{,}\DecValTok{800}\NormalTok{,}\DecValTok{1100}\NormalTok{,}\DecValTok{1300}\NormalTok{,}\DecValTok{1200}\NormalTok{,}\DecValTok{800}\NormalTok{,}\DecValTok{400}\NormalTok{,}\DecValTok{150}\NormalTok{)}
\end{Highlighting}
\end{Shaded}
    \item
      Air temperature in the rock habitat

\begin{Shaded}
\begin{Highlighting}[]
\NormalTok{rock\_a }\OtherTok{=} \FunctionTok{c}\NormalTok{(}\DecValTok{18}\NormalTok{,}\DecValTok{13}\NormalTok{,}\DecValTok{10}\NormalTok{,}\DecValTok{14}\NormalTok{,}\DecValTok{21}\NormalTok{,}\DecValTok{24}\NormalTok{,}\DecValTok{22}\NormalTok{,}\DecValTok{20}\NormalTok{,}\DecValTok{18}\NormalTok{)}
\end{Highlighting}
\end{Shaded}
    \item
      Solar radiation in the bush habitat

\begin{Shaded}
\begin{Highlighting}[]
\NormalTok{bush\_q }\OtherTok{=} \FunctionTok{c}\NormalTok{(}\DecValTok{150}\NormalTok{,}\DecValTok{150}\NormalTok{,}\DecValTok{450}\NormalTok{,}\DecValTok{600}\NormalTok{,}\DecValTok{650}\NormalTok{,}\DecValTok{650}\NormalTok{,}\DecValTok{350}\NormalTok{,}\DecValTok{200}\NormalTok{,}\DecValTok{150}\NormalTok{)}
\end{Highlighting}
\end{Shaded}
    \item
      Air temperature in the bush habitat

\begin{Shaded}
\begin{Highlighting}[]
\NormalTok{bush\_a }\OtherTok{=} \FunctionTok{c}\NormalTok{(}\DecValTok{18}\NormalTok{,}\DecValTok{13}\NormalTok{,}\DecValTok{10}\NormalTok{,}\DecValTok{14}\NormalTok{,}\DecValTok{21}\NormalTok{,}\DecValTok{24}\NormalTok{,}\DecValTok{22}\NormalTok{,}\DecValTok{20}\NormalTok{,}\DecValTok{18}\NormalTok{)}
\end{Highlighting}
\end{Shaded}
    \end{itemize}
  \item
    At which time of the day is the lizard on the rock, and at which
    time is it at the bush? Is there any time when the lizard cannot be
    at any of the locations?
  \end{enumerate}
\item
  \textbf{\emph{Body temperature of A. cristatellus in two different
  habitats}}
\end{enumerate}

Section~\ref{sec-anolis_cristatellus} presents a data frame containing
data on individuals \emph{A. cristatellus} living in urban and forested
areas in three cities in Puerto Rico, giving us the perfect opportunity
to assess the potential thermal effects of habitat upon this species. To
do this:

\begin{enumerate}
\def\labelenumi{\alph{enumi}.}
\tightlist
\item
  Make a linear model for individuals of \emph{A. cristatellus} living
  only in natural habitats (\texttt{context} = \texttt{natural}) and
  another for individuals living in urban habitats (\texttt{context} =
  \texttt{urban}).
\end{enumerate}

\textbf{Hint:} to filter the data you can use the \texttt{subset}
function in R

\begin{Shaded}
\begin{Highlighting}[]
\CommentTok{\# example filtering data from anoles only from San Jose sites}
\NormalTok{df\_sanjose }\OtherTok{\textless{}{-}} \FunctionTok{subset}\NormalTok{(df, Site }\SpecialCharTok{==} \StringTok{"San Jose"}\NormalTok{)}
\end{Highlighting}
\end{Shaded}

\begin{enumerate}
\def\labelenumi{\alph{enumi}.}
\setcounter{enumi}{1}
\item
  Is there a habitat where lizards are consistently warmer? If so,
  provide an ecological explanation (quantitatively) for this
  observation (assume an ambient temperature of 30 °C).
\item
  Is your model appropriate for explaining your data?
\end{enumerate}

\newpage{}

\section{Optimal foraging theory}\label{optimal-foraging-theory}

\begin{enumerate}
\def\labelenumi{\arabic{enumi}.}
\item
  \textbf{Predator with a strategy for maximizing energy.}\\
  Let \textbf{\emph{ai}} be the point abundance per unit of time of prey
  type \textbf{\emph{i}}, \textbf{\emph{ei}} be the caloric content of
  prey type \textbf{\emph{i}}, \textbf{\emph{hi}} be the time it takes
  to consume prey type \textbf{\emph{i}}, \textbf{\emph{ew}} is the
  energy expended per unit of time by the predator while searching for
  prey, and \textbf{\emph{eh}} is the energy expended per unit of time
  by the predator to ingest prey. Consider that there are only two types
  of prey and that:

  Strategy 1: Chosing prey of type 1\\
  Strategy 2: Chosing prey of type 2\\
  Strategy 3: Chosing both preys

  \begin{enumerate}
  \def\labelenumii{\alph{enumii}.}
  \item
    Show that if \textbf{\emph{e1\textgreater{} e2}} and
    \textbf{\emph{h1 \textless h2}} then the second strategy is never a
    optimal strategy
  \item
    Plot and comment the graph of the energy gain per unit of time for
    the strategy of consuming both preys and for the strategy of
    consuming only the best prey. Make the abundances vary between 0.005
    and 0.5 ind/s for the best prey. Consider the abundance of the worse
    prey to be 0.05 ind/s and 0.01 ind/s. Consider an active predator
    with the following physiological parameters:

    \textbf{\emph{e1}} = 10J; \textbf{\emph{e2}} = 100J;
    \textbf{\emph{h1}} = 1s; \textbf{\emph{h2}} = 60s;
    \textbf{\emph{ew}} = 1 J/s; \textbf{\emph{eh}} = 1 J/S
  \end{enumerate}
\item
  \textbf{Predator with a strategy of sit-and-wait.}\\
  \textbf{\emph{Ew}} is the energy expended per unit of time by the
  predator while waiting for its prey, \textbf{\emph{ep}} is the energy
  expended per unit of time by the predator while chasing the prey, v is
  the velocity of the predator in chase and \textbf{\emph{a}} is the
  point abundance of preys per unit of time.

  \begin{enumerate}
  \def\labelenumii{\alph{enumii}.}
  \tightlist
  \item
    Produce a graph of energy gain per unit of time in function of the
    size of the feeding territory. Consider the following values for the
    parameters of the predator and prey: \textbf{\emph{ew}} = 0,1 J/s;
    \textbf{\emph{ep}} = 1 J/S; \textbf{\emph{v}} = 0,5 m/s;
    \textbf{\emph{a}} = 0,005 ind/s/m2; and \textbf{\emph{e}} = 10;
  \end{enumerate}
\end{enumerate}

\newpage{}

\section{Evolutionary games}\label{evolutionary-games}

\begin{enumerate}
\def\labelenumi{\arabic{enumi}.}
\tightlist
\item
  \textbf{The Prisoners Dilemma Competition}. In this class we will have
  a competition of algorithms for playing the prisoner's dilemma.

  \begin{enumerate}
  \def\labelenumii{\alph{enumii})}
  \tightlist
  \item
    Write a function that takes as parameters two vectors that account
    with the history of the previous plays and returns one play (i.e.,
    Cooperate -- ``C'' or defect -- ``D''). This play should try to be
    the best response to the history of past moves. One example of such
    a function is:
  \end{enumerate}
\end{enumerate}

\begin{Shaded}
\begin{Highlighting}[]
\NormalTok{    strat}\OtherTok{\textless{}{-}}\ControlFlowTok{function}\NormalTok{(own,opponent)}
\NormalTok{    \{}
\NormalTok{        n }\OtherTok{\textless{}{-}} \FunctionTok{length}\NormalTok{(opponent)}
             \ControlFlowTok{if}\NormalTok{(n}\SpecialCharTok{==}\DecValTok{0}\NormalTok{) }\StringTok{"D"}
            \ControlFlowTok{else}
                \ControlFlowTok{if}\NormalTok{(own[n]}\SpecialCharTok{==}\StringTok{"D"}\NormalTok{) opponent[n]}
               \ControlFlowTok{else} \StringTok{"C"}
\NormalTok{    \}}
\end{Highlighting}
\end{Shaded}

\begin{enumerate}
\def\labelenumi{\alph{enumi})}
\setcounter{enumi}{1}
\item
  Use the function pd\_sim (see below) to play your strategy with the
  strategy of the other groups (ask for each group their function(s))
  and with itself, in an iterative game with x moves. Compare the
  results of the cumulative rewards as well as the sequence of moves.
  Report to the rest of groups the values of the rewards for each of the
  tournaments. What was the best strategy? What strategies are best
  responses to themselves?

\begin{Shaded}
\begin{Highlighting}[]
\NormalTok{pd\_sim}\OtherTok{\textless{}{-}}\ControlFlowTok{function}\NormalTok{(p1\_strat,p2\_strat,n)}
\NormalTok{\{}
\NormalTok{ w1}\OtherTok{\textless{}{-}}\DecValTok{0}          \CommentTok{\#accumulated pay{-}off (fitness) of player 1}
\NormalTok{ w2}\OtherTok{\textless{}{-}}\DecValTok{0}          \CommentTok{\#accumulated pay{-}offs of player 2}
\NormalTok{ h1}\OtherTok{\textless{}{-}}\ConstantTok{NULL}       \CommentTok{\#history of plays of player 1}
\NormalTok{ h2}\OtherTok{\textless{}{-}}\ConstantTok{NULL}       \CommentTok{\#history of plays of player 2}

 \ControlFlowTok{for}\NormalTok{ (t }\ControlFlowTok{in} \DecValTok{1}\SpecialCharTok{:}\NormalTok{n)            }
\NormalTok{   \{}
\NormalTok{    a1}\OtherTok{\textless{}{-}}\FunctionTok{p1\_strat}\NormalTok{(h1,h2) }
\NormalTok{    a2}\OtherTok{\textless{}{-}}\FunctionTok{p2\_strat}\NormalTok{(h2,h1) }
\NormalTok{    p1}\OtherTok{\textless{}{-}}\NormalTok{mat[a1,a2] }
\NormalTok{    p2}\OtherTok{\textless{}{-}}\NormalTok{mat[a2,a1]         }
\NormalTok{    w1}\OtherTok{\textless{}{-}}\NormalTok{w1}\SpecialCharTok{+}\NormalTok{p1              }
\NormalTok{    w2}\OtherTok{\textless{}{-}}\NormalTok{w2}\SpecialCharTok{+}\NormalTok{p2             }
\NormalTok{    h1[t]}\OtherTok{\textless{}{-}}\NormalTok{a1                   }
\NormalTok{h2[t]}\OtherTok{\textless{}{-}}\NormalTok{a2                   \}   }
 \FunctionTok{list}\NormalTok{(}\AttributeTok{w1=}\NormalTok{w1,}\AttributeTok{w2=}\NormalTok{w2,}\AttributeTok{h1=}\NormalTok{h1,}\AttributeTok{h2=}\NormalTok{h2)}
\NormalTok{\}}
\end{Highlighting}
\end{Shaded}
\end{enumerate}

\newpage{}

\section{Dispersal and the
random-walk}\label{dispersal-and-the-random-walk}

\begin{enumerate}
\def\labelenumi{\arabic{enumi}.}
\tightlist
\item
  \textbf{Random walk for 1 individual}

  \begin{enumerate}
  \def\labelenumii{\alph{enumii})}
  \tightlist
  \item
    Create your own code to simulate an individual random walk. Assume
    that the starting point is always x,y =(0,0) and the probability of
    an individual to choose any direction (i.e.~left, right, up or down)
    is the same. It should be a function taking as argument the number
    of steps and returning a list of two vectors, one with the x
    positions over time and another with the y positions overtime.
  \item
    Plot together the random walks (at least 5000 steps) of a few
    individuals. \includegraphics{images/Lab_RandomWalk.png}
  \end{enumerate}
\item
  \textbf{Random walk for several (n) individuals}

  \begin{enumerate}
  \def\labelenumii{\alph{enumii})}
  \tightlist
  \item
    Create your own code to simulate random walks by several
    (\textbf{\emph{n}})individuals and returns the last position of each
    individual. It should be a function that takes as arguments the
    number of individuals and the numbers of time steps, and returns a
    list of two vectors, the last x position of each individual, and the
    last y position of each individual.
  \item
    Create a histogram showing the distribution of the x and y last
    positions of 10 000 individuals after 10 time steps, 100 time steps
    and 1 000 time steps.
  \item
    Create a function that receives as parameters a vector of values
    xlast, a mean value (meanx) and a standard deviation (stdx) and
    returns the log-likelihood of observing those values for those
    parameters.
  \item
    Find the value meanx and stdx that maximize the likelihood of the
    observations. Are they the same as mean(xlast) and std(xlast)? Why?
  \item
    \textbf{Extra credit:} Create a histogram showing the distribution
    of the distance to the origin (sqrt(x\^{}+y\^{}2)) of the last
    positions of 10 000 individuals for 10 time steps, 100 time steps,
    1000 time steps and 10000 time steps. What is the relationship
    between the length of the randomwalk and the mean distance? Fit the
    Rayleigh distribution (2 r / sigma\^{}2) * exp(-r\^{}2/2 sigma\^{}2)
    using non-linear fitting to each of the histograms and overlay it on
    the graph.
  \end{enumerate}
\end{enumerate}

\newpage{}

\section{Pandemic growth}\label{pandemic-growth}

\begin{enumerate}
\def\labelenumi{\arabic{enumi}.}
\item
  \textbf{The exponential dynamics of the pandemic.} In early 2020, the
  first wave of the global pandemic of COVID-19 hit several European
  countries. Here we will plot the data for those countries and fit an
  exponential growth model.

  \begin{enumerate}
  \def\labelenumii{\alph{enumii}.}
  \tightlist
  \item
    Load the cumulative infected individuals time series from the
    COVID-19 pandemic in five countries. Each time series starts after
    the first fifteen infections are detected.\footnote{Data source:
      \url{https://github.com/owid/covid-19-data/tree/master/public/data},
      accessed 8 Nov 2020}
  \end{enumerate}

\begin{Shaded}
\begin{Highlighting}[]
\NormalTok{it }\OtherTok{\textless{}{-}} \FunctionTok{c}\NormalTok{(}\DecValTok{17}\NormalTok{, }\DecValTok{79}\NormalTok{, }\DecValTok{132}\NormalTok{, }\DecValTok{229}\NormalTok{, }\DecValTok{322}\NormalTok{, }\DecValTok{400}\NormalTok{, }\DecValTok{650}\NormalTok{, }\DecValTok{888}\NormalTok{, }\DecValTok{1128}\NormalTok{, }\DecValTok{1689}\NormalTok{, }\DecValTok{2036}\NormalTok{, }\DecValTok{2502}\NormalTok{, }
\DecValTok{3089}\NormalTok{, }\DecValTok{3858}\NormalTok{, }\DecValTok{4636}\NormalTok{, }\DecValTok{5883}\NormalTok{, }\DecValTok{7375}\NormalTok{, }\DecValTok{9172}\NormalTok{, }\DecValTok{10149}\NormalTok{, }\DecValTok{12462}\NormalTok{, }\DecValTok{15113}\NormalTok{, }\DecValTok{17660}\NormalTok{, }\DecValTok{21157}\NormalTok{,}
\DecValTok{23980}\NormalTok{, }\DecValTok{27980}\NormalTok{, }\DecValTok{31506}\NormalTok{, }\DecValTok{35713}\NormalTok{, }\DecValTok{41035}\NormalTok{, }\DecValTok{47021}\NormalTok{, }\DecValTok{53578}\NormalTok{) }\CommentTok{\#Feb22{-}March22}

\NormalTok{es }\OtherTok{\textless{}{-}} \FunctionTok{c}\NormalTok{(}\DecValTok{17}\NormalTok{, }\DecValTok{35}\NormalTok{, }\DecValTok{54}\NormalTok{, }\DecValTok{82}\NormalTok{, }\DecValTok{136}\NormalTok{, }\DecValTok{192}\NormalTok{, }\DecValTok{267}\NormalTok{, }\DecValTok{348}\NormalTok{, }\DecValTok{531}\NormalTok{, }\DecValTok{764}\NormalTok{, }\DecValTok{1094}\NormalTok{, }\DecValTok{1527}\NormalTok{, }\DecValTok{2299}\NormalTok{,}
\DecValTok{3274}\NormalTok{, }\DecValTok{4427}\NormalTok{, }\DecValTok{5958}\NormalTok{, }\DecValTok{7641}\NormalTok{, }\DecValTok{9785}\NormalTok{, }\DecValTok{11491}\NormalTok{, }\DecValTok{13994}\NormalTok{, }\DecValTok{17688}\NormalTok{, }\DecValTok{21735}\NormalTok{, }\DecValTok{26304}\NormalTok{, }\DecValTok{31750}\NormalTok{,}
\DecValTok{36616}\NormalTok{, }\DecValTok{41262}\NormalTok{, }\DecValTok{48953}\NormalTok{, }\DecValTok{57506}\NormalTok{, }\DecValTok{66460}\NormalTok{, }\DecValTok{75641}\NormalTok{) }\CommentTok{\#Feb27{-}March27}

\NormalTok{fr }\OtherTok{\textless{}{-}} \FunctionTok{c}\NormalTok{(}\DecValTok{17}\NormalTok{, }\DecValTok{38}\NormalTok{, }\DecValTok{57}\NormalTok{, }\DecValTok{100}\NormalTok{, }\DecValTok{130}\NormalTok{, }\DecValTok{178}\NormalTok{, }\DecValTok{212}\NormalTok{, }\DecValTok{285}\NormalTok{, }\DecValTok{423}\NormalTok{, }\DecValTok{613}\NormalTok{, }\DecValTok{716}\NormalTok{, }\DecValTok{1126}\NormalTok{, }\DecValTok{1412}\NormalTok{,}
\DecValTok{1784}\NormalTok{, }\DecValTok{2281}\NormalTok{, }\DecValTok{2876}\NormalTok{, }\DecValTok{3661}\NormalTok{, }\DecValTok{4499}\NormalTok{, }\DecValTok{5423}\NormalTok{, }\DecValTok{6633}\NormalTok{, }\DecValTok{7730}\NormalTok{, }\DecValTok{9134}\NormalTok{, }\DecValTok{10995}\NormalTok{, }\DecValTok{12612}\NormalTok{, }\DecValTok{14459}\NormalTok{, }
\DecValTok{16018}\NormalTok{, }\DecValTok{19856}\NormalTok{, }\DecValTok{22302}\NormalTok{, }\DecValTok{25233}\NormalTok{, }\DecValTok{29155}\NormalTok{) }\CommentTok{\#Feb27{-}March27}

\NormalTok{uk }\OtherTok{\textless{}{-}} \FunctionTok{c}\NormalTok{(}\DecValTok{18}\NormalTok{, }\DecValTok{22}\NormalTok{, }\DecValTok{30}\NormalTok{, }\DecValTok{42}\NormalTok{, }\DecValTok{47}\NormalTok{, }\DecValTok{69}\NormalTok{, }\DecValTok{109}\NormalTok{, }\DecValTok{164}\NormalTok{, }\DecValTok{220}\NormalTok{, }\DecValTok{271}\NormalTok{, }\DecValTok{352}\NormalTok{, }\DecValTok{412}\NormalTok{, }\DecValTok{469}\NormalTok{, }\DecValTok{617}\NormalTok{, }
\DecValTok{876}\NormalTok{, }\DecValTok{1282}\NormalTok{, }\DecValTok{1766}\NormalTok{, }\DecValTok{2244}\NormalTok{, }\DecValTok{2605}\NormalTok{, }\DecValTok{3047}\NormalTok{, }\DecValTok{3658}\NormalTok{, }\DecValTok{4427}\NormalTok{, }\DecValTok{5426}\NormalTok{, }\DecValTok{6481}\NormalTok{, }\DecValTok{7736}\NormalTok{, }\DecValTok{8934}\NormalTok{, }
\DecValTok{10312}\NormalTok{, }\DecValTok{12650}\NormalTok{, }\DecValTok{15025}\NormalTok{, }\DecValTok{17717}\NormalTok{) }\CommentTok{\#Feb27{-}March27}

\NormalTok{de }\OtherTok{\textless{}{-}} \FunctionTok{c}\NormalTok{(}\DecValTok{17}\NormalTok{, }\DecValTok{21}\NormalTok{, }\DecValTok{47}\NormalTok{, }\DecValTok{57}\NormalTok{, }\DecValTok{111}\NormalTok{, }\DecValTok{129}\NormalTok{, }\DecValTok{157}\NormalTok{, }\DecValTok{196}\NormalTok{, }\DecValTok{262}\NormalTok{, }\DecValTok{400}\NormalTok{, }\DecValTok{684}\NormalTok{, }\DecValTok{847}\NormalTok{, }\DecValTok{902}\NormalTok{, }\DecValTok{1139}\NormalTok{, }
\DecValTok{1296}\NormalTok{, }\DecValTok{1567}\NormalTok{, }\DecValTok{2369}\NormalTok{, }\DecValTok{3062}\NormalTok{, }\DecValTok{3795}\NormalTok{, }\DecValTok{4838}\NormalTok{, }\DecValTok{6012}\NormalTok{, }\DecValTok{7156}\NormalTok{, }\DecValTok{8198}\NormalTok{, }\DecValTok{14138}\NormalTok{, }\DecValTok{18187}\NormalTok{, }\DecValTok{21463}\NormalTok{, }
\DecValTok{24774}\NormalTok{, }\DecValTok{29212}\NormalTok{, }\DecValTok{31554}\NormalTok{, }\DecValTok{36508}\NormalTok{) }\CommentTok{\#Feb26{-}March26}
\end{Highlighting}
\end{Shaded}

  \begin{enumerate}
  \def\labelenumii{\alph{enumii}.}
  \setcounter{enumii}{2}
  \tightlist
  \item
    Plot the data in a linear plot, coloring each country with a
    different colour. What do you observe?
  \item
    Plot the data in a semi-log plot (using the plot option
    \texttt{log=”y”}), coloring each country with a different color.
    What do you observe?
  \item
    Carry out a linear regression with the data of each country (using
    the log of the n values) and estimate the growth rate R. Are the
    values the same for all countries? Do they vary over time?
  \item
    Knowing that \[R_0 = R^\tau\] and that the infectious period
    (\(\tau\)) duration is 10 days, what are the \(R_0\)'s in different
    countries?
  \item
    Write a for loop to simulate geometric (exponential) growth, going
    from time t in 1:100, and storing the population size values at each
    time step t+1 in variable n based on the population values at time
    t, i.e.~\texttt{n{[}t+1{]}\textless{}-R*n{[}t{]}}. Don't forget to
    initialize the population size before the for loop with
    \texttt{n\textless{}-1}. Plot a couple of runs of the for loop with
    different R values (for instance \texttt{R=1.01} and \texttt{R=1.1})
    in linear scale.
  \item
    Given that you know the solution of the geometric growth equation to
    be \(n(t)=n_0 R^t\), create a vector of values with this formula
    using the same R as you used above, but starting with
    \texttt{n0\textless{}-2} and overlay them on the plot.
  \end{enumerate}
\end{enumerate}

\newpage{}

\section{Population Viability
Analysis}\label{population-viability-analysis}

\includegraphics{images/IUCN_redlist.png}

\begin{enumerate}
\def\labelenumi{\arabic{enumi}.}
\item
  Consider a population with exponential growth but that exhibits
  environmental stochasticity, with the logarithm of the population
  growth rate following a normal distribution with mean
  \(\bar{r}=\log(\bar{R})\) and variance \(v\). Assume that the
  population has an on-off density dependence and cannot grow above the
  carrying capacity \(K\). According to Foley (1994)\footnote{Foley, P.
    (1994) Predicting Extinction Times from Environmental Stochasticity
    and Carrying-Capacity. \emph{Conservation Biology} \textbf{8}:
    124--137.} the expect time to extinction is
  \[T=\frac{1}{s r}[e^{s \log(k)}(1-e^{-s \log(N_0)} )-s \log(N_0)]\]
  where \(s=2r/v\).

  \begin{enumerate}
  \def\labelenumii{\alph{enumii}.}
  \item
    Plot the mean extinction times, \(T\), as a function of \(r\),
    \(v\), \(n_0\), and \(K\). Please comment each plot. \(r\) can carry
    from 0.01 to 0.2, \(v\) can vary from 0.05 to 1.0, \(n0\) can vary
    from 0 to 10, and \(K\) can vary from 10 to 500. Use as base
    parameters \texttt{n0=10}, \texttt{K=100}, \texttt{v=0.3} and
    \texttt{r=0.01}.
  \item
    Write a function that simulates a population with these dynamics
    numerically. Start by using the for loop that you developed in Lab 2
    and modify it to include a carrying capacity and growth rate that is
    taken every year from a normal distribution. The function should
    take parameter \texttt{n0}, \texttt{r}, \texttt{v}, and \texttt{K}.
    It should return the vector of the population sizes over time.
  \item
    Simulate the dynamics with the following parameters:

    \begin{enumerate}
    \def\labelenumiii{\arabic{enumiii}.}
    \tightlist
    \item
      \texttt{n0=3}; \texttt{r=0.01}; \texttt{v=0.2}; \texttt{K=500}
    \item
      \texttt{n0=100}; \texttt{r=0.01}; \texttt{v=0.2}; \texttt{K=500};
    \end{enumerate}
  \item
    \textbf{Extended credit}: Compare the model predictions with the
    results from the analytical approximation of Foley. You need to do
    many simulations to reach the predicted extinction times from Foley.
    For instance, you can create a function that call the function
    developed in (b) and executes it 100 times, returning the median
    time to extinction across the simulations.
  \end{enumerate}
\end{enumerate}

\newpage{}

\section{The camera trapper}\label{the-camera-trapper}

A researcher place camera traps at a grid of sites during 19 days to
observe roe-deer.

\begin{enumerate}
\def\labelenumi{\arabic{enumi}.}
\item
  Load the data on the file roe\_deer\_2016\_r.csv into R. Briefly
  describe the structure of the dataset.
\item
  Case 1: Ignoring detection probability.

  \begin{enumerate}
  \def\labelenumii{\alph{enumii})}
  \tightlist
  \item
    Build a vector that for each site takes value 1 when roe-deer is
    present and 0 when is absent.
  \item
    Using maximum-likelihood estimate the occupancy probability
    (\(\psi\)). First create a function that takes as parameters psi and
    a vector of presence/absences and returns the likelihood. Then plot
    that function for a range of \(\Psi\) values and find the \(\psi\)
    value that maximizes the function.
  \item
    Assume that based on previous work we know that occupancy is
    somewhat between 0.2 and 0.5. Using a Bayesian approach, calculate
    the posterior probability distribution for the occupancy
    \(P(\psi)\).
  \end{enumerate}

  Help: A function returning the \(P(\psi|data)\) . It takes as
  parameters a vector \texttt{y} of presences/absences, a value for
  \texttt{psi}, and a function for the prior distribution of \(\psi\)
  named \texttt{priorpsi}:

\begin{Shaded}
\begin{Highlighting}[]
\NormalTok{occupancybayesian }\OtherTok{\textless{}{-}} \ControlFlowTok{function}\NormalTok{(y,psi,priorpsi)}
\NormalTok{\{}
\NormalTok{  integrand }\OtherTok{\textless{}{-}} \ControlFlowTok{function}\NormalTok{(x)}
      \FunctionTok{occupancylikelihood}\NormalTok{(y,x)}\SpecialCharTok{*}\FunctionTok{priorpsi}\NormalTok{(x)}
  \FunctionTok{occupancylikelihood}\NormalTok{(y,psi)}\SpecialCharTok{*}\FunctionTok{priorpsi}\NormalTok{(psi)}\SpecialCharTok{/}
      \FunctionTok{integrate}\NormalTok{(integrand, }\AttributeTok{lower =} \FloatTok{0.01}\NormalTok{, }\AttributeTok{upper =} \FloatTok{0.99}\NormalTok{)}\SpecialCharTok{$}\NormalTok{value}
\NormalTok{\}}
\end{Highlighting}
\end{Shaded}

  A function returning \(P(\psi)\), the prior distribution of \(\psi\)
  values. It takes as parameters a value for \texttt{psi.}

\begin{Shaded}
\begin{Highlighting}[]
\NormalTok{prior }\OtherTok{\textless{}{-}} \ControlFlowTok{function}\NormalTok{(psi) }
\NormalTok{\{ }
  \FunctionTok{dunif}\NormalTok{(psi,}\FloatTok{0.2}\NormalTok{,}\FloatTok{0.5}\NormalTok{) }
\NormalTok{\}}
\end{Highlighting}
\end{Shaded}
\item
  Case 2: Using a hierarchical model with detection probability

  \begin{enumerate}
  \def\labelenumii{\alph{enumii}.}
  \tightlist
  \item
    Using maximum-likelihood estimate the occupancy probability
    (\(\Psi\)) and detection probability (\(p\)).
  \end{enumerate}
\end{enumerate}

\includegraphics{images/RoeDeer.jpg}

\newpage{}

\section{Managing a fishery}\label{managing-a-fishery}

\begin{enumerate}
\def\labelenumi{\arabic{enumi}.}
\tightlist
\item
  Consider that you are managing a fishery with the following dynamics:
  \[dn/dt = 0.2 n (1-n/10000) \]

  \begin{enumerate}
  \def\labelenumii{\alph{enumii})}
  \tightlist
  \item
    Plot the production function of the fishery, indicating the stock
    size for the maximum sustainable yield and the corresponding
    production level.
  \item
    Suppose you establish an annual quota (harvest) that is equivalent
    to 50\% of the maximum sustainable yield. What are the two stock
    size levels that sustainably allow that exploitation level?
  \item
    Explore what happen when you manage this fishery for 100 years.
    Start by writing a function that simulates the logistic growth above
    with a constant harvest. The function should take as parameters the
    initial population size n0 (the stock of the fishery), the annual
    harvest h, and the number of years for which you simulate the
    population dynamics and return a vector with the population sizes
    over time. Discuss the results of the following experiments.

    \begin{itemize}
    \tightlist
    \item
      \textbf{Experiment 1:} Set the harvest equal to the production at
      MSY and initial population size at carrying capacity.
    \item
      \textbf{Experiment 2}. Set the harvest equal to the production at
      MSY and initial population size at the stock size of MSY
    \item
      \textbf{Experiment 3.} Set the harvest equal to 50\% of the
      maximum sustainable yield and initial population size at carrying
      capacity.
    \item
      \textbf{Experiment 4.} Set the harvest equal to the production at
      MSY and initial population size at the stock size of MSY - 1000
      individuals.
    \item
      \textbf{Experiment 5.} Set the harvest equal to the production at
      MSY and initial population size at the stock size of MSY - 1000
      individuals.
    \end{itemize}
  \item
    Explain why in a system with constant quotas, MSY is not stable and
    should not be the management goal.
  \item
    \textbf{Extra credit}: Modify the logistic growth function to have
    stochastic dynamics so that there is a variance of 10\% in the
    annual productivity. Simulate again experiment 2 a few times. What
    happens?
  \item
    Discuss how these results relate with the paper of Worm et al (2009)
    about the need to rebuild fisheries.
  \end{enumerate}
\end{enumerate}

\begin{figure}[H]

{\centering \includegraphics{images/Gadus_morhua.JPG}

}

\caption{Atlantic cod (Gadus morhua). Source: Hans-Petter Fjeld,
CC-BY-SA, Wikipedia.}

\end{figure}%

\newpage{}

\section{Road Kill}\label{road-kill}

\textbf{Populations in space and the impacts of roads}

\begin{longtable}[]{@{}
  >{\raggedright\arraybackslash}p{(\columnwidth - 2\tabcolsep) * \real{0.4911}}
  >{\raggedright\arraybackslash}p{(\columnwidth - 2\tabcolsep) * \real{0.5089}}@{}}
\toprule\noalign{}
\endhead
\bottomrule\noalign{}
\endlastfoot
\emph{The \textbf{Red fox}}

\emph{\ldots So, while they may be most active at night.}

\emph{Red foxes are especially active during the}

\emph{daytime in \textbf{spring and summer (April-Sept)}}

\emph{as they are foraging for food to feed their young.}

Source photo: Wikipedia, CC0. & \hfill
\includegraphics[width=3.59375in,height=\textheight]{images/RedFox.jpg} \\
\end{longtable}

\textbf{Goals of the exercise}:

\begin{itemize}
\item
  Identify road segments with hotspots of red fox-car-collisions.
\item
  What impact do activity patterns have on the rate of
  animal-car-collision hotspots?
\item
  Use the False Discovery Rate (FDR) method to identify such hotspots.
\end{itemize}

\begin{enumerate}
\def\labelenumi{\arabic{enumi}.}
\tightlist
\item
  Please install the following R packages
\end{enumerate}

\begin{Shaded}
\begin{Highlighting}[]
\FunctionTok{library}\NormalTok{(readxl)}
\FunctionTok{library}\NormalTok{(dplyr)}
\FunctionTok{library}\NormalTok{(sf)}
\FunctionTok{library}\NormalTok{(ggplot2)}
\FunctionTok{library}\NormalTok{(fuzzySim)}
\FunctionTok{library}\NormalTok{(grid)}
\end{Highlighting}
\end{Shaded}

\begin{enumerate}
\def\labelenumi{\alph{enumi}.}
\tightlist
\item
  Load the \emph{roadkill} dataset.
\item
  Explain the structure of the dataset.
\item
  Show that differences in seasonal activity patterns impact the FDR
  hotspot results in red fox-car-collisions.
\item
  How many hotspots were detected for the active season?
\item
  Plot, based on the FDR method, true and ``false'' hotspots. Explain
  the general geographic locations of the hotspots and those vary
  between the two different seasons.
\item
  Think about conservation management implications.
\end{enumerate}

\newpage{}

\section{Simulating a neutral
community}\label{simulating-a-neutral-community}

The neutral theory in ecology provides a mechanistic explanation for
species abundance distributions. It seeks to understand the impact of
speciation, extinction, dispersal and ecological drift on the species
abundance distribution (SAD), assuming that all species have equal
opportunities (Hubbell, 2001).

It is important to note that the neutral theory is a model created to
explain a pattern of relative abundance within communities, but it does
not necessarily reflect reality (communities' mechanisms). The neutral
theory utilizes a model based on the dynamics of a species' population,
which is governed by generalized birth and death events, including
speciation, immigration, and emigration (Rosindell et al (2011).

\begin{figure}[H]

{\centering \includegraphics{images/neutral.png}

}

\caption{Diagram of the dynamics of communities in the neutral theory of
biodiversity (from Rosindell et al 2011).}

\end{figure}%

The neutral theory paradox is that in absence of migration or mutation
diversity gradually declines to zero or monodominance. Let's see what
happens to species richness and the species abundance distribution over
time for different parameters of the model. Start by installing the
package \texttt{untb.}

\begin{enumerate}
\def\labelenumi{\arabic{enumi}.}
\item
  \textbf{Without mutation or dispersal (pure ecological drift):} We
  start with a local community with 20 species, each with 25
  individuals. The simulation then runs for 2500 generations where 10
  individuals die per generation. Mutation rate is zero and immigration
  rate is zero

  \begin{enumerate}
  \def\labelenumii{\alph{enumii}.}
  \item
    Plot the number of species over time.
  \item
    Plot the species abundance distribution at time 1, 100 and 2500
  \item
    Plot the map of individuals at those time steps.
  \end{enumerate}
\item
  \textbf{With point mutation:} Same parameters as (1) but with
  speciation rate of 0.1.

  \begin{enumerate}
  \def\labelenumii{\alph{enumii}.}
  \item
    Plot the number of species over time.
  \item
    Plot the species abundance distribution at time 1, 100 and 2500
  \item
    Plot the map of individuals at those time steps.
  \end{enumerate}
\item
  \textbf{{[}Extra Credit{]} With immigration:} Same parameters as (1)
  but immigration rate greater than zero. Play with different abundance
  distributions for the metacommunity.
\end{enumerate}

\newpage{}

\section{Monitoring biodiversity}\label{monitoring-biodiversity}

\begin{enumerate}
\def\labelenumi{\arabic{enumi}.}
\tightlist
\item
  Exploring the BBS dataset

  \begin{enumerate}
  \def\labelenumii{\alph{enumii}.}
  \tightlist
  \item
    Load the file Florida.csv into R. Briefly describe the structure of
    the dataset.
  \item
    Map the monitored transects/routes in Florida's map. Use the maps
    library to plot the counties of florida as a base map.
  \item
    For transect 4 and transect 109 in year 2018 plot calculate the
    species richness, Shannon diversity index H and evenness J of both
    transects, with H being \[H=-\sum_i (p_i \ln p_i)\] where \(p_i\) is
    defined as the proportion of individuals found in species \(i\).
    Compare the two transects.
  \item
    Plot the species abundance distribution and abundance-rank for both
    transects. What do you observe?
  \item
    Choose one transect and plot the species richness, Shannon
    diversity, and geometric mean abundance over time
  \item
    \emph{Extra credit 1:} Produce a map of the trends of one these
    metrics across Florida (for instance coloring different points
    according to the trend)
  \item
    \emph{Extra credit 2:} Estimate the number of species in Florida by
    combining the different transects and using one of the estimators.
    Use bootstrap and/or jacknife to calculate confidence intervals.
  \end{enumerate}
\end{enumerate}

\newpage{}


\backmatter


\end{document}
